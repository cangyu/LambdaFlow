\documentclass[paper=a4, fontsize=11pt]{scrartcl}

\usepackage{lipsum}
\usepackage{amsmath, amsfonts, amssymb}

\title{Summary of "CFD modelling of combustion"}
\author{Yu Cang \ \ 018370210001}
\date{2019-03-29}

\begin{document}

\maketitle

Numerical simulation of gaseous combustions are brifely introduced in this article. Firstly, some basic concepts and definitions are reviewed. Then, classical models like infinitely fast chemistry, eddy break-up concept and laminar flamelet model are discussed. The most important model used nowadays is the laminar flamelet model, where flames are locally viewed as a steady flamelet as the timescale of chemical reaction is much smaller than that of flow. Finally, topics related to turbulent combustion are covered, where the difficulties faced currently are introduced. The lack of proper model to describe averaged chemical reaction source term poses difficulties in CFD practice.

	
\end{document}
