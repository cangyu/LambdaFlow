\documentclass[paper=a4, fontsize=11pt]{scrartcl}

\usepackage{lipsum}
\usepackage{amsmath, amsfonts, amssymb}

\title{Reparaphrase of "Analysis of the filtered non-premixed turbulent flame"}
\author{Yu Cang \ \ 018370210001}
\date{2019-04-19}

\begin{document}

\maketitle

\section{Original paragraph}
	In reactive turbulence the interaction between the flame and
	flow is extremely complex. For problems with sufficiently high Reynolds numbers of practical interest, because of the large scale separation, both spatially and temporally, direction numerical simulations (DNS) are intractable. Thus models of the required relations need to be constructed. In non-premixed combustion simulation, the flamelet models originally developed by Williams and Peters play important roles. Physically if the flame scale is locally small compared with the turbulent dissipative scale, the flame structure remains to be laminar. Based on the flamelet equations, it is possible to largely reduce the dimensionality of the modeling parameters. The flamelet models have been widely applied because of the clear physical meaning and high numerical efficiency. In recent years successful examples have been extended to various cases such as unsteady processes and non-unity Lewis number effects. The detailed budget analyses suggest that typically the flamelet assumption remains to be reasonable.

\section{Re-paraphrase}
	In reactive turbulence, interaction between flame and flow is extremely complex.For problem with sufficiently high Reynolds number, Direct Numerical Simulation(DNS) is intractable as there exists large scale separation, both spatially and temporally. To describle inner relations, properly constructed models are needed.As originally developed by Williams and Peters, the flamelet model play an important role in the simulation of non-premixed combustion. Physically, the local flame structure remains to be laminar when flame scale is smaller than turbulent dissipative scale locally. With the help of flamelet model, the dimensionality of modeling parameters can be largely reduced. Because of its clear physical meaning and high numerical efficiency, flamelet model has been widely adopted. Recelently, various cases like unsteady process and non-unity Lewis number effects have been extened to successfully, and detailed budget analysis indicates that the flamelet assumption remains reasonable.
	
\end{document}
