\documentclass[paper=a4, fontsize=11pt]{scrartcl} % A4 paper and 11pt font size

\usepackage[T1]{fontenc} % Use 8-bit encoding that has 256 glyphs
\usepackage{fourier} % Use the Adobe Utopia font for the document - comment this line to return to the LaTeX default
\usepackage[english]{babel} % English language/hyphenation
\usepackage{amsmath,amsfonts,amsthm,amssymb} % Math packages

\usepackage{algorithm, algorithmic}
\renewcommand{\algorithmicrequire}{\textbf{Input:}} %Use Input in the format of Algorithm  
\renewcommand{\algorithmicensure}{\textbf{Output:}} %UseOutput in the format of Algorithm  

\usepackage{graphicx}

\usepackage{listings}
\lstset{language=Matlab}

\usepackage{lipsum} % Used for inserting dummy 'Lorem ipsum' text into the template

\usepackage{sectsty} % Allows customizing section commands
\allsectionsfont{\centering \normalfont\scshape} % Make all sections centered, the default font and small caps

\usepackage{fancyhdr} % Custom headers and footers
\pagestyle{fancyplain} % Makes all pages in the document conform to the custom headers and footers
\fancyhead{} % No page header - if you want one, create it in the same way as the footers below
\fancyfoot[L]{} % Empty left footer
\fancyfoot[C]{} % Empty center footer
\fancyfoot[R]{\thepage} % Page numbering for right footer
\renewcommand{\headrulewidth}{0pt} % Remove header underlines
\renewcommand{\footrulewidth}{0pt} % Remove footer underlines
\setlength{\headheight}{13.6pt} % Customize the height of the header

\numberwithin{equation}{section} % Number equations within sections (i.e. 1.1, 1.2, 2.1, 2.2 instead of 1, 2, 3, 4)
\numberwithin{figure}{section} % Number figures within sections (i.e. 1.1, 1.2, 2.1, 2.2 instead of 1, 2, 3, 4)
\numberwithin{table}{section} % Number tables within sections (i.e. 1.1, 1.2, 2.1, 2.2 instead of 1, 2, 3, 4)

\setlength\parindent{0pt} % Removes all indentation from paragraphs - comment this line for an assignment with lots of text

\newcommand{\horrule}[1]{\rule{\linewidth}{#1}} % Create horizontal rule command with 1 argument of height

\title{	
\normalfont \normalsize 
\textsc{Shanghai Jiao Tong University, UM-SJTU JOINT INSTITUTE} \\ [25pt] % Your university, school and/or department name(s)
\horrule{0.5pt} \\[0.4cm] % Thin top horizontal rule
\huge Technical Communication\\ HW0 \\ % The assignment title
\horrule{2pt} \\[0.5cm] % Thick bottom horizontal rule
}

\author{Author: Yu Cang \quad 018370210001\\ Partner: Guiwen Tan \quad 118370910014} % Your name

\date{\normalsize \today} % Today's date or a custom date

\begin{document}

\maketitle % Print the title

\section{Summary of Yu Cang's  research project}
	Currently, my research focus on numerical simulation of the one-dimensional  counter-flow diffusion flame.\\
	Numerical schemes play an important role in solving this problem as proper schemes have better convergence performance. \\
	Generally speaking, a hybrid scheme consists of the damped Newton method and the semi-implicit time stepping scheme is used.\\
	At the beginning, I try to solve the steady-state equation set with the damped Newton method, once the newton iteration fails, it turns to solve the time-dependent equation set. In this way,  a new starting estimation can be provided for the next trial of the newton method. 

\section{Paraphrase of Guiwen Tan's work}
	A wide range of spatial and temporal scales exist in a turbulence field and they interact with each other.\\
	The spatial structure at different scales is the focus of his reseach. In order to show the spatial structures at different levels, the multi-level segment analysis(MSA) method is adopted.\\
	At present, he uses the sinusoidal function, fractional Brownain motion, and DNS result of real turbulence field to verify the MSA method. The statistical characteristics of each testing case indicates if the MSA method is justified.

\section{Improved Version of Yu Cang's work}
	The configuration of one-dimensional counter-fow diffusion flame involves a burning chamber and two opposing placed inlents, where fuel and oxdizer are pumped into the chamber. A flame front burning in between is sustained, and that is counter-flow diffusion flame.\\
	My research mainly focus on the numerical simulation of the one-dimensional counter-flow diffusion flame. Numerical method should be choosen carefully as some of schemes may converge faster and better. \\
	Usually, I use the damped Newton method and the semi-implicit time stepping scheme in numerical simulation. At first, I use the damped Newton method to deal with steady-state equation set. Once the damped Newton method fails or diverges, I switch to the time-dependent equation set with the semi-implicit time stepping scheme. In this way,  results from the time-dependent equation set can provide a new starting estimation for the newton method of next step.

\end{document}