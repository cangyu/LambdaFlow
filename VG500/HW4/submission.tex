\documentclass[paper=a4, fontsize=11pt]{scrartcl} % A4 paper and 11pt font size

\usepackage[T1]{fontenc} % Use 8-bit encoding that has 256 glyphs
\usepackage{fourier} % Use the Adobe Utopia font for the document - comment this line to return to the LaTeX default
\usepackage[english]{babel} % English language/hyphenation
\usepackage{amsmath,amsfonts,amsthm,amssymb} % Math packages

\usepackage{algorithm, algorithmic}
\renewcommand{\algorithmicrequire}{\textbf{Input:}} %Use Input in the format of Algorithm  
\renewcommand{\algorithmicensure}{\textbf{Output:}} %UseOutput in the format of Algorithm  

\usepackage{graphicx}
\usepackage{blindtext}
\usepackage{enumerate}
\usepackage{ulem} 
\usepackage{pdfpages}
\usepackage{multirow}

\usepackage{listings}
\lstset{language=Matlab}

\usepackage{lipsum} % Used for inserting dummy 'Lorem ipsum' text into the template

\usepackage{sectsty} % Allows customizing section commands
\allsectionsfont{\centering \normalfont\scshape} % Make all sections centered, the default font and small caps

\usepackage{fancyhdr} % Custom headers and footers
\pagestyle{fancyplain} % Makes all pages in the document conform to the custom headers and footers
\fancyhead{} % No page header - if you want one, create it in the same way as the footers below
\fancyfoot[L]{} % Empty left footer
\fancyfoot[C]{} % Empty center footer
\fancyfoot[R]{\thepage} % Page numbering for right footer
\renewcommand{\headrulewidth}{0pt} % Remove header underlines
\renewcommand{\footrulewidth}{0pt} % Remove footer underlines
\setlength{\headheight}{13.6pt} % Customize the height of the header

\numberwithin{equation}{section} % Number equations within sections (i.e. 1.1, 1.2, 2.1, 2.2 instead of 1, 2, 3, 4)
\numberwithin{figure}{section} % Number figures within sections (i.e. 1.1, 1.2, 2.1, 2.2 instead of 1, 2, 3, 4)
\numberwithin{table}{section} % Number tables within sections (i.e. 1.1, 1.2, 2.1, 2.2 instead of 1, 2, 3, 4)

\setlength\parindent{0pt} % Removes all indentation from paragraphs - comment this line for an assignment with lots of text

\newcommand{\horrule}[1]{\rule{\linewidth}{#1}} % Create horizontal rule command with 1 argument of height
\newcommand*{\dif}{\mathop{}\!\mathrm{d}}

\title{	
\normalfont \normalsize 
\textsc{Shanghai Jiao Tong University, UM-SJTU JOINT INSTITUTE} \\ [25pt] % Your university, school and/or department name(s)
\horrule{0.5pt} \\[0.4cm] % Thin top horizontal rule
\huge Technical Communication\\ HW4 \\ % The assignment title
\horrule{2pt} \\[0.5cm] % Thick bottom horizontal rule
}

\author{Yu Cang \quad 018370210001} % Your name

\date{\normalsize \today} % Today's date or a custom date

\begin{document}

\maketitle % Print the title

\section{Writting skills}
	\begin{enumerate}
		\item 
			As shown in Fig. 1, due to stencil aperture dimension variations and stencil cleanliness, variations across the die arise. Also, smaller variations would also show up, due to random defects, such as inclusions in the paste and contamination from the wafer or environment.
		\item 
			\begin{enumerate}[(a)]
				\item 
					While these systems are very sensitive, they are very reliable. As a result, many banks, shops, hotels and other businesses use them, because their components are of the highest quality.
				\item 
					 We are holding the above cheque, but there are insufficient funds in your account to clear it. Originally, our normal practice would be to charge you 150 RMB. However, we will waive the charge this time as this is your first time.
				\item 
					 For inspiring innovation and excellence in our customers and users within a secure and sustainable integrated community, we will create and manage distinctive spaces, and comprise mixed-use developments anchored by business space.
			\end{enumerate}
		\item 
			\begin{enumerate}[(a)]
				\item 
					We can deal with more problems of this kind in such a way, and bring about satisfactory solutions.
				\item 
					Some companies self-generating their own websites is a tendency in the industry, and Internet makes training for the skill of website development more accessible to everyone, so this activity can more and more be handled in-house.
				\item 
					 Harold recommended us to fire Mr. Harrison.
				\item 
					We will ask him to think about changing his work routine.
				\item 
					Their correspondence expired when he learned that they no longer supported his proposal.
				\item 
					This document is used to identify the recovered objects.
			\end{enumerate}
	\end{enumerate}

\section{Group Exercise}
	I'm the group leader and members are Ning Kang(017370910016) and Reze Rouhi Ardeshiri(018370990002).
	
	\subsection{Yu Cang's summary}
		This section mainly talks about the general guidelines on graphics to be used in scientific writings. \newline
		Firstly, one should spare enough time to generate figures. Roughly, spend as much as that of writing text.\newline 
		Secondly, it is better to prepare an outline and decide what are the most important elements to conveyed before drawing pictures.\newline 
		Thirdly, either placed in stand-alone form or in inlined form, figures should be linked and referenced properly. Usually, one should provide a detailed caption, which illustrates the key point of the figure, and a complete reference that faciliates readers a lot. It's better not to use abbreviations like "Fig.", which may slow readers down. \newline
		Then, consistency between graphics and text should be maintained through out the whole article. For example, it is suggest to use figures of same sacles, take same line width, keep fonts in figures as same as that in text, and choose color coding according to that used in text etc. In practice, instead of using different graphics software, the best way to keep consistency is to use a single graphics program. \newline
		Also, labels play an important role in plots. Text in labels should be of the same font as that in the main text, and fonts between labels should also be consistent with each other. Notions should not be different, for example, $\frac{1}{2}$ vs $0.5$, $e^{-i\pi}$ vs $-1$. In addition, labels should be both clear and well placed, which provide a straight forward explanation.\newline
		Most importantly, issues concerning plots and charts should be carefully handled. 3D pie charts may introduce distortions and should be avoided whenever possible. Instead, table can be used to explain in detail numbers and percentages. It is recommended to use colors that can group things or direct readers' focus. As background patterns may distract and is likely to blur important messages, they should not be used as well. \newline
		Finally, "avoid distractions" and "steer attention" are useful concepts when designing figures. Distraction like strong contrast produced by sparse vs dense, white vs black should be avoided. As human eyes are much better at grouping things according to colors instead of line patterns, distraction originates from dashed or dot lines should also be avoided. Backgourd image is usually unnecessary as it convey little message.  
		
		\begin{itemize}
			\item \textbf{Comment by Ning Kang} \newline
				Yu's summary is very good, and his summary covers almost all the points of this section. In addition, he clearly described the precautions for figures. However, there are some grammatical problems in his summary. For example, the last sentence of the last paragraph is better written as "it conveys little information".
				
			\item \textbf{Comment by Reze Rouhi Ardeshiri} \newline
				It is clear that Yu has fully studied the section 7 and has an accurate understanding of the content talked in this section. That's why he was able to make a good and understandable summary. It also used good descriptive words to describe the text so that the reader is not tired with reading the text. Although there are weaknesses in grammatically in the text, such as “suggested” instead of “suggest”, “same font like” instead of “same font as”, in general, I can say that it was great. Well done!
				
		\end{itemize}
		
	\subsection{Reze Rouhi Ardeshiri's summary}
		The author of “The TikZ and PGF Packages” book, in Section 7, describes the graphics, which points to good and useful tips.\newline
		In the first step, to write a paper with a lot of figures, you need to spend a lot of time on each page to have high-quality work and for the design of a figure, at least we need to consider time as much as a text or even more so.\newline
		In the next step, we must decide what kind of figure to put in the paper so that the reader reaches the goal of what we want that and the reader understands the issue. That is, the output of the figure you want to draw is clear to the reader.\newline
		Also, placing the figure in the text and linking it to the text is one of the remarkable things that can be done in two ways: stand-alone and inline.\newline
		Additionally, if you want to design the figure, we need to have the same sizes, fonts, and lines in all of the figures and not be different from the other figures and for the integrity of the figures, it's better to use the same software to design the figure. In general, there must be consistency between graphics and text to be nice.\newline
		Label in graphics is one of the significant points in the paper. With a good label, you can summarize the description of a figure, also important for labeling should be important points to be readable. It’s implying that it should be legible.\newline
		One of the most important and frequent issues in the papers are plots that can play an important role in understanding the problem and the reports. In this case, you should not use 3D pie or 3D bar charts because of the lack of realistic values and percentages. That’s why the table is proper replacing for providing the results.\newline
		Specifying the paper format is one of the things considered in graphics. It should not be large headlines, bold text, large, and white areas. Also, some of the words that the author has to emphasize should be illustrated with Italic font. So the reader will not be confused by reading the paper. In a word, avoid distractions!
		
		\begin{itemize}
			\item \textbf{Comment by Ning Kang} \newline
				Reza's summary is very good, and his summary contains the main content of this section. Here are two questions from his summary. First of all, his summary lacks a general description, and some paragraphs only mention the precautions. In addition, some paragraphs lack a description of specific considerations, such as his summary of the third part.
			
			\item \textbf{Comment by Yu Cang} \newline
				Obviously, Reze Rouhi Ardeshiri has read the paper thoroughly. Main ideas are conveyed with clear explanations. For each aspect, both reasons and understandings are well described. Examples are provided so that comparison can be imagined. In organizing all paragraphs, transitional words are used smoothly, which facilitate readers to grasp the structure quickly. But in terms of pronouns used, I think it is not good to use “you”, instead, “one” is much better. No spelling mistakes are found. In general, it is a wonderful summary.
			
		\end{itemize}
	
	\subsection{Ning Kang's summary}
		This section focuses on the typical good practices and bad practices of the figures used in scientific papers and presents general guidelines. \newline
		First, the authors should use enough time to generate figures. A typical mistake is to write text in a lot of time and to generate the figures in a very little time. In general, the cost of time in figures should be as much as writing text.\newline
		Second, when creating a paper, the authors should firstly write a rough outline, then fill the text to create a draft and continually revise.  \newline
		Third, whether you place figures in the stand-alone or inline form, you should make the correct links and references. In general, you should provide a title that explains the figures, as well as a complete reference. A bad practice is to use abbreviations, which may slow down the reader.\newline
		Good articles need to ensure consistency between all figures and text. For example, all figures should be consistent in style, size, font, etc. In practice, the same drawing software can be used to meet the previous requirements. \newline
		Almost all the figures contain labels. Good labels need to be the same font, consistent with each other and using the same notation. In addition, good labels need to help the reader, so the labels need to be close to the figures and let the reader put more focus on the figures. \newline
		Be careful with plots and charts. 3D pie charts may cause distortion, so try to avoid using them. Moreover, the irritating background should be avoided. Consider using a table instead of a pie chart. Also, avoid using colors arbitrarily, which will distract the reader. \newline
		Finally, when authors design figures, they need to work hard to guide the reader's attention and avoid distractions. Good typography is something you do not notice. In addition, the strong contrast in the figures will distract the readers, which is something need to avoid. In terms of guiding the reader's attention, it is best to use colors instead of lines.
		
		\begin{itemize}
			\item \textbf{Comment by Yu Cang} \newline
				Clearly, Ning Kang has noticed all the key points expressed by the author. He has listed all the aspects mentioned in the paper and explained them in detail. Both good and bad practices are highlighted and compared, which is convenient for readers to understand. Transitional words are properly used, making the summary reads quite smoothly. No spelling mistakes were found and, in my opinion, it is an excellent summary.
				
			\item \textbf{Comment by Reze Rouhi Ardeshiri} \newline
				Ning had a good summarize for this section. He has played very well with words to describe the text, and this is one of the techniques in writing. I did not see grammatical errors in this text, except for one that in the second paragraph should use “first” instead of “firstly”. In a word, he has done a good job.
		
		\end{itemize}
	
	\subsection{Final Version}
		This section mainly talks about the general guidelines on graphics to be used in scientific writings. \newline
		In the first step, to write a paper with a lot of figures, one need to spend a lot of time on each page to have high-quality work and for the design of a figure, at least we need to consider time as much as a text or even more so.\newline
		Next, one should decide what kind of figure to put in the paper so that the reader reaches the goal of what we want that and the reader understands the issue. That is, the output of the figure you want to draw is clear to the reader.\newline
		At the same time, either placed in stand-alone form or in inlined form, figures should be linked and referenced properly. Usually, one should provide a detailed caption, which illustrates the key point of the figure, and a complete reference that faciliates readers a lot. It's better not to use abbreviations like "Fig.", which may slow readers down. \newline
		Then, consistency between graphics and text should be maintained through out the whole article. For example, it is suggest to use figures of same sacles, take same line width, keep fonts in figures as same as that in text, and choose color coding according to that used in text etc. In practice, instead of using different graphics software, the best way to keep consistency is to use a single graphics program. \newline
		Also, labels play an important role in plots. Text in labels should be of the same font as that in the main text, and fonts between labels should also be consistent with each other. Notions should not be different, for example, $\frac{1}{2}$ vs $0.5$, $e^{-i\pi}$ vs $-1$. In addition, labels should be both clear and well placed, which provide a straight forward explanation.\newline
		Most importantly, issues concerning plots and charts should be carefully handled. 3D pie charts may introduce distortions and should be avoided whenever possible. Instead, table can be used to explain in detail numbers and percentages. It is recommended to use colors that can group things or direct readers' focus. As background patterns may distract and is likely to blur important messages, they should not be used as well. \newline
		Finally, when authors design figures, they need to work hard to guide the reader's attention and avoid distractions. Good typography is something you do not notice. In addition, the strong contrast in the figures will distract the readers, which is something need to avoid. In terms of guiding the reader's attention, it is best to use colors instead of lines.
		
\section{\LaTeX}
	\begin{table}[H]
		\begin{center}
			\begin{tabular}{c |c |c}
				\hline
				Component & Transition & Probability ($\lambda$)\\
				\hline
				\multirow{5}{*}{MDC$_{-}$A} & Limited Bandwidth &0.2\\
				& Memory Leak & 0.3\\
				&Bandwidth Upgrade & 0.5 \\
				& MDC$_{-}$A Reboot &0.4\\
				&MDC Exception & Fixed Rate = 0.8 \\
				\hline
				\multirow{5}{*}{MDC$_{-}$B}& Communication Delay & 0.6\\
				& No WIFI & 0.3\\
				&Protocol Adjustment & 0.2 \\
				& Router Reboot & 0.8  \\
				&MDC Exception & Fixed Rate = 0.8 \\
				\hline
				\multirow{5}{*}{MDC$_{-}$C} & Retrieval Exception &0.3\\
				& System No Response & 0.4 \\
				&Omit Retrieval & 0.8  \\
				& MDC$_{-}$C Reboot & 0.9\\
				&MDC Exception & Fixed Rate = 0.8 \\
				\hline
				\multirow{5}{*}{MSC} & Overload & 0.1\\
				& System Crashed & 0.2 \\
				& Load Adjustment& 0.7 \\
				& MSC  Reboot & 0.9\\
				&MSC Exception& Fixed Rate = 0.9 \\
				\hline
			\end{tabular}
		\end{center}
		\caption{ Transition and probability of MCAS}
	\end{table}
	
	\begin{table}
		\begin{center}
			\begin{tabular}{ c c}
				\hline
				Trapdoor Size Before Compression (bit) & 65536 \\
				Trapdoor Size After Compression (bit) &3260.32 \\
				Compression Rate (\%) & 4.97 \\
				\hline
			\end{tabular}
		\end{center}
		\caption{Compression Rate}
	\end{table}
	
	\begin{table}
		\begin{center}
			\begin{tabular}{c |c |c c |c c}
				\hline\multirow{3}{*}{} & \multirow{2}{*}{Native} & \multicolumn{2}{c}{SLA-aware} & \multicolumn{2}{c}{Proportional-share} \\
				& & \multicolumn{2}{c}{Scheduling} & \multicolumn{2}{c}{Scheduling}\\
				\cline{2-6}
				& FPS & FPS & Overhead(\%) & FPS &Overhead(\%)\\
				\hline
				GT1 & 43.023 & 42.044 & 2.28 & 42.221 & 1.86 \\
				GT2 & 48.686 & 45.996 & 5.53 & 48.284 & 0.83\\
				HDRT1 & 59.062 & 57.923 & 1.93 & 57.700 & 2.31 \\
				HDRT2 & 65.808 & 62.854 & 4.90 & 65.984 & -0.27 \\
				\hline
			\end{tabular}
		\end{center}
		\caption{Macrobenchmark Results.}
	\end{table}
\end{document}
