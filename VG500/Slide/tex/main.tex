\input{preamble.tex}
\title{Numerical Simulation of Turbulent Combustion}
\subtitle{Using the Turbulent Flamelet Model}
\author{Yu Cang}

\institute{\color{white}
    yu.cang@sjtu.edu.cn\\
} %
\date{\footnotesize\color{mainthemecolour} CNF, Atlanta 2019. }




\begin{document}

\maketitle

\section{Introduction}
	\subsection{part1.1}
		\begin{xframe}{Backgound}
			Turbulent combustion is encountered in most practical combustion system such as rocket, ICE, and aircraft engien.
			\begin{figure}
				\begin{minipage}{3.5cm}
					\centering
					\includegraphics[height=3.5cm, width=3cm]{../pic/rocket.jpg}
				\end{minipage}%
				\begin{minipage}{7.5cm}
					\centering
					\includegraphics[height=3.5cm, width=7cm]{../pic/engien.jpg}
				\end{minipage}%
			\end{figure}
			Meaningful to practical systems:
			\begin{enumerate}[(a)]
				\item 
				Improve efficiency, meet demanding standards\cite{RN1}.
				\item
				Reduce pollution, environment friendly.
			\end{enumerate}
		\end{xframe}
	\subsection{part1.2}
		\begin{xframe}{Research aspects}
			Why is numerical simulation adopted?
			\begin{enumerate}[(a)]
				\item 
					Analytical techniques are difficult to handle.
				\item
					Experiments are too expensive to be widely used.
			\end{enumerate}
			What are the key points in numerical practice?
			\begin{itemize}
				\item Interaction between turbulence and combustion.
				\item Proper flame model.
			\end{itemize}
		\end{xframe}
	\subsection{part1.3}
		\begin{xframe}{Numerical simulation of flowfield}
			Based on CFD, there're 3 possible approaches:
			\begin{enumerate}[(a)]
				\item{Direct Numerical Simulation(DNS)}\newline
					--Precise, but costly(Tremendous memory and CPU). %\XSolid
				\item Large Eddy Simulation(LES)\newline
					--Compromise between accuracy and computational cost. %\Checkmark
				\item Reynolds Averaged Navier-Stokes(RANS)\newline
					--Inaccurate for combustion phenomenon. %\XSolid
			\end{enumerate}
			\includegraphics[width=11cm, height=3cm]{../pic/comparision.png}
		\end{xframe}
	\subsection{part1.4}
		\begin{xframe}{Numerical simulation of flame}
			Although based on LES, traditional flame models are inadequate:
			\begin{itemize}
				\item
					Distribution of flame properties are mannually assumed.
				\item
					Parameter tuning may be unphysical.
			\end{itemize}
			These drawbacks are overcome by the turbulent flamelet model to be introduced:
			\begin{itemize}
				\item
					Designed especially for LES, with tfewer approximation.
				\item
					Relations are provided through scaling law, which is based on DNS database\cite{RN2}.
			\end{itemize}		
		\end{xframe}
\section{Turbulent Flamelet Model}
	\subsection{part2.1}
		\begin{xframe}{Theory}
			Original G.E. in the context of LES:
			\begin{equation}
				\frac{\partial \bar{\rho}\tilde{Z}}{\partial t} + \nabla \cdot (\bar{\rho} \tilde{Z}\tilde{\vec{u}}) = \nabla \cdot [\bar{\rho}(D+D_T)\nabla\tilde{Z}]
			\end{equation}
			\begin{equation}
				\frac{\partial \bar{\rho}\tilde{Y_i}}{\partial t} + \nabla \cdot (\bar{\rho} \tilde{Y_i}\tilde{\vec{u}}) = \nabla \cdot [\bar{\rho}(D+D_T)\nabla\tilde{Y_i}] + \overline{\omega_i}
			\end{equation}
			After coordinate transformation$(x_1,x_2,x_3,t)\rightarrow(Z,Z_2,Z_3,\tau)$:
			\begin{equation}
				\begin{split}
					\bar{\rho}\frac{\partial \tilde{Y_i}}{\partial \tau} + \bar{\rho} \Big(\tilde{\vec{u}} \cdot \nabla_\perp \tilde{Y_i} + \frac{\partial \tilde{Y_i}}{\partial Z_2}\frac{\partial Z_2}{\partial t} + \frac{\partial \tilde{Y_i}}{\partial Z_3}\frac{\partial Z_3}{\partial t}\Big) = \frac{\bar{\rho}\chi}{2Le_T}\frac{\partial^2 \tilde{Y_i}}{\partial^2 \tilde{Z}}\\ + \frac{\partial \tilde{Y_i}}{\partial \tilde{Z}}\nabla\cdot\Bigg[\bar{\rho}(\mathcal{D}_{T,i}-\mathcal{D}_T)\vec{n}\cdot\frac{\partial \tilde{Z}}{\partial \vec{n}}\Bigg] + \nabla \cdot (\bar{\rho}\mathcal{D}_{T,i}\nabla_\perp\tilde{Y_i}) + \overline{\omega_i}
				\end{split}	
			\end{equation}
		\end{xframe}
	\subsection{part2.2}
		\begin{xframe}{Laminar Flamelet assumption}
			Locally, the characteristic timescale of chemical reaction is much smaller that that of flow$(t_c \ll t_f)$.\newline
			Thus, local flame structure can be described by the difffusion flame under counterflow configuration.
			\includegraphics[width=10cm, height=3cm]{../pic/flamelet.png}
			
			Each micro flamelet can be described by $Z$ and $\chi$:
			\begin{itemize}
				\item $Z$ describes chemical reaction.
				\item $\chi$ indicates turbulence stretching effect.
			\end{itemize}
			Thus, a database can be pre-computed for later looking-up. 
		\end{xframe}
	\subsection{part2.3}
		\begin{xframe}{Turbulent Flamelet}
			\begin{multicols}{2}
				\includegraphics[width=5.5cm, height=7cm]{../pic/counterflow.png}	
				
				Unlike the laminar flamelet introduced above,\\
				G.E. of the counterflow flame is slightly modified by our turbulet flamelet model from
				\begin{equation}
					\rho \frac{D Y_i}{D t} = \mathcal{D}_i\frac{\partial^2 Y_i}{\partial^2 x} + \omega_i(T, \vec{Y})
				\end{equation}
				to
				\begin{equation}
					\bar{\rho} \frac{D \tilde{Y_i}}{D t} = \mathcal{D}_i\frac{\partial^2 \tilde{Y_i}}{\partial^2 x} + \tilde{\omega_i}(\tilde{T}, \tilde{\vec{Y}})
				\end{equation}
				The two equations share similar form, but have totally different meanings.
			\end{multicols}
		\end{xframe}
	\subsection{part2.4}
		\begin{xframe}{Solution procedure}
			Based on the filtered turbulent flamelet database generated in the way descirbed above, the full solution procedure that incorporates a CFD solver can be described as follows:
			\centering
			\includegraphics[width=9cm, height=6.5cm]{../pic/solver.png}
		\end{xframe}
\section{Numerical Results}
	\subsection{part3.1}
		\begin{xframe}{Comparsion of ``S'' curve}
			The $T_{max}$ plot:
			\begin{itemize}
				\item
					One of the most convincing testing cases.
				\item
					Difference and transition position are clearly revealed\cite{RN11}.
			\end{itemize}			
			\includegraphics[width=11cm, height=6cm]{../pic/Tmax.jpg}	
		\end{xframe}
	\subsection{part3.2}
		\begin{xframe}{Standard case}
			Comparsion between experimental data, which is widely used as benchmark\cite{RN14}.
			\begin{figure}
				\begin{minipage}{3.5cm}
					\centering
					\includegraphics[height=3cm, width=5cm]{../pic/slice.png}
				\end{minipage}%
				\begin{minipage}{7.5cm}
					\centering
					\includegraphics[height=5.5cm, width=7cm]{../pic/line.png}
				\end{minipage}%
			\end{figure}
		\end{xframe}

	\begin{xframe}{Reference}
		\printbibliography
	\end{xframe}
\end{document}
