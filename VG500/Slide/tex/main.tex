\input{preamble.tex}
\input{pdfinfo.tex}
\title{Numerical Simulation of Turbulent Combustion}
\subtitle{Using the Turbulent Flamelet Model}
\author{Yu Cang}

\institute{\color{white}
    yu.cang@sjtu.edu.cn\\
} %
\date{\footnotesize\color{mainthemecolour} CNF, Atlanta 2019. }



\begin{document}

\maketitle


\section{Themes}

\subsection{Background images}

\begin{xframe}{Themes for the Elegance}

    You need a \hl{style/images} directory
    with these files inside:

    \begin{itemize}
        \item style/background-section.pdf
        \item style/background-slide.pdf
        \item style/background-title.pdf
        \item style/logo.png
    \end{itemize}

\end{xframe}


{
\usebackgroundtemplate{\includegraphics[width=\paperwidth]{../pic/theme-1.png}}
\begin{frame}[plain]
    .
\end{frame}
}


{
\usebackgroundtemplate{\includegraphics[width=\paperwidth]{../pic/theme-2.png}}
\begin{frame}[plain]
    .
\end{frame}
}


{
\usebackgroundtemplate{\includegraphics[width=\paperwidth]{../pic/theme-3.png}}
\begin{frame}[plain]
    .
\end{frame}
}





\section{Examples}

\subsection{Showing code}

\begin{xframe}{Code snippets}

    Do you have some code to show on the slide?

    And the same frame should also contain text?

    \begin{cxxcodebox}
        class example {
            // \codedots shows grayed-out dots
            @ \codedots @
        };
    \end{cxxcodebox}

    You can use \verb|cxxcodebox| environment.
    It has \verb|cxx| in the name,
    but no syntax highlighting is performed.

    For short code snippets,
    it is better just to highlight the important parts.

\end{xframe}


\subsection{Code slides and escapes}

\begin{xframe}{Second slide}

    \begin{cxxcode}
        class example {
            // There are a few useful escapes here
            @ \codedots @ // \codedots shows grayed-out dots

            // Invalid parts can be marked with \hlErr
            @\hlErr{operator;}@

            // Good parts can be marked with \hlOk
            @\hlOk{operator() ()}@

            // Other highlighting commands can be seen in
            // the preamble.tex file
        };
    \end{cxxcode}

\end{xframe}

\end{document}
