\documentclass[paper=a4, fontsize=11pt]{scrartcl}

\usepackage{lipsum}
\usepackage{amsmath, amsfonts, amssymb}

\title{Reparaphrase of "CFD modelling of combustion"}
\author{Yu Cang \ \ 018370210001}
\date{2019-04-06}

\begin{document}

\maketitle

\section{Original paragraph}
	The properties of laminar flamelets, e.g. the temperature, density and
species mass fractions as a function of the mixture fraction $\xi$, are evaluated
once and for all outside the flow field calculation to yield a so-called laminar
flamelet library. A library of such relationships used in a calculation is
shown later in an example (see Figure 12.16). The library comprises a set of
relationships $\phi(\xi)$ between scalar flow properties $\phi$ and the mixture fraction
$\xi$. Their form is similar to Figure 12.2, but turbulence causes the flames to
be stretched, which alters the details of the relationships $\phi(\xi)$. To accommodate
the effect of flame stretching due to the flow field in an actual turbulent
flame, it is common practice to incorporate as a parameter either the strain
rate or the scalar dissipation rate, both of which are flow properties, into
the laminar flamelet library calculations.

\section{Re-paraphrase}
	Temperature, density and species mass fractions are properties of laminar flamelets. They are related to the mixture fraction $\xi$, and these relations are called a laminar flamelet library. Such a library is needed to calculate the outside flow, and examples will be shown later (see Figure 12.16). The library is a collection of relations describing how the scalar flow properties $\phi$ changes when the mixtures fraction $\xi$ changes. They are similar to that in Figure 12.2, expect the stretching efffect caused by turbulence alters the details of the relationships $\phi(\xi)$. Usually, an extra parameter, either the strain rate or the scalar dissipation rate, is added into the laminar flamelet library to account for this stretching effect, as both of these two are flow properties.
	
\end{document}
