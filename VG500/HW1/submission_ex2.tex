% ****** Start of file apssamp.tex ******
%
%   This file is part of the APS files in the REVTeX 4.1 distribution.
%   Version 4.1r of REVTeX, August 2010
%
%   Copyright (c) 2009, 2010 The American Physical Society.
%
%   See the REVTeX 4 README file for restrictions and more information.
%
% TeX'ing this file requires that you have AMS-LaTeX 2.0 installed
% as well as the rest of the prerequisites for REVTeX 4.1
%
% See the REVTeX 4 README file
% It also requires running BibTeX. The commands are as follows:
%
%  1)  latex apssamp.tex
%  2)  bibtex apssamp
%  3)  latex apssamp.tex
%  4)  latex apssamp.tex
%
\documentclass[%
 reprint,
%superscriptaddress,
%groupedaddress,
%unsortedaddress,
%runinaddress,
%frontmatterverbose, 
%preprint,
%showpacs,preprintnumbers,
%nofootinbib,
%nobibnotes,
%bibnotes,
 amsmath,amssymb,
 aps,
%pra,
%prb,
%rmp,
%prstab,
%prstper,
%floatfix,
]{revtex4-1}

\usepackage{graphicx}% Include figure files
\usepackage{dcolumn}% Align table columns on decimal point
\usepackage{bm}% bold math
%\usepackage{hyperref}% add hypertext capabilities
%\usepackage[mathlines]{lineno}% Enable numbering of text and display math
%\linenumbers\relax % Commence numbering lines

%\usepackage[showframe,%Uncomment any one of the following lines to test 
%%scale=0.7, marginratio={1:1, 2:3}, ignoreall,% default settings
%%text={7in,10in},centering,
%%margin=1.5in,
%%total={6.5in,8.75in}, top=1.2in, left=0.9in, includefoot,
%%height=10in,a5paper,hmargin={3cm,0.8in},
%]{geometry}

\begin{document}

%\preprint{APS/123-QED}

\title{Summary of 2 paper related to turbulent combustion}% Force line breaks with \\
\thanks{VG500 course from JI SJTU}%

\author{Yu Cang}
\affiliation{Shanghai Jiaotong University}
\date{\today}

\begin{abstract}
	Summary of 2 papers published in PRL related to turbulent combustion are discussed.
\end{abstract}

\maketitle

%\tableofcontents

\section{Summary of "Spontaneous Transition of Turbulent Flames to Detonations in Unconfined Media"}

Deflagration-to-detonation transition (DDT) can occur in a wide range of environments, ranging from industrial systems to astrophysical thermonuclear supernovae explosions. \\
This paper talks mainly about the nature of DDT in confined systems with walls, internal obstacles, or pre-existing shocks, although it remains unclear if DDT can occur in unconfined media or not.\\
Direct numerical simulations (DNS) was used so that high enough turbulent intensities can be achived. Numerical results show that unconfined, subsonic, premixed, turbulent flames are inherently unstable to DDT. The associated mechanism, based on the nonsteady evolution of flames faster than the Chapman-Jouguet deflagrations, is qualitatively different from the traditionally suggested spontaneous reaction-wave model. Critical turbulent flame speeds, predicted by this mechanism for the onset of DDT, are in agreement with DNS results.

\section{Summary of "Facilitated Ignition in Turbulence through Differential Diffusion"}
In this paper, authors indicate that ignition in turbulence is much more likely to be done, compared with the case in quiescence. This is quite out of intuition and the reason behind is explained. \\
Experiments were carried out showing that it is thermal diffusivity that play an important role in spreading inside the mixture rather than the mass diffusivity. \\
In such cases, turbulence breaks the otherwise single spherical flame of positive curvature, and hence positive aerodynamics stretch, into a multitude of wrinkled flamelets subjected to either positive or negative stretch, such that the intensified burning of the latter constitutes local sources to facilitate ignition.

\end{document}
%
% ****** End of file apssamp.tex ******
