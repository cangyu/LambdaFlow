\documentclass[paper=a4, fontsize=11pt]{scrartcl} % A4 paper and 11pt font size

\usepackage[T1]{fontenc} % Use 8-bit encoding that has 256 glyphs
\usepackage{fourier} % Use the Adobe Utopia font for the document - comment this line to return to the LaTeX default
\usepackage[english]{babel} % English language/hyphenation
\usepackage{amsmath,amsfonts,amsthm,amssymb} % Math packages

\usepackage{algorithm, algorithmic}
\renewcommand{\algorithmicrequire}{\textbf{Input:}} %Use Input in the format of Algorithm  
\renewcommand{\algorithmicensure}{\textbf{Output:}} %UseOutput in the format of Algorithm  

\usepackage{graphicx}
\usepackage{blindtext}
\usepackage{enumerate}

\usepackage{listings}
\lstset{language=Matlab}

\usepackage{lipsum} % Used for inserting dummy 'Lorem ipsum' text into the template

\usepackage{sectsty} % Allows customizing section commands
\allsectionsfont{\centering \normalfont\scshape} % Make all sections centered, the default font and small caps

\usepackage{fancyhdr} % Custom headers and footers
\pagestyle{fancyplain} % Makes all pages in the document conform to the custom headers and footers
\fancyhead{} % No page header - if you want one, create it in the same way as the footers below
\fancyfoot[L]{} % Empty left footer
\fancyfoot[C]{} % Empty center footer
\fancyfoot[R]{\thepage} % Page numbering for right footer
\renewcommand{\headrulewidth}{0pt} % Remove header underlines
\renewcommand{\footrulewidth}{0pt} % Remove footer underlines
\setlength{\headheight}{13.6pt} % Customize the height of the header

\numberwithin{equation}{section} % Number equations within sections (i.e. 1.1, 1.2, 2.1, 2.2 instead of 1, 2, 3, 4)
\numberwithin{figure}{section} % Number figures within sections (i.e. 1.1, 1.2, 2.1, 2.2 instead of 1, 2, 3, 4)
\numberwithin{table}{section} % Number tables within sections (i.e. 1.1, 1.2, 2.1, 2.2 instead of 1, 2, 3, 4)

\setlength\parindent{0pt} % Removes all indentation from paragraphs - comment this line for an assignment with lots of text

\newcommand{\horrule}[1]{\rule{\linewidth}{#1}} % Create horizontal rule command with 1 argument of height
\newcommand*{\dif}{\mathop{}\!\mathrm{d}}

\title{	
\normalfont \normalsize 
\textsc{Shanghai Jiao Tong University, UM-SJTU JOINT INSTITUTE} \\ [25pt] % Your university, school and/or department name(s)
\horrule{0.5pt} \\[0.4cm] % Thin top horizontal rule
\huge Technical Communication\\ HW2 \\ % The assignment title
\horrule{2pt} \\[0.5cm] % Thick bottom horizontal rule
}

\author{Yu Cang \quad 018370210001} % Your name

\date{\normalsize \today} % Today's date or a custom date

\begin{document}

\maketitle % Print the title

\section{Writing Mathematics}
	As a final example, consider the triple compositions
	\begin{eqnarray}
		f(x)=\int_a^{\big(\int_a^{x^3} {\frac{1}{1+\sin^2{t}}} dt \big)} \frac{1}{1+\sin^2{t}} dt,
		\quad
		 g(x)=\int_a^{\left[\begin{matrix}(\int_a^x \frac{1}{1+\sin^2{t}} dt)\\\int_a \frac{1}{1+\sin^2{t}} dt\end{matrix}\right]} \frac{1}{1+\sin^2{t}} dt,\notag
	\end{eqnarray}
	which can be written
	\[
		f=F\circ F\circ C \quad \mbox{and}\quad g=F\circ F\circ F.
	\]
	Suppose $F(x)=\int_a^x \dfrac{1}{1+\sin^2{t}} dt$ and $C(x)=x^3$, we have
	\begin{align}
		f'(x) & = F'(F(C(x)))\cdot F'(C(x))\cdot C'(x) \notag\\
		       & = \dfrac{1}{1+\sin^2{\left( \int_a^{x^3} \dfrac{1}{1+\sin^2{t}} dt \right)}} \cdot \dfrac{1}{1+\sin^2{x^3}} \cdot 3x^2.\notag
	\end{align}
	Likewise, we have
	\begin{align}
		g'(x) & = F'(F(F(x)))\cdot F'(F(x))\cdot F'(x) \notag\\
			   & = \dfrac{1}{1+\sin^2{\left[ \int_a^{ \left( \int_a^x \dfrac{1}{1+\sin^2{t}} dt \right) } \dfrac{1}{1+\sin^2{t}} dt \right]}} \cdot \dfrac{1}{1+\sin^2{\left( \int_a^{x} \dfrac{1}{1+\sin^2{t}} dt \right)}} \cdot \dfrac{1}{1+\sin^2{x}}.\notag
	\end{align}

\section{\LaTeX}
	\subsection{Exercise on Slide 242}
		\begin{eqnarray}\notag
			x^{y+z}_{4t},\ \frac{y+\frac{3z}{2}}{b},\ \sqrt[n]{\Omega},\ \sum\limits_{n=0}^{\infty}n,\  \int_{0}^{1}\frac{1}{x} dx,\ \forall n \in \mathbb{N} \exists m\ \mbox{such that}\ n-m=0
		\end{eqnarray}
	\subsection{Exercise on Slide 246}
		\begin{enumerate}
			\item[1] \textbf{Using package blindtext} \newline
				\blindmathpaper
			
			\item[2] \textbf{Reproduce equations} \newline
				\begin{equation}\notag
					\bar{x}=\dfrac{1}{n}\sum_{i=1}^{i=n} x_i=\dfrac{x_1+x_2+\ldots+x_n}{n} 
				\end{equation}
				
				\begin{equation}\notag
					\int_0^\infty e^{-\alpha x^2} \dif{x}=\dfrac{1}{2} \sqrt{\int_{-\infty}^\infty e^{-\alpha x^2}}\dif{x} \int_{-\infty}^\infty e^{-\alpha y^2}\dif{y} =\dfrac{1}{2} \sqrt{\dfrac{\pi}{\alpha}}
				\end{equation}
				
				\begin{equation}\notag
					\sum_{k=0}^\infty a_0q^k=\lim_{n\rightarrow \infty}\sum_{k=0}^n a_0q^k=\lim_{n\rightarrow \infty}a_0 \dfrac{1-q^{n+1}}{1-q}=\dfrac{a_0}{1-q}
				\end{equation}
				
				\begin{equation}\notag
					x_{1,2}=\dfrac{-b\pm \sqrt{b^2-4ac}}{2a}=\dfrac{-p\pm \sqrt{p^2-4q}}{2}
				\end{equation}
				
				\begin{equation}\notag
					\dfrac{\partial^2 \Phi}{\partial x^2}+\dfrac{\partial^2 \Phi}{\partial y^2}+\dfrac{\partial^2 \Phi}{\partial z^2}=\dfrac{1}{c^2}\dfrac{\partial^2 \Phi}{\partial t^2}
				\end{equation}
	
			\item[3] \textbf{Vandermonde matrix}
				\begin{equation}\notag
					\left[\begin{matrix}
						1 & \alpha_1 & \alpha^2_1 &\ldots &\alpha^{n-1}_1 \\
						1 & \alpha_2 & \alpha^2_2 &\ldots &\alpha^{n-1}_2 \\
						1 & \alpha_3 & \alpha^2_3 &\ldots &\alpha^{n-1}_3 \\
						\vdots & \vdots & \vdots & \ddots & \vdots \\
						1 & \alpha_m & \alpha^2_m &\ldots &\alpha^{n-1}_m
					\end{matrix}\right]
				\end{equation}
		\end{enumerate}
	\subsection{Exercise on Slide 250}
		\begin{equation}\notag
			\begin{aligned}
				\tilde{E}(\omega)=& \int_{-\infty}^{\infty}E_0\sin(\omega_0t)e^{-2i\pi\omega t}dt \\
				\sin(x)= & \frac{e^{ix}-e^{-ix}}{2i} \\
				e^ae^b= & e^{a+b} \\
				\int_{-\infty}^{\infty}e^{it(x-x')}dt=&2\pi \delta(x-x')\\
				\tilde{E}(\omega)=& \frac{E_0}{2i}\int_{-\infty}^{\infty}e^{-2i\pi t(\omega-\frac{\omega_0}{2\pi})}-e^{-2i\pi t(\omega+\frac{\omega_0}{2\pi})}dt \\
				\tilde{E}(\omega)=&\frac{-2\pi^2}{i}E_0[\delta(\omega-\frac{\omega_0}{2\pi})-\delta(\omega+\frac{\omega_0}{2\pi})]
			\end{aligned}
		\end{equation}
		
\section{Group Exercise}
	I'm in Group 2, responsible for paragraph 4, 14 and 19.
	\subsection{Summary}
		In paragraph 4, the author mainly talks about the necessity, the advantange and the way towards a well-organized structure. Organization is the core competence of a book. A clear structure helps readers follow the logic and understand easily. Writters need to prepare an outline at first, which may take a long period, and then decide what are around the kernel and put them in.\newline
		In paragraph 14, the author indicated the importance of using technical terms correctly and gave 3 examples explaining it. It's better not to use newly created words, and use old ones instead.\newline
		In paragraph 19, the author suggest writters to stop, without hesitation, after writting, not to worry about minor modifications. Some tips were provided for checking the manuscripts quickly.  
		
	\subsection{Remarks on peer's work}
		\begin{enumerate}[(a)]
			\item \textbf{Remarks on Shuixin Xiao's summary}\newline
				All the key points in the article are listed, especially for section 4, of which the main ideas from the author are clearly summarized. It would be better if some conjunction words are used to connect points. This is a good summary without doubt. 
			\item \textbf{Remarks on Yijie Wang's summary}\newline
				In this summary, the expression is clear and coherent. The core parts of the sections are extracted. It will be much better if the statement can be more concise.
			\item \textbf{Remarks on Yaoxia Shao's summary}\newline
				This groupmate makes a really good summary, of which the perspective is quite different and profound. Obviously, he has clearly understood the author’s ideas, and paraphrase them in his own words. 
		\end{enumerate}
	
	\subsection{Discussion}
		This part is submitted by Xiao Shuixin, whose student ID is 018370910023.

\section{Grammar}
	\subsection{Plurals of given words}
		\begin{enumerate}
			\item[(a)] means: means
			\item[(b)] paralysis: paralyses
			\item[(c)] curriculum: curricula/curriculums
			\item[(d)] oasis: oases
			\item[(e)] offspring: offspring/offsprings
			\item[(f)] criterion: criteria/criterions
			\item[(g)] Chinese: Chinese
			\item[(h)] antenna: antennae/antennas
			\item[(i)] stimulus: stimuluses/stimuli
			\item[(j)] fungus: fung/funguses
			\item[(k)] alumnus: alumni
			\item[(l)] series: series
			\item[(m)] diagnosis: diagnoses
			\item[(n)] vita: vitae
			\item[(o)] American: American
			\item[(p)] synopsis: synopses
		\end{enumerate}
	
	\subsection{Filling the blanks}
		(a) \underline{\makebox[1cm]{The}} Decline and Fall of \underline{\makebox[1cm]{the}} Roman Empire\\
		(b) \underline{\makebox[1cm]{The}} complexity of \underline{\makebox[1cm]{the}} problem of \underline{\makebox[1cm]{the}} decline and fall of the Roman Empire is made evident by \underline{\makebox[1cm]{a}} wide variety of causes that are emphasized in varying degrees by \makebox[1cm]{\hrulefill}different authors.\\
		(c) Fortunately, \underline{\makebox[1cm]{a}} concise formulation of Edward Gibbon serves as \underline{\makebox[1cm]{a}} widely accepted basis for \makebox[1cm]{\hrulefill} modern discussion of \underline{\makebox[1cm]{the}} problem.\\
		(d) According to Gibbon,\underline{\makebox[1cm]{the}} empire reached its peak during \underline{\makebox[1cm]{the}} administration of \makebox[1cm]{\hrulefill} two Antonines.\\
		(e) After that, however, \underline{\makebox[1cm]{the}}extent of \makebox[1cm]{\hrulefill} Roman conquest became too great to be managed by \makebox[1cm]{\hrulefill} Roman government,\underline{\makebox[1cm]{the}} decline began.\\
		(f) \underline{\makebox[1cm]{The}} military government was weakened and finally dissolved as \makebox[1cm]{\hrulefill} barbarians were allowed to constitute \underline{\makebox[1cm]{an}} ever-growing percentage of \underline{\makebox[1cm]{the}} Roman legions.\\
		(g) \underline{\makebox[1cm]{The}} victorious legions began to dominate and corrupt \underline{\makebox[1cm]{the}} government, weakening it at \underline{\makebox[1cm]{the}} time when it most needed \underline{\makebox[1cm]{a}} strength to overcome \makebox[1cm]{\hrulefill} other problems.

\end{document}