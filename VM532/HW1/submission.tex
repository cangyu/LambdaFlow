\documentclass[paper=a4, fontsize=11pt]{scrartcl} % A4 paper and 11pt font size
\usepackage[T1]{fontenc} % Use 8-bit encoding that has 256 glyphs
\usepackage{fourier} % Use the Adobe Utopia font for the document - comment this line to return to the LaTeX default
\usepackage[english]{babel} % English language/hyphenation
\usepackage{amsmath,amsfonts,amsthm,amssymb} % Math packages
\usepackage{mathrsfs}
\usepackage{algorithm, algorithmic}
\renewcommand{\algorithmicrequire}{\textbf{Input:}} %Use Input in the format of Algorithm  
\renewcommand{\algorithmicensure}{\textbf{Output:}} %UseOutput in the format of Algorithm  
\usepackage{listings}
\lstset{language=Matlab}
\usepackage{lipsum} % Used for inserting dummy 'Lorem ipsum' text into the template
\usepackage{sectsty} % Allows customizing section commands
\allsectionsfont{\centering \normalfont\scshape} % Make all sections centered, the default font and small caps
\usepackage{fancyhdr} % Custom headers and footers
\pagestyle{fancyplain} % Makes all pages in the document conform to the custom headers and footers
\fancyhead{} % No page header - if you want one, create it in the same way as the footers below
\fancyfoot[L]{} % Empty left footer
\fancyfoot[C]{} % Empty center footer
\fancyfoot[R]{\thepage} % Page numbering for right footer
\renewcommand{\headrulewidth}{0pt} % Remove header underlines
\renewcommand{\footrulewidth}{0pt} % Remove footer underlines
\setlength{\headheight}{13.6pt} % Customize the height of the header
\numberwithin{equation}{section} % Number equations within sections (i.e. 1.1, 1.2, 2.1, 2.2 instead of 1, 2, 3, 4)
\numberwithin{figure}{section} % Number figures within sections (i.e. 1.1, 1.2, 2.1, 2.2 instead of 1, 2, 3, 4)
\numberwithin{table}{section} % Number tables within sections (i.e. 1.1, 1.2, 2.1, 2.2 instead of 1, 2, 3, 4)
\setlength\parindent{0pt} % Removes all indentation from paragraphs - comment this line for an assignment with lots of text
\newcommand{\horrule}[1]{\rule{\linewidth}{#1}} % Create horizontal rule command with 1 argument of height

\title{\normalfont \normalsize 
	\textsc{Shanghai Jiao Tong University, UM-SJTU JOINT INSTITUTE} \\ [25pt]
	\horrule{0.5pt} \\[0.4cm] 
	\huge Advanced Convection \\ HW1 \\
	\horrule{2pt} \\[0.5cm]}

\author{Yu Cang \\ 018370210001}
\date{\normalsize \today}

\begin{document}

\maketitle

\section{PROBLEM 1}
When $u_2= 0$ and the flow is entirely in the x-direction, the governing equations can be simplified as follows
\begin{equation}
	u_1 \frac{\partial u_1}{\partial x_1} + \frac{1}{\rho}\frac{\partial p}{\partial x_1} = 0
\end{equation}
\begin{equation}
	\frac{\partial p}{\partial x_2} = 0
\end{equation}
where the body-force are neglected.\\
It can be concluded that pressure doesn't change in y-direction when x is fixed.\\
Also, pressure gradient in x-direction is the driven force of the fluids.

\section{PROBLEM 2}
When it comes to constant property
\begin{equation}
	\frac{\partial}{\partial y}(k\frac{\partial T}{\partial y}) = k \frac{\partial T^2}{\partial^2 y}
\end{equation}
Since $i=C_p T$ when considering ideal gas
\begin{equation}
	\rho u \frac{\partial i}{\partial x} + \rho v \frac{\partial i}{\partial y} = \rho C_p (u\frac{\partial T}{\partial x} + v \frac{\partial T}{\partial y})
\end{equation}
As $(\frac{\partial u}{\partial y})^2$ is a small term compared to other terms, it can be seen as 0.\\
(Sorry, haven't figured out how to eliminate $\frac{\partial p}{\partial x}$...)

\section{PROBLEM 3}


\end{document}