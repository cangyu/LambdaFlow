\documentclass[paper=a4, fontsize=11pt]{scrartcl} % A4 paper and 11pt font size

\usepackage[T1]{fontenc} % Use 8-bit encoding that has 256 glyphs
\usepackage{fourier} % Use the Adobe Utopia font for the document - comment this line to return to the LaTeX default
\usepackage[english]{babel} % English language/hyphenation
\usepackage{amsmath,amsfonts,amsthm,amssymb} % Math packages

\usepackage{caption}
\usepackage{graphicx, subfig}

\usepackage{algorithm, algorithmic}
\renewcommand{\algorithmicrequire}{\textbf{Input:}} %Use Input in the format of Algorithm  
\renewcommand{\algorithmicensure}{\textbf{Output:}} %UseOutput in the format of Algorithm  

\usepackage{listings}
\lstset{language=Matlab}

\usepackage{lipsum} % Used for inserting dummy 'Lorem ipsum' text into the template

\usepackage{sectsty} % Allows customizing section commands
\allsectionsfont{\centering \normalfont\scshape} % Make all sections centered, the default font and small caps

\usepackage{fancyhdr} % Custom headers and footers
\pagestyle{fancyplain} % Makes all pages in the document conform to the custom headers and footers
\fancyhead{} % No page header - if you want one, create it in the same way as the footers below
\fancyfoot[L]{} % Empty left footer
\fancyfoot[C]{} % Empty center footer
\fancyfoot[R]{\thepage} % Page numbering for right footer
\renewcommand{\headrulewidth}{0pt} % Remove header underlines
\renewcommand{\footrulewidth}{0pt} % Remove footer underlines
\setlength{\headheight}{13.6pt} % Customize the height of the header

\numberwithin{equation}{section} % Number equations within sections (i.e. 1.1, 1.2, 2.1, 2.2 instead of 1, 2, 3, 4)
\numberwithin{figure}{section} % Number figures within sections (i.e. 1.1, 1.2, 2.1, 2.2 instead of 1, 2, 3, 4)
\numberwithin{table}{section} % Number tables within sections (i.e. 1.1, 1.2, 2.1, 2.2 instead of 1, 2, 3, 4)

\setlength\parindent{0pt} % Removes all indentation from paragraphs - comment this line for an assignment with lots of text

\newcommand{\horrule}[1]{\rule{\linewidth}{#1}} % Create horizontal rule command with 1 argument of height

\title{	
\normalfont \normalsize 
\textsc{Shanghai Jiao Tong University, UM-SJTU JOINT INSTITUTE} \\ [25pt] % Your university, school and/or department name(s)
\horrule{0.5pt} \\[0.4cm] % Thin top horizontal rule
\huge Turbulence \\ HW2 \\ % The assignment title
\horrule{2pt} \\[0.5cm] % Thick bottom horizontal rule
}

\author{Yu Cang \\ 018370210001}

\date{\normalsize \today}

\begin{document}

\maketitle

\section{Exercise1}
	Consider the N-S equation
	\begin{equation}
		\frac{\partial u_j}{\partial t} + u_k \frac{\partial u_j}{\partial x_k} = -\frac{1}{\rho}\frac{\partial p}{\partial x_j} + \nu \frac{\partial^2 u_j}{\partial x_k \partial x_k} + f_j
	\end{equation}
	
	With the ensemble average, each quantity can be decomposed into mean and fluctuation. Namely
	\begin{equation}
		\begin{aligned}
			u_j & = U_j + u_j'\\
			p & = P + p'\\
			f_j &= F_j + f_j'
		\end{aligned}
	\end{equation}
	Thus, the N-S equation can be expanded as
	\begin{equation}
		\Bigg(\frac{\partial U_j}{\partial t} + \frac{\partial u_j'}{\partial t}\Bigg) + \Bigg(U_k\frac{\partial U_j}{\partial x_k} + u_k'\frac{\partial U_j}{\partial x_k} + U_k \frac{\partial u_j'}{\partial x_k} + u_k'\frac{\partial u_j'}{\partial x_k}\Bigg) = -\frac{1}{\rho}\Bigg(\frac{\partial P}{\partial x_j} + \frac{\partial p'}{\partial x_j}\Bigg) + \nu \Bigg(\frac{\partial^2 U_j}{\partial x_k \partial x_k} + \frac{\partial^2 u_j'}{\partial x_k \partial x_k}\Bigg) + (F_j + f_j')
	\end{equation}
	
	Taking ensemble average on the equation above yiels the so called Reynolds equation
	\begin{equation}
		\frac{\partial U_j}{\partial t} + U_k \frac{\partial U_j}{\partial x_k} + \overline{u_k'\frac{\partial u_j'}{\partial x_k}} = -\frac{1}{\rho}\frac{\partial P}{\partial x_j} + \nu \frac{\partial^2 U_j}{\partial x_k \partial x_k} + F_j
	\end{equation}
	
	Substract the Reynolds equation by N-S equation that has been expaneded yields
	\begin{equation}
		\frac{\partial u_j'}{\partial t} + u_k'\frac{\partial U_j}{\partial x_k} + U_k \frac{\partial u_j'}{\partial x_k} + u_k'\frac{\partial u_j'}{\partial x_k} - \overline{u_k'\frac{\partial u_i'}{\partial x_k}} = -\frac{1}{\rho}\frac{\partial p'}{\partial x_j} + \nu \frac{\partial^2 u_j'}{\partial x_k \partial x_k} + f_j'
	\end{equation}
	
	With the continuity equation for incompressible flow
	\begin{equation}
		\frac{\partial u_k'}{\partial x_k} = 0 
	\end{equation}
	the substracted equation can be re-written as
	\begin{equation}
		N_j\{\vec{x}, t\} \triangleq \frac{\partial u_j'}{\partial t} + u_k' \frac{\partial \overline{u_j}}{x_k} + \overline{u_k}\frac{\partial u_j'}{\partial x_k} + \frac{\partial [u_j' u_k']}{\partial x_k} - \frac{\partial \overline{u_j' u_k'}}{\partial x_k} + \frac{1}{\rho} \frac{\partial p'}{\partial x_j}-\nu \frac{\partial^2 u_j'}{\partial x_k \partial x_k} - f_j' = 0
		\label{eq:sns}
	\end{equation}

	Also, an identity is frequently used within the exercise
	\begin{equation}
		\frac{d u(t+s)}{dt} = \frac{d u(t+s)}{d(t+s)} = \frac{d u(t+s)}{ds} 
	\end{equation}
	
	Since the two-point correlation function is defined as
	\begin{equation}
		R_{ij} = \overline{u_i'(\vec{x}, t) u_j'(\vec{x}+\vec{r}, t + \tau)}
	\end{equation}

	Thus, for $D^{(i)}\{R_{ij}\} = \overline{u_j'(\vec{x} + \vec{r}, t+\tau) N_i\{\vec{x}, t\}}$, components in the expansion are calculated as
	\begin{equation}
		\begin{aligned}
			\overline{u_j'(\vec{x} + \vec{r}, t+\tau) \frac{\partial u_i'(\vec{x}, t)}{\partial t}}  
			& = \overline{\frac{\partial (u_i'(\vec{x}, t)u_j'(\vec{x} + \vec{r}, t+\tau))}{\partial t} - u_i'(\vec{x}, t) \frac{\partial u_j'(\vec{x} + \vec{r}, t+\tau)}{\partial t}}\\
			& = \frac{\partial \overline{u_i'(\vec{x}, t)u_j'(\vec{x} + \vec{r}, t+\tau)}}{\partial t} - \overline{u_i'(\vec{x}, t) \frac{\partial u_j'(\vec{x} + \vec{r}, t+\tau)}{\partial \tau}}\\
			& = \frac{\partial R_{ij}}{\partial t} - \frac{\partial \overline{u_i'(\vec{x}, t) u_j'(\vec{x} + \vec{r}, t+\tau)}}{\partial \tau}\\
			& = \frac{\partial R_{ij}}{\partial t} - \frac{\partial R_{ij}}{\partial \tau}
		\end{aligned}
	\end{equation} 
	
	\begin{equation}
		\begin{aligned}
			\overline{u_j'(\vec{x} + \vec{r}, t+\tau) \overline{u_k}\frac{\partial u_i'}{\partial x_k}} 
			& = \overline{u_k(\vec{x}, t)} \overline{u_j'(\vec{x} + \vec{r}, t+\tau) \frac{\partial u_i'(\vec{x}, t)}{\partial x_k}}\\
			& = \overline{u_k(\vec{x}, t)} \Bigg[\frac{\partial \overline{u_i'(\vec{x}, t) u_j'(\vec{x} + \vec{r}, t+\tau)}}{\partial x_k}- \overline{u_i'(\vec{x}, t) \frac{\partial u_j'(\vec{x} + \vec{r}, t+\tau)}{\partial (x_k + r_k)}}\Bigg]\\
			& = \overline{u_k(\vec{x}, t)} \Bigg[\frac{\partial R_{ij}}{\partial x_k}- \overline{u_i'(\vec{x}, t) \frac{\partial u_j'(\vec{x} + \vec{r}, t+\tau)}{\partial r_k}}\Bigg]\\
			& = \overline{u_k(\vec{x}, t)} \Bigg[\frac{\partial R_{ij}}{\partial x_k}- \frac{\partial \overline{u_i'(\vec{x}, t) u_j'(\vec{x} + \vec{r}, t+\tau)}}{\partial r_k}\Bigg]\\
			& = \overline{u_k(\vec{x}, t)} \Bigg[\frac{\partial R_{ij}}{\partial x_k}- \frac{\partial R_{ij}}{\partial r_k}\Bigg]
		\end{aligned}
	\end{equation}
	
	\begin{equation}
			\overline{u_j'(\vec{x} + \vec{r}, t+\tau) u_k'(\vec{x}, t) \frac{\partial \overline{u_i(\vec{x}, t)}}{\partial x_k}} = \overline{u_k'(\vec{x}, t) u_j'(\vec{x} + \vec{r}, t+\tau)} \frac{\partial \overline{u_i(\vec{x}, t)}}{\partial x_k} = R_{kj}\frac{\partial \overline{u_i(\vec{x}, t)}}{\partial x_k}
	\end{equation}
	
	\begin{equation}
		\begin{aligned}
			\overline{u_j'(\vec{x} + \vec{r}, t+\tau) \frac{\partial [u_i'(\vec{x}, t)u_k'(\vec{x}, t)]}{\partial x_k}} 
			& = \overline{\frac{\partial [u_i'(\vec{x}, t)u_k'(\vec{x}, t) u_j'(\vec{x} + \vec{r}, t+\tau)]}{\partial x_k}} - \overline{u_i'(\vec{x}, t)u_k'(\vec{x}, t) \frac{\partial u_j'(\vec{x} + \vec{r}, t+\tau)}{\partial x_k}}\\
			& = \frac{\partial R_{(ik)j}}{\partial x_k} - \overline{u_i'(\vec{x}, t)u_k'(\vec{x}, t) \frac{\partial u_j'(\vec{x} + \vec{r}, t+\tau)}{\partial r_k}}\\
			& = \frac{\partial R_{(ik)j}}{\partial x_k} - \overline{\frac{\partial [u_i'(\vec{x}, t)u_k'(\vec{x}, t)u_j'(\vec{x} + \vec{r}, t+\tau)]}{\partial r_k}}\\
			& = \frac{\partial R_{(ik)j}}{\partial x_k} - \frac{\partial R_{(ik)j}}{\partial r_k}
		\end{aligned}
	\end{equation}
	
	\begin{equation}
		\overline{u_j'(\vec{x} + \vec{r}, t+\tau) \frac{\partial \overline{u_i'(\vec{x}, t)u_k'(\vec{x}, t)}}{\partial x_k}} = \overline{u_j'(\vec{x} + \vec{r}, t+\tau)} \frac{\partial \overline{u_i'(\vec{x}, t)u_k'(\vec{x}, t)}}{\partial x_k} = 0
	\end{equation}
	
	\begin{equation}
		\begin{aligned}
			\overline{u_j'(\vec{x} + \vec{r}, t+\tau) \frac{\partial p'(\vec{x}, t)}{\partial x_i}} 
			& = \overline{\frac{\partial p'(\vec{x}, t) u_j'(\vec{x} + \vec{r}, t+\tau)}{\partial x_i}} - \overline{p'(\vec{x}, t) \frac{\partial u_j'(\vec{x} + \vec{r}, t+\tau)}{\partial x_i}}\\
			& = \frac{\partial \overline{p'(\vec{x}, t) u_j'(\vec{x} + \vec{r}, t+\tau)}}{\partial x_i} - \overline{p'(\vec{x}, t) \frac{\partial u_j'(\vec{x} + \vec{r}, t+\tau)}{\partial (x_i + r_i)}}\\
			& = \frac{\partial \overline{p'(\vec{x}, t) u_j'(\vec{x} + \vec{r}, t+\tau)}}{\partial x_i} - \overline{p'(\vec{x}, t) \frac{\partial u_j'(\vec{x} + \vec{r}, t+\tau)}{\partial r_i}}\\
			& = \frac{\partial \overline{p'(\vec{x}, t) u_j'(\vec{x} + \vec{r}, t+\tau)}}{\partial x_i} - \overline{\frac{\partial p'(\vec{x}, t) u_j'(\vec{x} + \vec{r}, t+\tau)}{\partial r_i}}\\
			& = \frac{\partial \overline{p'(\vec{x}, t) u_j'(\vec{x} + \vec{r}, t+\tau)}}{\partial x_i} - \frac{\partial \overline{p'(\vec{x}, t) u_j'(\vec{x} + \vec{r}, t+\tau)}}{\partial r_i}
		\end{aligned}
	\end{equation}
	
	\begin{equation}
		\begin{aligned}
			& \overline{u_j'(\vec{x} + \vec{r}, t + \tau) \frac{\partial^2 u_i'(\vec{x}, t)}{\partial x_k \partial x_k}} 
			= \overline{u_j'(\vec{x} + \vec{r}, t + \tau) \frac{\partial}{\partial x_k}\Bigg(\frac{\partial u_i'(\vec{x}, t)}{\partial x_k}\Bigg)}\\
			 = & \overline{\frac{\partial}{\partial x_k} \Bigg(u_j'(\vec{x} + \vec{r}, t + \tau) \frac{\partial u_i'(\vec{x}, t)}{\partial x_k}\Bigg)} - \overline{\frac{\partial u_j'(\vec{x} + \vec{r}, t + \tau)}{\partial x_k} \frac{\partial u_i'(\vec{x}, t)}{\partial x_k}}\\
		= 	& \overline{\frac{\partial}{\partial x_k} \Bigg(\frac{\partial [u_j'(\vec{x} + \vec{r}, t + \tau) u_i'(\vec{x}, t)]}{\partial x_k} - u_i'(\vec{x}, t) \frac{\partial u_j'(\vec{x} + \vec{r}, t + \tau)}{\partial x_k}\Bigg)} - \overline{\frac{\partial u_j'(\vec{x} + \vec{r}, t + \tau)}{\partial x_k} \frac{\partial u_i'(\vec{x}, t)}{\partial x_k}}\\
			 = &\frac{\partial^2 R_{ij}}{\partial x_k \partial x_k} - \overline{\frac{\partial}{\partial x_k} \Bigg(u_i'(\vec{x}, t) \frac{\partial u_j'(\vec{x} + \vec{r}, t + \tau)}{\partial r_k}\Bigg)} - \overline{\frac{\partial u_j'(\vec{x} + \vec{r}, t + \tau)}{\partial x_k} \frac{\partial u_i'(\vec{x}, t)}{\partial x_k}}\\
			 =& \frac{\partial^2 R_{ij}}{\partial x_k \partial x_k} - \overline{\frac{\partial}{\partial x_k} \frac{\partial [u_i'(\vec{x}, t) u_j'(\vec{x} + \vec{r}, t + \tau)]}{\partial r_k}} - \overline{\frac{\partial u_j'(\vec{x} + \vec{r}, t + \tau)}{\partial r_k} \frac{\partial u_i'(\vec{x}, t)}{\partial x_k}}\\
		 =	& \frac{\partial^2 R_{ij}}{\partial x_k \partial x_k} - \frac{\partial^2 R_{ij}}{\partial x_k \partial r_k} - \overline{\Bigg(\frac{\partial}{\partial x_k}\Bigg(u_i'(\vec{x}, t) \frac{\partial u_j'(\vec{x} + \vec{r}, t + \tau)}{\partial r_k} \Bigg) - u_i'(\vec{x}, t) \frac{\partial^2 u_j'(\vec{x} + \vec{r}, t + \tau)}{\partial x_k \partial r_k}\Bigg)}\\
		 = 	&\frac{\partial^2 R_{ij}}{\partial x_k \partial x_k} - \frac{\partial^2 R_{ij}}{\partial x_k \partial r_k} - \overline{\Bigg(\frac{\partial}{\partial x_k}\Bigg(\frac{\partial [u_i'(\vec{x}, t) u_j'(\vec{x} + \vec{r}, t + \tau)]}{\partial r_k} \Bigg) - u_i'(\vec{x}, t) \frac{\partial^2 u_j'(\vec{x} + \vec{r}, t + \tau)}{\partial r_k \partial r_k}\Bigg)}\\
			 = &\frac{\partial^2 R_{ij}}{\partial x_k \partial x_k} - 2\frac{\partial^2 R_{ij}}{\partial x_k \partial r_k} + \overline{\frac{\partial^2 [u_i'(\vec{x}, t)u_j'(\vec{x} + \vec{r}, t + \tau)]}{\partial r_k \partial r_k}}\\
			 =& \frac{\partial^2 R_{ij}}{\partial x_k \partial x_k} - 2\frac{\partial^2 R_{ij}}{\partial x_k \partial r_k} + \frac{\partial^2 R_{ij}}{\partial r_k \partial r_k}\\
		\end{aligned}
	\end{equation}
	
	then, detailed expression for  $D^{(i)}\{R_{ij}\}$ is given as
	\begin{equation}
		\begin{aligned}
			D^{(i)}\{R_{ij}\} = \frac{\partial R_{ij}}{\partial t} - \frac{\partial R_{ij}}{\partial \tau} + \overline{u_k(\vec{x}, t)} \Bigg[\frac{\partial R_{ij}}{\partial x_k}- \frac{\partial R_{ij}}{\partial r_k}\Bigg] + R_{kj}\frac{\partial \overline{u_i(\vec{x}, t)}}{\partial x_k} \\ + \frac{\partial R_{(ik)j}}{\partial x_k} - \frac{\partial R_{(ik)j}}{\partial r_k} + \frac{1}{\rho} \Bigg(\frac{\partial \overline{p'(\vec{x}, t) u_j'(\vec{x} + \vec{r}, t+\tau)}}{\partial x_i} - \frac{\partial \overline{p'(\vec{x}, t) u_j'(\vec{x} + \vec{r}, t+\tau)}}{\partial r_i}\Bigg) \\ - \nu \Bigg(\frac{\partial^2 R_{ij}}{\partial x_k \partial x_k} - 2\frac{\partial^2 R_{ij}}{\partial x_k \partial r_k} + \frac{\partial^2 R_{ij}}{\partial r_k \partial r_k}\Bigg) - \overline{u_j'(\vec{x} + \vec{r}, t+\tau) f_i'(\vec{x}, t)} = 0
		\end{aligned}
	\end{equation}
	
	 For $D^{(j)}\{R_{ij}\} = \overline{u_i'(\vec{x}, t) N_j\{\vec{x} + \vec{r}, t + \tau\}}$, components in the expansion are calculated as
	 \begin{equation}
	 	\overline{u_i'(\vec{x}, t) \frac{\partial u_j'(\vec{x} + \vec{r}, t + \tau)}{\partial (t + \tau)}} = \overline{u_i'(\vec{x}, t) \frac{\partial u_j'(\vec{x} + \vec{r}, t + \tau)}{\partial \tau}} = \overline{ \frac{\partial u_i'(\vec{x}, t) u_j'(\vec{x} + \vec{r}, t + \tau)}{\partial \tau}} = \frac{\partial R_{ij}}{\partial \tau}
	 \end{equation}
	 
	 \begin{equation}
	 	\begin{aligned}
	 		\overline{u_i'(\vec{x}, t) \overline{u_k(\vec{x} + \vec{r}, t + \tau)}\frac{\partial u_j'(\vec{x} + \vec{r}, t + \tau)}{\partial (x_k + r_k)}} 
	 		 = \overline{u_k(\vec{x} + \vec{r}, t + \tau)} \overline{u_i'(\vec{x}, t) \frac{\partial u_j'(\vec{x} + \vec{r}, t + \tau)}{\partial r_k}}\\
	 		 = \overline{u_k(\vec{x} + \vec{r}, t + \tau)} \overline{\frac{\partial [u_i'(\vec{x}, t)u_j'(\vec{x} + \vec{r}, t + \tau)]}{\partial r_k}} = \overline{u_k(\vec{x} + \vec{r}, t + \tau)} \frac{\partial R_{ij}}{\partial r_k}
	 	\end{aligned}
	 \end{equation}
	 
	 \begin{equation}
	 	\overline{u_i'(\vec{x}, t)u_k'(\vec{x} + \vec{r}, t + \tau)\frac{\partial \overline{u_j(\vec{x} + \vec{r}, t + \tau)}}{\partial (x_k+r_k)}} = R_{ik} \frac{\partial \overline{u_j(\vec{x} + \vec{r}, t + \tau)}}{\partial r_k}
	 \end{equation}
	 
	 \begin{equation}
	 	\begin{aligned}
	 		\overline{u_i'(\vec{x}, t) \frac{\partial [u_j'(\vec{x} + \vec{r}, t + \tau)u_k'(\vec{x} + \vec{r}, t + \tau)]}{\partial  (x_k+r_k)}} = \overline{u_i'(\vec{x}, t) \frac{\partial [u_j'(\vec{x} + \vec{r}, t + \tau)u_k'(\vec{x} + \vec{r}, t + \tau)]}{\partial r_k}}\\
	 		= \overline{\frac{\partial [u_i'(\vec{x}, t) u_j'(\vec{x} + \vec{r}, t + \tau)u_k'(\vec{x} + \vec{r}, t + \tau)]}{\partial r_k}} = \frac{\partial R_{i(jk)}}{\partial r_k}
	 	\end{aligned}
	 \end{equation}
	 
	 \begin{equation}
	 	\overline{u_i'(\vec{x}, t) \frac{\partial \overline{u_j'(\vec{x} + \vec{r}, t + \tau)u_k'(\vec{x} + \vec{r}, t + \tau)}}{\partial (x_k+r_k)}} = \overline{u_i'(\vec{x}, t)} \frac{\partial \overline{u_j'(\vec{x} + \vec{r}, t + \tau)u_k'(\vec{x} + \vec{r}, t + \tau)}}{\partial r_k} = 0
	 \end{equation}
	 
	 \begin{equation}
	 	\overline{u_i'(\vec{x}, t) \frac{\partial p'(\vec{x} + \vec{r}, t + \tau)}{\partial (x_j + r_j)}} = \overline{u_i'(\vec{x}, t) \frac{\partial p'(\vec{x} + \vec{r}, t + \tau)}{\partial r_j}} = \overline{\frac{\partial p'(\vec{x} + \vec{r}, t + \tau)u_i'(\vec{x}, t)}{\partial r_j}}
	 \end{equation}
	 
	 \begin{equation}
	 	\begin{aligned}
	 		\overline{u_i'(\vec{x}, t) \frac{\partial^2 u_j'(\vec{x} + \vec{r}, t + \tau)}{\partial (x_k+r_k) \partial (x_k+r_k)}}
	 		&= \overline{u_i'(\vec{x}, t) \frac{\partial^2 u_j'(\vec{x} + \vec{r}, t + \tau)}{\partial r_k \partial r_k}} \\
	 		& = \overline{ \frac{\partial^2 u_i'(\vec{x}, t) u_j'(\vec{x} + \vec{r}, t + \tau)}{\partial r_k \partial r_k}} = \frac{\partial R_{ij}}{\partial r_k \partial r_k}
	 	\end{aligned}
	 \end{equation}
	 
	 then, detailed expression for  $D^{(j)}\{R_{ij}\}$ is given as
	 \begin{equation}
		 \begin{aligned}
		 	D^{(j)}\{R_{ij}\} = \frac{\partial R_{ij}}{\partial \tau} + \overline{u_k(\vec{x} + \vec{r}, t + \tau)} \frac{\partial R_{ij}}{\partial r_k} + R_{ik} \frac{\partial \overline{u_j(\vec{x} + \vec{r}, t + \tau)}}{\partial r_k} + \frac{\partial R_{i(jk)}}{\partial r_k} \\
		 	+ \frac{1}{\rho}\frac{\partial \overline{p'(\vec{x} + \vec{r}, t + \tau)u_i'(\vec{x}, t)}}{\partial r_j} - \nu \frac{\partial R_{ij}}{\partial r_k \partial r_k} - \overline{u_i'(\vec{x}, t) f_j'(\vec{x} + \vec{r}, t + \tau)} = 0
		 \end{aligned}
	 \end{equation}
	 
	 Therefore, the two-point correlation equation is calculated as
	 \begin{equation}
	 	\begin{aligned}
	 		D^{(i)}\{R_{ij}\} + D^{(j)}\{R_{ij}\}
	 		= \frac{\partial R_{ij}}{\partial t}
	 		+ \overline{u_k(\vec{x}, t)} \frac{\partial R_{ij}}{\partial x_k}
	 		+ \Big[\overline{u_k(\vec{x} + \vec{r}, t + \tau)} - \overline{u_k(\vec{x}, t)}\Big]\frac{\partial R_{ij}}{\partial r_k}\\
	 		+ R_{kj}\frac{\partial \overline{u_i(\vec{x}, t)}}{\partial x_k} + R_{ik} \frac{\partial \overline{u_j(\vec{x} + \vec{r}, t + \tau)}}{\partial r_k} 
	 		 + \frac{\partial R_{(ik)j}}{\partial x_k} - \frac{\partial [R_{(ik)j}-R_{i(jk)}]}{\partial r_k}\\
	 		 + \frac{1}{\rho} \Bigg(\frac{\partial \overline{p'(\vec{x}, t) u_j'(\vec{x} + \vec{r}, t+\tau)}}{\partial x_i} - \frac{\partial \overline{p'(\vec{x}, t) u_j'(\vec{x} + \vec{r}, t+\tau)}}{\partial r_i} + \frac{\partial \overline{p'(\vec{x} + \vec{r}, t + \tau)u_i'(\vec{x}, t)}}{\partial r_j}\Bigg) \\ 
	 		 - \nu \Bigg(\frac{\partial^2 R_{ij}}{\partial x_k \partial x_k} - 2\frac{\partial^2 R_{ij}}{\partial x_k \partial r_k} + 2\frac{\partial^2 R_{ij}}{\partial r_k \partial r_k}\Bigg) - \overline{u_j'(\vec{x} + \vec{r}, t+\tau) f_i'(\vec{x}, t)}
	 		 - \overline{u_i'(\vec{x}, t) f_j'(\vec{x} + \vec{r}, t + \tau)} = 0
	 	\end{aligned}
	 \end{equation}
	 
\section{Exercise 2}
	Refer to (\ref{eq:sns}), kinetic energy equation of each velocity component is given as
	\begin{equation}
		\begin{aligned}
			u_1'N_1 & = 0\\
			u_2'N_2 & = 0\\
			u_3'N_3 & = 0\\
		\end{aligned}
		\label{eq:kc}
	\end{equation}
	In the pure shear flow case, assumptions can be made as
	\begin{equation}
		\begin{aligned}
			\frac{\partial}{\partial x_1} & = \frac{\partial}{\partial x_3} = 0\\
			U_2 & = U_3 = 0\\
			U_1 & = f(x_2)
		\end{aligned}
	\end{equation}
	Hence, (\ref{eq:kc}) can be simplified as
	\begin{equation}
		\begin{aligned}
			\frac{\overline{D}}{\overline{D} t}\Big(\frac{u_1'^2}{2}\Big) & = -u_2' u_1'\frac{\partial U_1}{\partial x_2} - u_1'\frac{\partial [u_1'u_2']}{\partial x_2} + u_1'\frac{\partial \overline{u_1' u_2'}}{\partial x_2} + u_1'\nu \frac{\partial^2 u_1'}{\partial x_2 \partial x_2}\\
			\frac{\overline{D}}{\overline{D} t}\Big(\frac{u_2'^2}{2}\Big) & = -u_2'\frac{\partial [u_2'u_2']}{\partial x_2} + u_2'\frac{\partial \overline{u_2' u_2'}}{\partial x_2} - \frac{u_2'}{\rho}\frac{\partial p'}{\partial x_2}+ u_2'\nu \frac{\partial^2 u_2'}{\partial x_2 \partial x_2}\\
			\frac{\overline{D}}{\overline{D} t}\Big(\frac{u_3'^2}{2}\Big) & = -u_3'\frac{\partial [u_3'u_2']}{\partial x_2} + u_3'\frac{\partial \overline{u_3' u_2'}}{\partial x_2} + u_3'\nu \frac{\partial^2 u_3'}{\partial x_2 \partial x_2}\\
		\end{aligned}
		\label{eq:kt}
	\end{equation}
	where
	\begin{equation}
		\frac{\overline{D}}{\overline{D} t} = \frac{\partial}{\partial t} + U_k \frac{\partial}{\partial x_k}
	\end{equation}
	and it should be noted that the $U_k \frac{\partial}{\partial x_k}$ term vanishes in (\ref{eq:kt}).\\
	Estimations on each term can be made respectively.\\
	For the first equation in (\ref{eq:kt})
	\begin{equation}
		\begin{aligned}
			u_2' u_1'\frac{\partial U_1}{\partial x_2} & \sim \frac{u'^2 U}{L}\\
			u_1'\frac{\partial [u_1'u_2']}{\partial x_2} & \sim \frac{u'^3}{L}\\
			u_1'\frac{\partial \overline{u_1' u_2'}}{\partial x_2} & \sim u'\nu \frac{\partial U_1}{\partial x_2} \sim \nu \frac{u' U}{L}\\
			u_1'\nu \frac{\partial^2 u_1'}{\partial x_2 \partial x_2} & \sim \nu \frac{u'^2}{L^2}
		\end{aligned}
	\end{equation}
	
	For the second equation in (\ref{eq:kt})
	\begin{equation}
		\begin{aligned}
			u_2'\frac{\partial [u_2'u_2']}{\partial x_2} & \sim \frac{u'^3}{L}\\
			u_2'\frac{\partial \overline{u_2' u_2'}}{\partial x_2} & \sim \frac{u'^3}{L}\\
			\frac{u_2'}{\rho}\frac{\partial p'}{\partial x_2} & \sim \frac{u'}{\rho} \rho \\
			u_2'\nu \frac{\partial^2 u_2'}{\partial x_2 \partial x_2} & \sim \nu \frac{u'^2}{L^2}
		\end{aligned}
	\end{equation}
	
	For the third equation in (\ref{eq:kt})
	\begin{equation}
		\begin{aligned}
			u_3'\frac{\partial [u_3'u_2']}{\partial x_2} & \sim \frac{u'^3}{L}\\
			u_3'\frac{\partial \overline{u_3' u_2'}}{\partial x_2} & \sim \frac{u'^3}{L}\\
			u_3'\nu \frac{\partial^2 u_3'}{\partial x_2 \partial x_2} & \sim \nu \frac{u'^2}{L^2}
		\end{aligned}
	\end{equation}
	
\end{document}