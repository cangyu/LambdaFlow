\documentclass[paper=a4, fontsize=11pt]{scrartcl} % A4 paper and 11pt font size

\usepackage[T1]{fontenc} % Use 8-bit encoding that has 256 glyphs
\usepackage{fourier} % Use the Adobe Utopia font for the document - comment this line to return to the LaTeX default
\usepackage[english]{babel} % English language/hyphenation
\usepackage{amsmath,amsfonts,amsthm,amssymb} % Math packages

\usepackage{caption}
\usepackage{graphicx, subfig}

\usepackage{algorithm, algorithmic}
\renewcommand{\algorithmicrequire}{\textbf{Input:}} %Use Input in the format of Algorithm  
\renewcommand{\algorithmicensure}{\textbf{Output:}} %UseOutput in the format of Algorithm  

\usepackage{listings}
\lstset{language=Matlab}

\usepackage{lipsum} % Used for inserting dummy 'Lorem ipsum' text into the template

\usepackage{sectsty} % Allows customizing section commands
\allsectionsfont{\centering \normalfont\scshape} % Make all sections centered, the default font and small caps

\usepackage{fancyhdr} % Custom headers and footers
\pagestyle{fancyplain} % Makes all pages in the document conform to the custom headers and footers
\fancyhead{} % No page header - if you want one, create it in the same way as the footers below
\fancyfoot[L]{} % Empty left footer
\fancyfoot[C]{} % Empty center footer
\fancyfoot[R]{\thepage} % Page numbering for right footer
\renewcommand{\headrulewidth}{0pt} % Remove header underlines
\renewcommand{\footrulewidth}{0pt} % Remove footer underlines
\setlength{\headheight}{13.6pt} % Customize the height of the header

\numberwithin{equation}{section} % Number equations within sections (i.e. 1.1, 1.2, 2.1, 2.2 instead of 1, 2, 3, 4)
\numberwithin{figure}{section} % Number figures within sections (i.e. 1.1, 1.2, 2.1, 2.2 instead of 1, 2, 3, 4)
\numberwithin{table}{section} % Number tables within sections (i.e. 1.1, 1.2, 2.1, 2.2 instead of 1, 2, 3, 4)

\setlength\parindent{0pt} % Removes all indentation from paragraphs - comment this line for an assignment with lots of text

\newcommand{\horrule}[1]{\rule{\linewidth}{#1}} % Create horizontal rule command with 1 argument of height

\title{	
\normalfont \normalsize 
\textsc{Shanghai Jiao Tong University, UM-SJTU JOINT INSTITUTE} \\ [25pt] % Your university, school and/or department name(s)
\horrule{0.5pt} \\[0.4cm] % Thin top horizontal rule
\huge Turbulence \\ HW2 \\ % The assignment title
\horrule{2pt} \\[0.5cm] % Thick bottom horizontal rule
}

\author{Yu Cang \\ 018370210001}

\date{\normalsize \today}

\begin{document}

\maketitle

\section{Exercise1}
	Since the structure function and two-point correlation function are defined as
	\begin{equation}
		N_i\{\vec{x}, t\} = \frac{\partial u_i'}{\partial t} + \overline{u_k}\frac{\partial u_i'}{\partial x_k} + u_k' \frac{\partial \overline{u_i}}{x_k} + \frac{\partial (u_i' u_k')}{\partial x_k} - \frac{\partial \overline{u_i' u_k'}}{\partial x_k} + \frac{1}{\rho} \frac{\partial p'}{\partial x_i}-\nu \frac{\partial^2 u_i'}{\partial x_k \partial x_k} - f_i' = 0
	\end{equation}
	and
	\begin{equation}
		R_{ij} = \overline{u_i'(\vec{x}, t) u_j'(\vec{x}+\vec{r}, t + \tau)}
	\end{equation}
	Thus, for $D^{(i)}\{R_{ij}\} = \overline{u_j'(\vec{x} + \vec{r}, t+\tau) N_i\{\vec{x}, t\}}$, components in the expansion are calculated as
	\begin{equation}
		\begin{aligned}
			\overline{u_j'(\vec{x} + \vec{r}, t+\tau) \frac{\partial u_i'(\vec{x}, t)}{\partial t}}  
			& = \overline{\frac{\partial (u_i'(\vec{x}, t)u_j'(\vec{x} + \vec{r}, t+\tau))}{\partial t} - u_i'(\vec{x}, t) \frac{\partial u_j'(\vec{x} + \vec{r}, t+\tau)}{\partial t}}\\
			& = \frac{\partial \overline{u_i'(\vec{x}, t)u_j'(\vec{x} + \vec{r}, t+\tau)}}{\partial t} - \overline{u_i'(\vec{x}, t) \frac{\partial u_j'(\vec{x} + \vec{r}, t+\tau)}{\partial (t+\tau)}}\\
			& = \frac{\partial R_{ij}}{\partial t} - \overline{u_i'(\vec{x}, t) \frac{\partial u_j'(\vec{x} + \vec{r}, t+\tau)}{\partial \tau}}\\
			& = \frac{\partial R_{ij}}{\partial t} - \frac{\partial \overline{u_i'(\vec{x}, t) u_j'(\vec{x} + \vec{r}, t+\tau)}}{\partial \tau}\\
			& = \frac{\partial R_{ij}}{\partial t} - \frac{\partial R_{ij}}{\partial \tau}
		\end{aligned}
	\end{equation} 
	
	\begin{equation}
		\begin{aligned}
			\overline{u_j'(\vec{x} + \vec{r}, t+\tau) \overline{u_k}\frac{\partial u_i'}{\partial x_k}} 
			& = \overline{u_k(\vec{x}, t)} \overline{u_j'(\vec{x} + \vec{r}, t+\tau) \frac{\partial u_i'(\vec{x}, t)}{\partial x_k}}\\
			& = \overline{u_k(\vec{x}, t)} \Bigg[\frac{\partial \overline{u_i'(\vec{x}, t) u_j'(\vec{x} + \vec{r}, t+\tau)}}{\partial x_k}- \overline{u_i'(\vec{x}, t) \frac{\partial u_j'(\vec{x} + \vec{r}, t+\tau)}{\partial (x_k + r_k)}}\Bigg]\\
			& = \overline{u_k(\vec{x}, t)} \Bigg[\frac{\partial R_{ij}}{\partial x_k}- \overline{u_i'(\vec{x}, t) \frac{\partial u_j'(\vec{x} + \vec{r}, t+\tau)}{\partial r_k}}\Bigg]\\
			& = \overline{u_k(\vec{x}, t)} \Bigg[\frac{\partial R_{ij}}{\partial x_k}- \frac{\partial \overline{u_i'(\vec{x}, t) u_j'(\vec{x} + \vec{r}, t+\tau)}}{\partial r_k}\Bigg]\\
			& = \overline{u_k(\vec{x}, t)} \Bigg[\frac{\partial R_{ij}}{\partial x_k}- \frac{\partial R_{ij}}{\partial r_k}\Bigg]
		\end{aligned}
	\end{equation}
	
	\begin{equation}
			\overline{u_j'(\vec{x} + \vec{r}, t+\tau) u_k'(\vec{x}, t) \frac{\partial \overline{u_i(\vec{x}, t)}}{\partial x_k}} = \overline{u_k'(\vec{x}, t) u_j'(\vec{x} + \vec{r}, t+\tau)} \frac{\partial \overline{u_i(\vec{x}, t)}}{\partial x_k} = R_{kj}\frac{\partial \overline{u_i(\vec{x}, t)}}{\partial x_k}
	\end{equation}
	
	\begin{equation}
		\begin{aligned}
			\overline{u_j'(\vec{x} + \vec{r}, t+\tau) \frac{\partial p'(\vec{x}, t)}{\partial x_i}} 
			& = \overline{\frac{\partial p'(\vec{x}, t) u_j'(\vec{x} + \vec{r}, t+\tau)}{\partial x_i}} - \overline{p'(\vec{x}, t) \frac{\partial u_j'(\vec{x} + \vec{r}, t+\tau)}{\partial x_i}}\\
			& = \frac{\partial \overline{p'(\vec{x}, t) u_j'(\vec{x} + \vec{r}, t+\tau)}}{\partial x_i} - \overline{p'(\vec{x}, t) \frac{\partial u_j'(\vec{x} + \vec{r}, t+\tau)}{\partial (x_i + r_i)}}\\
			& = \frac{\partial \overline{p'(\vec{x}, t) u_j'(\vec{x} + \vec{r}, t+\tau)}}{\partial x_i} - \overline{p'(\vec{x}, t) \frac{\partial u_j'(\vec{x} + \vec{r}, t+\tau)}{\partial r_i}}\\
			& = \frac{\partial \overline{p'(\vec{x}, t) u_j'(\vec{x} + \vec{r}, t+\tau)}}{\partial x_i} - \overline{\frac{\partial p'(\vec{x}, t) u_j'(\vec{x} + \vec{r}, t+\tau)}{\partial r_i}}\\
			& = \frac{\partial \overline{p'(\vec{x}, t) u_j'(\vec{x} + \vec{r}, t+\tau)}}{\partial x_i} - \frac{\partial \overline{p'(\vec{x}, t) u_j'(\vec{x} + \vec{r}, t+\tau)}}{\partial r_i}
		\end{aligned}
	\end{equation}
	
	\begin{equation}
		\begin{aligned}
			\overline{u_j'(\vec{x} + \vec{r}, t + \tau) \frac{\partial^2 u_i'(\vec{x}, t)}{\partial x_k \partial x_k}} = 
		\end{aligned}
	\end{equation}
	
	
\section{Exercise 2}
	
	
\end{document}