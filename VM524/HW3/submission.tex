\documentclass[paper=a4, fontsize=11pt]{scrartcl} % A4 paper and 11pt font size

\usepackage[T1]{fontenc} % Use 8-bit encoding that has 256 glyphs
\usepackage{fourier} % Use the Adobe Utopia font for the document - comment this line to return to the LaTeX default
\usepackage[english]{babel} % English language/hyphenation
\usepackage{amsmath,amsfonts,amsthm,amssymb} % Math packages
\usepackage{extarrows}

\usepackage{caption}
\usepackage{graphicx, subfig}

\usepackage{algorithm, algorithmic}
\renewcommand{\algorithmicrequire}{\textbf{Input:}} %Use Input in the format of Algorithm  
\renewcommand{\algorithmicensure}{\textbf{Output:}} %UseOutput in the format of Algorithm  

\usepackage{listings}
\lstset{language=Matlab}

\usepackage{lipsum} % Used for inserting dummy 'Lorem ipsum' text into the template

\usepackage{sectsty} % Allows customizing section commands
\allsectionsfont{\centering \normalfont\scshape} % Make all sections centered, the default font and small caps

\usepackage{fancyhdr} % Custom headers and footers
\pagestyle{fancyplain} % Makes all pages in the document conform to the custom headers and footers
\fancyhead{} % No page header - if you want one, create it in the same way as the footers below
\fancyfoot[L]{} % Empty left footer
\fancyfoot[C]{} % Empty center footer
\fancyfoot[R]{\thepage} % Page numbering for right footer
\renewcommand{\headrulewidth}{0pt} % Remove header underlines
\renewcommand{\footrulewidth}{0pt} % Remove footer underlines
\setlength{\headheight}{13.6pt} % Customize the height of the header

\numberwithin{equation}{section} % Number equations within sections (i.e. 1.1, 1.2, 2.1, 2.2 instead of 1, 2, 3, 4)
\numberwithin{figure}{section} % Number figures within sections (i.e. 1.1, 1.2, 2.1, 2.2 instead of 1, 2, 3, 4)
\numberwithin{table}{section} % Number tables within sections (i.e. 1.1, 1.2, 2.1, 2.2 instead of 1, 2, 3, 4)

\setlength\parindent{0pt} % Removes all indentation from paragraphs - comment this line for an assignment with lots of text

\newcommand{\horrule}[1]{\rule{\linewidth}{#1}} % Create horizontal rule command with 1 argument of height

\title{	
\normalfont \normalsize 
\textsc{Shanghai Jiao Tong University, UM-SJTU JOINT INSTITUTE} \\ [25pt] % Your university, school and/or department name(s)
\horrule{0.5pt} \\[0.4cm] % Thin top horizontal rule
\huge Turbulence \\ HW3 \\ % The assignment title
\horrule{2pt} \\[0.5cm] % Thick bottom horizontal rule
}

\author{Yu Cang \\ 018370210001}

\date{\normalsize \today}

\begin{document}

\maketitle

\section{Exercise1}
	When the gravity is considered, the N-S equation is written as
	\begin{equation}
		\rho \frac{\partial u_j}{\partial t} + \rho u_k \frac{\partial u_j}{\partial x_k} = -\frac{\partial p}{\partial x_j} + \mu \frac{\partial^2 u_j}{\partial x_k \partial x_k} + \rho g_j
	\end{equation}
	Decompose the density into mean and fluctuation as $\rho = \rho_0 + \tilde{\rho}$, and substitute into the equation above yields
	\begin{equation}\label{eq:raw}
		\rho_0 \frac{\partial u_j}{\partial t} + \tilde{\rho} \frac{\partial u_j}{\partial t} + \rho_0 u_k \frac{\partial u_j}{\partial x_k} + \tilde{\rho} \frac{\partial u_j}{\partial x_k} = -\frac{\partial p}{\partial x_j} + \mu \frac{\partial^2 u_j}{\partial x_k \partial x_k} + \rho_0 g_j + \tilde{\rho} g_j
	\end{equation}
 	The 2nd and 4th term in the LHS of (\ref{eq:raw}) can be neglected as $\tilde{\rho} << \rho_0$. Assuming $\frac{\mu}{\rho_0} \approxeq \frac{\mu}{\rho} = \nu$ yields
 	\begin{equation}\label{eq:rs}
 		\frac{\partial u_j}{\partial t} + u_k \frac{\partial u_j}{\partial x_k} = - \frac{1}{\rho_0} \frac{\partial p}{\partial x_j} + \nu \frac{\partial^2 u_j}{\partial x_k \partial x_k} + g_j + g_j \frac{\tilde{\rho}}{\rho_0} 
 	\end{equation}
 	When the EOS for ideal gas($p = \rho R T$) is adopted, $\rho T$ can be assumed to be constant when the change of $p$ is negligible and velocity is small. Hence
 	\begin{equation}
 		(\rho_0 + \tilde{\rho})(\bar{T} + \tilde{T}) = \rho_0 \bar{T}
 	\end{equation}
 	Thus
 	\begin{equation}
 		\frac{\tilde{\rho}}{\rho_0} + \frac{\tilde{T}}{\bar{T}} = 0
 	\end{equation}
 	This is achieved by neglecting the 2nd-order small quantities.
 	\newpage
 	Substitute into (\ref{eq:rs}) yields
 	\begin{equation}\label{eq:boussineq}
 		\frac{\partial u_j}{\partial t} + u_k \frac{\partial u_j}{\partial x_k} = - \frac{1}{\rho_0} \frac{\partial p}{\partial x_j} + \nu \frac{\partial^2 u_j}{\partial x_k \partial x_k} + g_j - g_j \frac{\tilde{T}}{\bar{T}}
 	\end{equation}
 	Decompose the velocity and pressure into mean and fluctuation as
 	\begin{equation}
 		\begin{aligned}
 			u_j & = U_j + u_j'\\
 			p & = P + p'
 		\end{aligned}
 	\end{equation}
 	Substitute into (\ref{eq:boussineq})
 	\begin{equation}
 		\frac{\partial (U_j + u_j')}{\partial t} + (U_k + u_k') \frac{\partial (U_j + u_j')}{\partial x_k} = - \frac{1}{\rho_0} \frac{\partial (P + p')}{\partial x_j} + \nu \frac{\partial^2 (U_j + u_j')}{\partial x_k \partial x_k} + g_j - g_j \frac{\tilde{T}}{\bar{T}}
 	\end{equation}
 	Multiply $u_j'$ and take ensemble average yields
 	\begin{equation}\label{eq:ea}
 		LHS \triangleq \overline{u_j' \frac{\partial u_j'}{\partial t}} + U_k \overline{u_j' \frac{\partial u_j'}{\partial x_k}} + \overline{u_j' u_k'}\frac{\partial U_j}{\partial x_k} + \overline{u_j' u_k' \frac{\partial u_j'}{\partial x_k}} = -\frac{1}{\rho_0} \overline{u_j' \frac{\partial p'}{\partial x_j}} + \nu \overline{u_j'\frac{\partial^2 u_j'}{\partial x_k \partial x_k}} + \frac{g_j}{\bar{T}}\overline{u_j' \tilde{T}} \triangleq RHS
 	\end{equation}
 	Denote $k_T = \frac{1}{2} \overline{u_j' u_j'}$ and follow from the continuity equation
 	\begin{equation}\label{eq:bl}
 		LHS = \frac{\bar{D}k_T}{\bar{D}t} + \overline{u_j' u_k'}\frac{\partial U_j}{\partial x_k} + \frac{1}{2}\frac{\partial}{\partial x_k}\overline{u_k' u_j' u_j'}
 	\end{equation}
	where
 	\begin{equation}
 	\frac{\bar{D}}{\bar{D}t} = \frac{\partial}{\partial t} + U_k \frac{\partial}{\partial x_k}
 	\end{equation}
 	Since
 	\begin{equation}
 		\overline{u_j'\frac{\partial^2 u_j'}{\partial x_k \partial x_k}} = \overline{\frac{\partial}{\partial x_k} \big(u_j'\frac{\partial u_j'}{\partial x_k}\big)} - \overline{\frac{\partial u_j'}{\partial x_k} \frac{\partial u_j'}{\partial x_k}} = \frac{\partial^2 k_T}{\partial x_k \partial x_k} - \overline{\frac{\partial u_j'}{\partial x_k} \frac{\partial u_j'}{\partial x_k}}
 	\end{equation}
 	then, also follows from the continuity equation
 	\begin{equation}\label{eq:br}
 		RHS = -\frac{1}{\rho_0} \frac{\partial \overline{u_j' p'}}{\partial x_j} + \nu \Big(\frac{\partial^2 k_T}{\partial x_k \partial x_k} - \overline{\frac{\partial u_j'}{\partial x_k} \frac{\partial u_j'}{\partial x_k}}\Big) + \frac{g_j}{\bar{T}}\overline{u_j' \tilde{T}}
 	\end{equation}
 	Combining (\ref{eq:bl}) and (\ref{eq:br}) yields the turbulent kinetic energy equation for the buoyancy case
 	\begin{equation}
 		\frac{\bar{D}k_T}{\bar{D}t} = -\overline{u_j' u_k'}\frac{\partial U_j}{\partial x_k} - \frac{1}{2}\frac{\partial}{\partial x_k}\overline{u_k' u_j' u_j'} -\frac{1}{\rho_0} \frac{\partial \overline{u_j' p'}}{\partial x_j} + \nu \Big(\frac{\partial^2 k_T}{\partial x_k \partial x_k} - \overline{\frac{\partial u_j'}{\partial x_k} \frac{\partial u_j'}{\partial x_k}}\Big) + \frac{g_j}{\bar{T}}\overline{u_j' \tilde{T}}
 	\end{equation}
 	\newpage
 
\section{Exercise 2}
	The governing equation for the passive scalar $\phi$ is
	\begin{equation}\label{eq:passive}
		\frac{D \phi}{D t} = \Gamma \triangledown^2 \phi
	\end{equation}
	The gradient of (\ref{eq:passive}) is
	\begin{equation}
		\frac{\partial \triangledown \phi}{\partial t} + \triangledown(\vec{U}\cdot\triangledown\phi) = \Gamma \triangledown^2(\triangledown\phi)
	\end{equation}
	An vector operation identity is used
	\begin{equation}
		\triangledown(\vec{f}\cdot\vec{g}) = (\vec{g}\cdot\triangledown)\vec{f} + \vec{g} \times (\triangledown \times \vec{f}) + (\vec{f} \cdot \triangledown)\vec{g} + \vec{f} \times (\triangledown \times \vec{g})
	\end{equation}
	Thus
	\begin{equation}
		\begin{aligned}
			\triangledown(\vec{U} \cdot \triangledown\phi) 
			& = (\triangledown\phi \cdot \vec{U})\vec{U} + \triangledown\phi \times (\triangledown \times \vec{U}) + (\vec{U} \cdot \triangledown) \triangledown \phi + \vec{U} \times (\triangledown \times \triangledown\phi)\\
			& = (\triangledown\phi \cdot \vec{U})\vec{U} + \triangledown\phi \times (\triangledown \times \vec{U}) + (\vec{U} \cdot \triangledown) \triangledown \phi
		\end{aligned}
	\end{equation}
	Hence, the governing equation for $\triangledown\phi$ is
	\begin{equation}\label{eq:gp}
		\frac{D}{Dt}\triangledown\phi = -(\vec{U} \cdot \triangledown\phi)\vec{U} - \triangledown\phi \times (\triangledown \times \vec{U}) + \Gamma \triangledown^2(\triangledown\phi)  
	\end{equation}
	Denote $\vec{W} = \triangledown \phi$, then (\ref{eq:gp}) can be re-written as
	\begin{equation}
		\frac{D}{Dt} \vec{W} = -(\vec{U} \cdot \vec{W})\vec{U} - \vec{W} \times (\triangledown \times \vec{U}) + \Gamma \triangledown^2 \vec{W} \triangleq RHS
	\end{equation}
	Inner product with $\vec{W}$, the 2nd term in $RHS$ vanishes, remaining parts yields
	\begin{equation}
		\frac{D}{Dt} \big(\frac{1}{2}|\vec{W}|^2\big) = -(\vec{U} \cdot \vec{W})^2  + \Gamma \vec{W} \triangledown^2 \vec{W}
	\end{equation}
	since
	\begin{equation}
			\vec{W} \triangledown^2 \vec{W} = \triangledown^2 \big(\frac{1}{2}|\vec{W}|^2\big) - \triangledown \vec{W} : \triangledown \vec{W}
	\end{equation}
	then, the governing equation for the scalar gradient energy $\triangledown\phi \cdot \triangledown\phi$ is
	\begin{equation}
		\frac{D}{Dt} \big(\frac{1}{2}|\vec{W}|^2\big) = -(\vec{U} \cdot \vec{W})^2  + \Gamma \Bigg(\triangledown^2 \big(\frac{1}{2}|\vec{W}|^2\big) - \triangledown \vec{W} : \triangledown \vec{W}\Bigg)
	\end{equation}
	Characteristic scales of dissipation term is estimated as
	\begin{equation}
		\Gamma \triangledown^2(\frac{1}{2} |\vec{W}|^2) \sim \Gamma \frac{\phi^2}{L^4}
	\end{equation}
\end{document}