\documentclass[paper=a4, fontsize=11pt]{scrartcl} % A4 paper and 11pt font size

\usepackage[T1]{fontenc} % Use 8-bit encoding that has 256 glyphs
\usepackage{fourier} % Use the Adobe Utopia font for the document - comment this line to return to the LaTeX default
\usepackage[english]{babel} % English language/hyphenation
\usepackage{amsmath,amsfonts,amsthm,amssymb} % Math packages

\usepackage{algorithm, algorithmic}
\renewcommand{\algorithmicrequire}{\textbf{Input:}} %Use Input in the format of Algorithm  
\renewcommand{\algorithmicensure}{\textbf{Output:}} %UseOutput in the format of Algorithm  

\usepackage{graphicx}

\usepackage{listings}
\lstset{language=Matlab}

\usepackage{lipsum} % Used for inserting dummy 'Lorem ipsum' text into the template

\usepackage{sectsty} % Allows customizing section commands
\allsectionsfont{\centering \normalfont\scshape} % Make all sections centered, the default font and small caps

\usepackage{fancyhdr} % Custom headers and footers
\pagestyle{fancyplain} % Makes all pages in the document conform to the custom headers and footers
\fancyhead{} % No page header - if you want one, create it in the same way as the footers below
\fancyfoot[L]{} % Empty left footer
\fancyfoot[C]{} % Empty center footer
\fancyfoot[R]{\thepage} % Page numbering for right footer
\renewcommand{\headrulewidth}{0pt} % Remove header underlines
\renewcommand{\footrulewidth}{0pt} % Remove footer underlines
\setlength{\headheight}{13.6pt} % Customize the height of the header

\numberwithin{equation}{section} % Number equations within sections (i.e. 1.1, 1.2, 2.1, 2.2 instead of 1, 2, 3, 4)
\numberwithin{figure}{section} % Number figures within sections (i.e. 1.1, 1.2, 2.1, 2.2 instead of 1, 2, 3, 4)
\numberwithin{table}{section} % Number tables within sections (i.e. 1.1, 1.2, 2.1, 2.2 instead of 1, 2, 3, 4)

\setlength\parindent{0pt} % Removes all indentation from paragraphs - comment this line for an assignment with lots of text

%----------------------------------------------------------------------------------------
%	TITLE SECTION
%----------------------------------------------------------------------------------------

\newcommand{\horrule}[1]{\rule{\linewidth}{#1}} % Create horizontal rule command with 1 argument of height

\title{	
\normalfont \normalsize 
\textsc{Shanghai Jiao Tong University, UM-SJTU JOINT INSTITUTE} \\ [25pt] % Your university, school and/or department name(s)
\horrule{0.5pt} \\[0.4cm] % Thin top horizontal rule
\huge Methods of Applied Mathematics I\\ HW4 \\ % The assignment title
\horrule{2pt} \\[0.5cm] % Thick bottom horizontal rule
}

\author{Yu Cang \\ 018370210001} % Your name

\date{\normalsize \today} % Today's date or a custom date

\begin{document}

\maketitle % Print the title

\section{Exercise4.1}
	Let $f(x)$ be extended as
	\begin{equation}
		f(x) = \left\{
			\begin{aligned}
				x(\pi - x) & x\in [2n\pi, (2n+1)\pi] \\
				-x(\pi - x) & x\in [-(2n-1)\pi, 2n\pi] \\
			\end{aligned}
		\right.
	\end{equation}
	Then $f(x)$ is both odd and periodic. Thus fouier-sine series can be employed.
	\begin{equation}
		f(x) = \sum_{n=1}^{\infty} b_n sin(nx)
	\end{equation} 
	Coefficients $b_n$ are calculated by
	\begin{equation}
		\begin{aligned}
			b_n  & = \frac{2}{\pi} \int_{0}^{\pi} f(x) sin(nx) dx \\
				 & = \frac{2}{\pi} \int_{0}^{\pi} x(\pi - x) sin(nx) dx \\
			     & = \frac{4[1-(-1)^n]}{n^3 \pi} \quad \text{(Integrate by parts)}
		\end{aligned}
	\end{equation}
	Thus
	\begin{equation}
		f(x) = \sum_{k=0}^{\infty} \frac{8 sin(2k+1)x}{\pi (2k+1)^3}
	\end{equation}
	Taking $x = \frac{\pi}{2}$ yields
	\begin{equation}
		\sum_{k=0}^{\infty} \frac{(-1)^k}{(2k+1)^3} = \frac{\pi^3}{32}
	\end{equation}

\section{Exercise4.2}
	\begin{enumerate}
		\item 
			\begin{proof}
				The orthogonal property is justified as
				\begin{equation}
					\int_{0}^{\pi} (\frac{1}{\sqrt{\pi}})^2 dx = \frac{1}{\pi} \int_{0}^{\pi} dx = 1
				\end{equation}
				\begin{equation}
					\int_{0}^{\pi} (\sqrt{\frac{2}{\pi}}cos(nx))^2 dx = \frac{2}{\pi} \int_{0}^{\pi} cos^2(nx)dx = \frac{1}{\pi} \int_{0}^{\pi} (cos(2nx) + 1) dx = 1
				\end{equation}
				\begin{equation}
					\int_{0}^{\pi} \frac{1}{\sqrt{\pi}} \sqrt{\frac{2}{\pi}}cos(nx) dx = 0
				\end{equation}
				\begin{equation}
					\int_{0}^{\pi} \sqrt{\frac{2}{\pi}}cos(nx) \sqrt{\frac{2}{\pi}}cos(mx) dx = \frac{2}{\pi} \int_{0}^{\pi} cos(nx) cos(mx) dx = 0
				\end{equation}
			\end{proof}
		\item 
			\begin{proof}
				It's trival to show both $K=0$ and $K=1$ are valid, and $K=2$ is also justified as
				\begin{equation}
					cos^2(x) = \frac{1+cos(2x)}{2}
				\end{equation}
				Assume the proposition is also valid for $K=n$, that means
				\begin{equation}
					span\{1, cos(x), cos(2x), ... cos(nx)\} = span\{1, cos(x), cos^2(x), ... cos^n(x)\}
				\end{equation}
				which indicates that $\exists a_k$ and $b_k (k=0, 1, ..., n)$ s.t. 
				\begin{equation}
					cos^n(x) = \sum_{k=0}^{n} a^{(n)}_k \cdot cos(kx)
				\end{equation}
				\begin{equation}
					cos(nx) = \sum_{k=0}^{n} b^{(n)}_k \cdot cos^k(x)
				\end{equation}
				When $K = n+1$, the proposition is still valid as
				\begin{equation}
					\begin{aligned}
						cos^{n+1}(x) & = cos(x) \sum_{k=0}^{n} a_k cos(kx)\\
									 & = a_0 cos(x) + cos(x) \sum_{k=1}^{n-1} a_k cos(kx) + a_n cos(x) cos(nx)\\
									 & = a_0 cos(x) + \sum_{k=1}^{n-1} \frac{a_k}{2} [cos(k-1)x + cos(k+1)x] + \frac{a_n}{2} [cos(n-1)x + cos(n+1)x]\\
									 & = \frac{a_1}{2} + (a_0 + \frac{a_2}{2}) cos(x) + \sum_{k=2}^{n-1}\frac{a_{k-1} + a_{k+1}}{2} cos(kx) + \frac{a_n}{2} cos(n+1)x
					\end{aligned}
				\end{equation}
				and
				\begin{equation}
					\begin{aligned}
						cos(n+1)x & = 2 cos(nx)cos(x) - cos(n-1)x\\
								  & = 2 cos(x)\sum_{k=0}^{n} b^{(n)}_k cos^k(x) - \sum_{k=0}^{n-1} b^{(n-1)}_k cos^k(x)\\
								  & = -b^{(n-1)}_0 + \sum_{k=0}^{n-1} (2b^{(n)}_k - b^{(n-1)}_k) cos^{k+1}(x) + 2b^{(n)}_k cos^{n+1}(x)
					\end{aligned}
				\end{equation}
			\end{proof}
		\item 
			
		
		\item 
		
		\item 
		
		\item 
		
	\end{enumerate}

\end{document}