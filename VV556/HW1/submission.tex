\documentclass[paper=a4, fontsize=11pt]{scrartcl} % A4 paper and 11pt font size

\usepackage[T1]{fontenc} % Use 8-bit encoding that has 256 glyphs
\usepackage{fourier} % Use the Adobe Utopia font for the document - comment this line to return to the LaTeX default
\usepackage[english]{babel} % English language/hyphenation
\usepackage{amsmath,amsfonts,amsthm,amssymb} % Math packages

\usepackage{algorithm, algorithmic}
\renewcommand{\algorithmicrequire}{\textbf{Input:}} %Use Input in the format of Algorithm  
\renewcommand{\algorithmicensure}{\textbf{Output:}} %UseOutput in the format of Algorithm  

\usepackage{listings}
\lstset{language=Matlab}

\usepackage{lipsum} % Used for inserting dummy 'Lorem ipsum' text into the template

\usepackage{sectsty} % Allows customizing section commands
\allsectionsfont{\centering \normalfont\scshape} % Make all sections centered, the default font and small caps

\usepackage{fancyhdr} % Custom headers and footers
\pagestyle{fancyplain} % Makes all pages in the document conform to the custom headers and footers
\fancyhead{} % No page header - if you want one, create it in the same way as the footers below
\fancyfoot[L]{} % Empty left footer
\fancyfoot[C]{} % Empty center footer
\fancyfoot[R]{\thepage} % Page numbering for right footer
\renewcommand{\headrulewidth}{0pt} % Remove header underlines
\renewcommand{\footrulewidth}{0pt} % Remove footer underlines
\setlength{\headheight}{13.6pt} % Customize the height of the header

\numberwithin{equation}{section} % Number equations within sections (i.e. 1.1, 1.2, 2.1, 2.2 instead of 1, 2, 3, 4)
\numberwithin{figure}{section} % Number figures within sections (i.e. 1.1, 1.2, 2.1, 2.2 instead of 1, 2, 3, 4)
\numberwithin{table}{section} % Number tables within sections (i.e. 1.1, 1.2, 2.1, 2.2 instead of 1, 2, 3, 4)

\setlength\parindent{0pt} % Removes all indentation from paragraphs - comment this line for an assignment with lots of text

%----------------------------------------------------------------------------------------
%	TITLE SECTION
%----------------------------------------------------------------------------------------

\newcommand{\horrule}[1]{\rule{\linewidth}{#1}} % Create horizontal rule command with 1 argument of height

\title{	
\normalfont \normalsize 
\textsc{Shanghai Jiao Tong University, UM-SJTU JOINT INSTITUTE} \\ [25pt] % Your university, school and/or department name(s)
\horrule{0.5pt} \\[0.4cm] % Thin top horizontal rule
\huge Methods of Applied Mathematics I\\ HW1 \\ % The assignment title
\horrule{2pt} \\[0.5cm] % Thick bottom horizontal rule
}

\author{Yu Cang \\ 018370210001} % Your name

\date{\normalsize \today} % Today's date or a custom date

\begin{document}

\maketitle % Print the title

\section{Exercise1.1}
\begin{enumerate}
	\item 
		\begin{equation}
			\dim{U} = 2
		\end{equation}
		As there are 4 components within $x$, and 2 constraints.
	\item 
		\begin{equation}
			\dim{U} = 3
		\end{equation}
		As there are 4 components within $x$, and 1 constraint.
	\item 
		\begin{equation}
			\begin{aligned}
				\dim{U+V} & = \dim{U} + \dim{V} - \dim{U \cap V} \\
				          & = 2 + 3 - 2 \\
				          & = 3
			\end{aligned}
		\end{equation}
		The dimension of $U \cap V$ can be observed from
		\begin{equation}
			\begin{aligned}
				U \cap V & = \{x\in \mathbb{R}^4| x_1 + x_2 + x_3 = 0, x_1 + 3 x_2 = x_4, x_1 = x_4\}\\
				         & = \{x\in \mathbb{R}^4| x_2 = 0, x_1 + x_3 = 0\}
			\end{aligned}
		\end{equation}
\end{enumerate}

\section{Exercise1.2}
\begin{enumerate}
	\item
		The pointwise limit is given as follows
		\begin{equation}
			f = \left\{\begin{aligned}
			0, & x = 0\\
			1, & x\in(0,1]
			\end{aligned}
			\right.
		\end{equation}
		It's not of uniform convergence as
		\begin{equation}
			\sup_{x\in[0, 1]} |f_n(x) - f(x)| = 1
		\end{equation}
	\item 
		The pointwise limit is given as
		\begin{equation}
			f(x) = x
		\end{equation}
		It's not of uniform convergence.
	\item 
		The pointwise limit is given as
		\begin{equation}
			f(x) = 0
		\end{equation}
		It's of uniform convergence as
		\begin{equation}
			\begin{aligned}
				&	\lim\limits_{n\rightarrow\infty}\sup_{x\in(0, \infty)}|f_n(x) - f(x)|\\
				& =\lim\limits_{n\rightarrow\infty}\sup_{x\in(0, \infty)}\Biggl|\sqrt{\frac{1}{n} + x} - \sqrt{x}\Biggr|\\
				& =\lim\limits_{n\rightarrow\infty}\sup_{x\in(0, \infty)}\Biggl|\frac{1}{n(\sqrt{\frac{1}{n} + x} + \sqrt{x})}\Biggr|\\
				& = \lim\limits_{n\rightarrow\infty, x \rightarrow 0}\frac{1}{n(\sqrt{\frac{1}{n} + x} + \sqrt{x})}\\
				& = \lim\limits_{n\rightarrow\infty, x \rightarrow 0}\frac{1}{2n\sqrt{x}} = 0
			\end{aligned}
		\end{equation}
\end{enumerate}

\section{Exercise1.3}
	\begin{enumerate}
		\item 
			\begin{proof}
				Given a sequence $(a_n) \in l^p$, then $\lim\limits_{n\rightarrow \infty} |a_n|^p =0$.\\
				Thus, $\lim\limits_{n\rightarrow \infty} |a_n|^{q-p} = 0$, which indicates that $|a_n|^{q-p}$ is bounded.\\
				Hence $\exists C > 0$ s.t. $|a_n|^{q-p} < C$, which results in
				\begin{equation}
					\sum_{n=0}^{\infty} |a_n|^q = \sum_{n=0}^{\infty} |a_n|^p \cdot |a_n|^{q-p} < C \sum_{n=0}^{\infty} |a_n|^p < \infty 
				\end{equation}
				Therefore $(a_n) \in l^q$.
			\end{proof}
		\item 
			For example
			\begin{equation}
				a_n = \frac{1}{n}
			\end{equation}
		\item 
			Consider a sequence constructed as follows
			\begin{equation}
				\begin{aligned}
					& x_1 = \frac{1}{1} & (1^1=1 \ item)\\
					& x_2 = ... = x_5 = \frac{1}{2} & (2^2=4 \ items)\\
					& x_6 = ... = x_{32} = \frac{1}{3} & (3^3=27 \ items)\\
					& ...
				\end{aligned}
			\end{equation}
			This sequence certainly converges to 0. But not in $L^p$ for any $p>=1$ as
			\begin{equation}
				\sum_{n=1}^{\infty} = 1 + \frac{2^2}{2^p} + \frac{3^3}{3^p} + ... + \frac{n^n}{n^p} + ...
			\end{equation}
			for any $n > p$, $\frac{n^n}{n^p} > 1$.
	\end{enumerate}


\end{document}