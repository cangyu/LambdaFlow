\documentclass[paper=a4, fontsize=11pt]{scrartcl} % A4 paper and 11pt font size

\usepackage[T1]{fontenc} % Use 8-bit encoding that has 256 glyphs
\usepackage{fourier} % Use the Adobe Utopia font for the document - comment this line to return to the LaTeX default
\usepackage[english]{babel} % English language/hyphenation
\usepackage{amsmath,amsfonts,amsthm,amssymb} % Math packages

\usepackage{algorithm, algorithmic}
\renewcommand{\algorithmicrequire}{\textbf{Input:}} %Use Input in the format of Algorithm  
\renewcommand{\algorithmicensure}{\textbf{Output:}} %UseOutput in the format of Algorithm  

\usepackage{graphicx}

\usepackage{listings}
\lstset{language=Matlab}

\usepackage{lipsum} % Used for inserting dummy 'Lorem ipsum' text into the template

\usepackage{sectsty} % Allows customizing section commands
\allsectionsfont{\centering \normalfont\scshape} % Make all sections centered, the default font and small caps

\usepackage{fancyhdr} % Custom headers and footers
\pagestyle{fancyplain} % Makes all pages in the document conform to the custom headers and footers
\fancyhead{} % No page header - if you want one, create it in the same way as the footers below
\fancyfoot[L]{} % Empty left footer
\fancyfoot[C]{} % Empty center footer
\fancyfoot[R]{\thepage} % Page numbering for right footer
\renewcommand{\headrulewidth}{0pt} % Remove header underlines
\renewcommand{\footrulewidth}{0pt} % Remove footer underlines
\setlength{\headheight}{13.6pt} % Customize the height of the header

\numberwithin{equation}{section} % Number equations within sections (i.e. 1.1, 1.2, 2.1, 2.2 instead of 1, 2, 3, 4)
\numberwithin{figure}{section} % Number figures within sections (i.e. 1.1, 1.2, 2.1, 2.2 instead of 1, 2, 3, 4)
\numberwithin{table}{section} % Number tables within sections (i.e. 1.1, 1.2, 2.1, 2.2 instead of 1, 2, 3, 4)

\setlength\parindent{0pt} % Removes all indentation from paragraphs - comment this line for an assignment with lots of text

%----------------------------------------------------------------------------------------
%	TITLE SECTION
%----------------------------------------------------------------------------------------

\newcommand{\horrule}[1]{\rule{\linewidth}{#1}} % Create horizontal rule command with 1 argument of height

\title{	
\normalfont \normalsize 
\textsc{Shanghai Jiao Tong University, UM-SJTU JOINT INSTITUTE} \\ [25pt] % Your university, school and/or department name(s)
\horrule{0.5pt} \\[0.4cm] % Thin top horizontal rule
\huge Methods of Applied Mathematics I\\ HW8 \\ % The assignment title
\horrule{2pt} \\[0.5cm] % Thick bottom horizontal rule
}

\author{Yu Cang \quad 018370210001\\ Zhiming Cui \quad 017370910006} % Your name

\date{\normalsize \today} % Today's date or a custom date

\begin{document}

\maketitle % Print the title

\section{Exercise8.1}
	\begin{enumerate}
		\item 
			\begin{proof}
				Since
				\begin{equation}
					\sum_{i=1}^{\infty}\sum_{j=1}^{\infty}|a_{ij}|^2 = \sum_{k=1}^{\infty} \frac{1}{k} \rightarrow \infty
				\end{equation}
				Thus, $L$ is not a Hilbert-Schmidt operator.
			\end{proof}
		\item 
			\begin{proof}
				$\forall x, y \in l^2$, the inner product can be expressed as
				\begin{equation}
					\begin{aligned}
						<x, Ly> & =  <(x_1, x_2, ... , x_n, ...), (y_1, \frac{y_2}{\sqrt{2}}, ... , \frac{y_n}{\sqrt{n}}, ...)>\\
								& = \sum_{i=1}^{\infty} \frac{x_i y_i}{\sqrt{i}}\\
								& = <(x_1, \frac{x_2}{\sqrt{2}}, ... , \frac{x_n}{\sqrt{n}}, ...), (y_1, y_2, ... , y_n, ...)>\\
								& = <Lx, y>
					\end{aligned}
				\end{equation}
				Thus, $L$ is self adjoint.
			\end{proof}
		\item 
			\begin{proof}
				Denote $L_n : l^2 \rightarrow l^2$ by
				\begin{equation}
					L_n(x_n) = (x_1, \frac{x_2}{\sqrt{2}}, \frac{x_3}{\sqrt{3}}, ... , \frac{x_n}{\sqrt{n}}, 0, 0, ...)
				\end{equation}
				$L_n$ is a Hilbert-Schmidt operator as 
				\begin{equation}
					\sum_{i=1}^{\infty}\sum_{j=1}^{\infty}|a_{ij}|^2 = \sum_{k=1}^{n} \frac{1}{k} < \infty
				\end{equation}
				Thus, $L_n$ is compact. Then, $L$ is compact if $(L_n)$ converges to $L$ in norm, i.e.
				\begin{equation}
					\lim\limits_{n \rightarrow \infty} ||L_n - L|| = 0
				\end{equation}
				To show this, expand the operator norm as
				\begin{equation}
					\begin{aligned}
						||L_n - L||& = \sup\limits_{x\in l^2}\frac{||(L_n - L)x||}{||x||}\\
						& = \sup\limits_{x\in l^2}\frac{||(0, ... , 0, \frac{x_{n+1}}{\sqrt{n+1}}, \frac{x_{n+2}}{\sqrt{n+2}},...)||}{||x||}\\
						& \leq \frac{1}{\sqrt{n+1}} \frac{||(0, ... , 0, x_{n+1}, x_{n+2}, ...)||}{||x||}\\
						& \leq  \frac{1}{\sqrt{n+1}}
					\end{aligned}
				\end{equation}
				So, $\lim\limits_{n \rightarrow \infty} ||L_n - L|| \leq  \lim\limits_{n \rightarrow \infty} \frac{1}{\sqrt{n+1}} = 0$.\\
				According to the positivity of norm,  $\lim\limits_{n \rightarrow \infty} ||L_n - L||\geq 0$.\\
				Hence  $\lim\limits_{n \rightarrow \infty} ||L_n - L|| = 0$.
			\end{proof}
		\item
			Since
			\begin{equation}
				||L|| = \sup\limits_{x\in l^2} \frac{||Lx||}{||x||} \geq \frac{||Le_1||}{||e_1||} = 1
			\end{equation}
			and
			\begin{equation}
				||L|| = \sup\limits_{x\in l^2} \frac{||Lx||}{||x||} \leq \frac{||x||}{||x||} = 1
			\end{equation}
			Thus $||L|| = 1$.\\
			As the norm of operator is a bound for all its eigenvalues, then $|\lambda| \leq 1$. Thus the upper bound for the spcetrum is found.\\
			For the lower bound, Rayleigh Quotient is used. For $x\neq 0$, 
			\begin{equation}
				\lambda \geq L_T \triangleq \inf\limits_{x\in dom(L)} R(x) = \inf\limits_{x\in dom(L)} \frac{<x, Lx>}{||x||^2} = \lim\limits_{n \rightarrow \infty} \frac{<e_n, Le_n>}{||e_n||^2} = \lim\limits_{n \rightarrow \infty} \frac{1}{\sqrt{n}} = 0
			\end{equation}
			Thus the lower bound for the spcetrum is found.
		\item 
			Clearly, $\{1, \frac{1}{\sqrt{2}}, \frac{1}{\sqrt{3}}, ...\}$ is the point specturm as they are eigenvalues of $L$.

	\end{enumerate}


\section{Exercise8.2}
	\begin{enumerate}
		\item 
			\begin{proof}
				$\forall u \in M$, with the B.C. of $u$, the transform by operator $KL$ yields
				\begin{equation}
					\begin{aligned}
						(KL)u(x) & = \int_{0}^{1} g(x, \xi) (Lu)(\xi) d\xi\\
					 			 & = \int_{0}^{x} g(x, \xi) (Lu)(\xi) d\xi + \int_{x}^{1} g(x, \xi) (Lu)(\xi) d\xi\\
					 			 & = \int_{0}^{x} \xi(x-1) u^{\prime\prime}(\xi) d\xi + \int_{x}^{1} x(\xi - 1) u^{\prime\prime}(\xi) d\xi\\
					 			 & = \int_{0}^{x} \xi(x-1) du^{\prime}(\xi) + \int_{x}^{1} x(\xi - 1) du^{\prime}(\xi)\\
					 			 & = (\xi(x-1)u')\Big|_0^x - \int_{0}^{x} u' (x-1) d\xi + (x(\xi-1)u')\Big|_x^1 - \int_{x}^{1} u' x d\xi\\
					 			 & = x(x-1)u'(x) - (x-1) (u(x) - u(0)) - x(x-1)u'(x) - x (u(1)-u(x))\\
					 			 & = u(x)
					\end{aligned}
				\end{equation}
				Thus, $KL = I$ on $M$.
			\end{proof}
		\item 
			\begin{proof}
				For example, take $u(x) = \sqrt{x}$, then
				\begin{equation}
					||L|| \geq \frac{||Lu||}{||u||} = \frac{||x^{-\frac{3}{2}}||}{4||x^\frac{1}{2}||} \rightarrow \infty
				\end{equation}
				Thus, $L$ is unbounded.
			\end{proof}
		\item 

		\item 
			\begin{proof}
				Swap $\xi$ and $x$, then
				\begin{equation}
					g(\xi, x) = 
					\left\{
						\begin{aligned}
							\xi(1-x) & \quad \xi<x\\
							x(1-\xi) & \quad \xi\geq x
						\end{aligned}
					\right.
					=
					\left\{
						\begin{aligned}
							x(1-\xi) & \quad x < \xi\\
							\xi(1-x) & \quad x \geq \xi
						\end{aligned}
					\right.
					=g(x,\xi)
				\end{equation}
				Thus, $\forall u, v\in M$,
				\begin{equation}
					\begin{aligned}
						<u, Kv> & = \int_{0}^{1} u(x) \Bigg(\int_{0}^{1} g(x, \xi) v(\xi) d\xi\Bigg) dx\\
								& = \int_{0}^{1} \int_{0}^{1} u(x) g(\xi, x) v(\xi) d\xi dx\\
								& = \int_{0}^{1} \Bigg(\int_{0}^{1} g(\xi, x) u(x) dx\Bigg) v(\xi) d\xi\\
								& = <Ku, v>
					\end{aligned}
				\end{equation}
				Hence, $K$ is self-adjoint.
			\end{proof}
		\item 
			\begin{equation}
				L_T = \inf\limits_{u\in M} \frac{<u, Ku>}{<u, u>} = \inf\limits_{u\in M} \frac{<Ku, u>}{<u, u>}
			\end{equation}
			and
			\begin{equation}
				U_T = \sup\limits_{u\in M} \frac{<u, Ku>}{<u, u>} = \sup\limits_{u\in M} \frac{<Ku, u>}{<u, u>}
			\end{equation}
			
	\end{enumerate}

\end{document}