\documentclass[paper=a4, fontsize=11pt]{scrartcl} % A4 paper and 11pt font size

\usepackage[T1]{fontenc} % Use 8-bit encoding that has 256 glyphs
\usepackage{fourier} % Use the Adobe Utopia font for the document - comment this line to return to the LaTeX default
\usepackage[english]{babel} % English language/hyphenation
\usepackage{amsmath,amsfonts,amsthm,amssymb} % Math packages

\usepackage{algorithm, algorithmic}
\renewcommand{\algorithmicrequire}{\textbf{Input:}} %Use Input in the format of Algorithm  
\renewcommand{\algorithmicensure}{\textbf{Output:}} %UseOutput in the format of Algorithm  

\usepackage{graphicx}

\usepackage{listings}
\lstset{language=Matlab}

\usepackage{lipsum} % Used for inserting dummy 'Lorem ipsum' text into the template

\usepackage{sectsty} % Allows customizing section commands
\allsectionsfont{\centering \normalfont\scshape} % Make all sections centered, the default font and small caps

\usepackage{fancyhdr} % Custom headers and footers
\pagestyle{fancyplain} % Makes all pages in the document conform to the custom headers and footers
\fancyhead{} % No page header - if you want one, create it in the same way as the footers below
\fancyfoot[L]{} % Empty left footer
\fancyfoot[C]{} % Empty center footer
\fancyfoot[R]{\thepage} % Page numbering for right footer
\renewcommand{\headrulewidth}{0pt} % Remove header underlines
\renewcommand{\footrulewidth}{0pt} % Remove footer underlines
\setlength{\headheight}{13.6pt} % Customize the height of the header

\numberwithin{equation}{section} % Number equations within sections (i.e. 1.1, 1.2, 2.1, 2.2 instead of 1, 2, 3, 4)
\numberwithin{figure}{section} % Number figures within sections (i.e. 1.1, 1.2, 2.1, 2.2 instead of 1, 2, 3, 4)
\numberwithin{table}{section} % Number tables within sections (i.e. 1.1, 1.2, 2.1, 2.2 instead of 1, 2, 3, 4)

\setlength\parindent{0pt} % Removes all indentation from paragraphs - comment this line for an assignment with lots of text

%----------------------------------------------------------------------------------------
%	TITLE SECTION
%----------------------------------------------------------------------------------------

\newcommand{\horrule}[1]{\rule{\linewidth}{#1}} % Create horizontal rule command with 1 argument of height

\title{	
\normalfont \normalsize 
\textsc{Shanghai Jiao Tong University, UM-SJTU JOINT INSTITUTE} \\ [25pt] % Your university, school and/or department name(s)
\horrule{0.5pt} \\[0.4cm] % Thin top horizontal rule
\huge Methods of Applied Mathematics I\\ HW6 \\ % The assignment title
\horrule{2pt} \\[0.5cm] % Thick bottom horizontal rule
}

\author{Yu Cang \quad 018370210001\\ Zhiming Cui \quad 017370910006} % Your name

\date{\normalsize \today} % Today's date or a custom date

\begin{document}

\maketitle % Print the title

\section{Exercise6.1}
	\begin{enumerate}
		\item
			\begin{proof}
				With the Cauchy-Schwartz inequality
				\begin{equation}
					\sum_{n=1}^{\infty} |y_n|^2 = \sum_{n=1}^{\infty} (\sum_{m=1}^{\infty} a_{nm}x_m)^2 \leq \sum_{n=1}^{\infty} \Big[\sum_{m=1}^{\infty}x_m^2 \sum_{m=1}^{\infty} a_{nm}^2\Big] = \Big(\sum_{m=1}^{\infty}x_m^2\Big) \Big(\sum_{n=1}^{\infty} \sum_{m=1}^{\infty} a_{nm}^2\Big) < \infty
				\end{equation}
				The last setp is valid as $x$ is a square-summable sequence and $\sum_{m,n=1}^{\infty} a_{nm}^2 = M^2 < \infty$.\\
				Hence $y\in l^2$.
			\end{proof}
		\item 
			According to the definition of $L$
			\begin{equation}
				L e_j = (a_{1j}, a_{2j}, ... , a_{nj})
			\end{equation}
			Thus
			\begin{equation}
				L_{ij} = <e_i, L e_j> = a_{ij}
			\end{equation}
		\item 
			\begin{proof}
				Follow from the first part
				\begin{equation}
					||y||_2 \leq ||x||_2 \cdot M
				\end{equation}
				Thus
				\begin{equation}
					||L|| = \sup_{x \in l^2} \frac{||Lx||_2}{||x||_2} \leq \frac{||x||_2 M}{||x||_2} = M
				\end{equation}
			\end{proof}
		\item 
			\begin{proof}
				Denote $(e_j)$ being the standard basis, then
				\begin{equation}
					L e_j = (0, 0, ..., \frac{1}{j}, 0, ... , 0)
				\end{equation}
				Thus
				\begin{equation}
					a_{ij} \triangleq L_{ij} = \delta_{ij}\frac{1}{j}
				\end{equation}
				Hence
				\begin{equation}
					\sum_{n, m = 1}^{\infty} |a_{nm}|^2 = \sum_{n=1}^{\infty} \frac{1}{n^2} = \frac{\pi^2}{6}
				\end{equation}
				which indicates that $L$ is a Hilbert-Schmidt operator.\\
				Let $x_0 = (1, 0, 0, ... , 0)$, then
				\begin{equation}
					||L|| = \sup_{x \in l^2} \frac{||Lx||_2}{||x||_2} \geq \frac{||Lx_0||_2}{||x_0||_2} = \frac{||(1, 0, ...)||_2}{||(1, 0, ...)||_2} = 1
				\end{equation}
				Also, with the Cauchy-Schwartz inequality
				\begin{equation}
					||L|| = \sup_{x \in l^2} \frac{||Lx||_2}{||x||_2} = \sup_{x \in l^2} \frac{\sqrt{\sum_{n=1}^{\infty} (\frac{x_n}{n})^2}}{||x||_2} \leq \sup_{x \in l^2} \frac{\sqrt{\sum_{n=1}^{\infty} x_n^2}}{||x||_2} = \sup_{x \in l^2} \frac{||x||_2}{||x||_2} = 1
				\end{equation}
				Therefore $||L|| = 1$.\\
				It can be seen that $M$ is the upper bound of $||L||$.
			\end{proof}
		
	\end{enumerate}

\end{document}