\documentclass[paper=a4, fontsize=11pt]{scrartcl} % A4 paper and 11pt font size

\usepackage[T1]{fontenc} % Use 8-bit encoding that has 256 glyphs
\usepackage{fourier} % Use the Adobe Utopia font for the document - comment this line to return to the LaTeX default
\usepackage[english]{babel} % English language/hyphenation
\usepackage{amsmath,amsfonts,amsthm,amssymb} % Math packages

\usepackage{algorithm, algorithmic}
\renewcommand{\algorithmicrequire}{\textbf{Input:}} %Use Input in the format of Algorithm  
\renewcommand{\algorithmicensure}{\textbf{Output:}} %UseOutput in the format of Algorithm  

\usepackage{listings}
\lstset{language=Matlab}

\usepackage{lipsum} % Used for inserting dummy 'Lorem ipsum' text into the template

\usepackage{sectsty} % Allows customizing section commands
\allsectionsfont{\centering \normalfont\scshape} % Make all sections centered, the default font and small caps

\usepackage{fancyhdr} % Custom headers and footers
\pagestyle{fancyplain} % Makes all pages in the document conform to the custom headers and footers
\fancyhead{} % No page header - if you want one, create it in the same way as the footers below
\fancyfoot[L]{} % Empty left footer
\fancyfoot[C]{} % Empty center footer
\fancyfoot[R]{\thepage} % Page numbering for right footer
\renewcommand{\headrulewidth}{0pt} % Remove header underlines
\renewcommand{\footrulewidth}{0pt} % Remove footer underlines
\setlength{\headheight}{13.6pt} % Customize the height of the header

\numberwithin{equation}{section} % Number equations within sections (i.e. 1.1, 1.2, 2.1, 2.2 instead of 1, 2, 3, 4)
\numberwithin{figure}{section} % Number figures within sections (i.e. 1.1, 1.2, 2.1, 2.2 instead of 1, 2, 3, 4)
\numberwithin{table}{section} % Number tables within sections (i.e. 1.1, 1.2, 2.1, 2.2 instead of 1, 2, 3, 4)

\setlength\parindent{0pt} % Removes all indentation from paragraphs - comment this line for an assignment with lots of text

%----------------------------------------------------------------------------------------
%	TITLE SECTION
%----------------------------------------------------------------------------------------

\newcommand{\horrule}[1]{\rule{\linewidth}{#1}} % Create horizontal rule command with 1 argument of height

\title{	
\normalfont \normalsize 
\textsc{Shanghai Jiao Tong University, UM-SJTU JOINT INSTITUTE} \\ [25pt] % Your university, school and/or department name(s)
\horrule{0.5pt} \\[0.4cm] % Thin top horizontal rule
\huge Methods of Applied Mathematics I\\ HW2 \\ % The assignment title
\horrule{2pt} \\[0.5cm] % Thick bottom horizontal rule
}

\author{Yu Cang \\ 018370210001} % Your name

\date{\normalsize \today} % Today's date or a custom date

\begin{document}

\maketitle % Print the title

\section{Exercise2.1}
	\begin{enumerate}
		\item 
			\begin{proof}
				\begin{equation}
					\begin{aligned}
						RHS & = \frac{1}{4} (||x+y||^2  - ||x-y||^2)\\
						    & = \frac{1}{4} (<x+y, x+y> - <x-y, x-y>)\\
						    & = \frac{1}{4} (<x+y, x> + <x+y, y> - <x-y, x> - <x-y, -y>)\\
						    & = \frac{1}{4} (\overline{<x, x+y>} + \overline{<y, x+y>} - \overline{<x, x-y>} + \overline{<y, x-y>})\\
						    & = \frac{1}{4} (\overline{<x, x>} + \overline{<x, y>} + \overline{<y, x>} + \overline{<y, y>} - \overline{<x, x>} + \overline{<x, y>} + \overline{<y, x>} - \overline{<y, y>})\\
						    & = \frac{1}{2}(\overline{<x,y>} + \overline{<y, x>})\\
						    & = \frac{1}{2}(<x, y> + <y, x>)\\
						    & = <x, y> = LHS
					\end{aligned}
				\end{equation}
				The last line is valid as the inner-product is defined on real space s.t. $<x, y>=<y, x>$.
			\end{proof}
		\item 
			\begin{proof}
				As have been proved aboved
				\begin{equation}
					\begin{aligned}
						   & \frac{1}{4} (||x+y||^2  - ||x-y||^2)\\
						=  & \frac{1}{2}(<x, y> + <y, x>)
					\end{aligned}
				\end{equation}
				Also, replace $y$ with $iy$ yields
				\begin{equation}
					\begin{aligned}
						  & \frac{i}{4} (||x+iy||^2  - ||x-iy||^2)\\
						= & \frac{i}{2} (<x, iy> + <iy, x>)\\
						= & \frac{i}{2} (i<x, y> + \overline{i}<y, x>)\\
						= & \frac{1}{2} (-<x, y> + <y,x>)
					\end{aligned}
				\end{equation}
				Thus
				\begin{equation}
					\begin{aligned}
						RHS & = \frac{1}{4} (||x+y||^2  - ||x-y||^2) - \frac{i}{4} (||x+iy||^2  - ||x-iy||^2)\\
						    & = \frac{1}{2}(<x, y> + <y, x>) - \frac{1}{2} (-<x, y> + <y,x>)\\
						    & = <x, y> = LHS
					\end{aligned}
				\end{equation}
			\end{proof}
		\item 
			\begin{proof}
				\begin{equation}
					\begin{aligned}
						LHS & = ||x+y||^2 + ||x-y||^2\\
						    & = <x+y, x+y> + <x-y, x-y>\\
						    & = <x+y, x> + <x+y, y> + <x-y, x> - <x-y, y>\\
						    & = \overline{<x, x+y>} + \overline{<y, x+y>} + \overline{<x, x-y>} - \overline{<y, x-y>}\\
						    & = \overline{<x, x>} + \overline{<x, y>} + \overline{<y, x>} + \overline{<y, y>} + \overline{<x, x>} - \overline{<x, y>} - \overline{<y, x>} + \overline{<y, y>}\\
						    & = 2(<x, x>+<y, y>)\\
						    & = 2(||x||^2 + ||y||^2) = RHS
					\end{aligned}
				\end{equation}
			\end{proof}	
		\item 
			\begin{proof}
				As any norm induced by inner product should satisfy the parallelogram law, it's easy to show that the norm $||.||_\infty : f\mapsto \sup_{x\in [a, b]} |f(x)|$ on $C[a, b]$ can not be induced by any inner product since it does not satisfy the parallelogram law. For example, let
				\begin{equation}
					f(x) = \frac{x-a}{b-a}
				\end{equation}
				and
				\begin{equation}
					g(x) = (\frac{x-a}{b-a})^2
				\end{equation}
				then
				\begin{equation}
					||f+g|| = \max_{x\in[a, b]} |(f+g)(x)| = |(f+g)(b)| = 2
				\end{equation}
				\begin{equation}
					||f-g|| = \max_{x\in[a, b]} |(f-g)(x)| = |(f-g)(\frac{a+b}{2})| = \frac{1}{4}
				\end{equation}
				\begin{equation}
					||f|| = \max_{x\in[a, b]} |f(x)| = |f(b)| = 1
				\end{equation}
				\begin{equation}
					||g|| = \max_{x\in[a, b]} |g(x)| = |g(b)| = 1
				\end{equation}
				It's obvious that
				\begin{equation}
					||f+g||^2 + ||f-g||^2 = \frac{65}{4} \neq 2(||f||^2 + ||g||^2) = 4
				\end{equation}
			\end{proof}
		
		\item 
			Assume the norm satisfies the parallelogram law, that is
			\begin{equation}
				||x+y||^2 + ||x-y||^2 = 2(||x||^2 + ||y||^2)
			\end{equation}
			Let
			\begin{equation}
				<x, y> \triangleq \frac{1}{4}(||x+y||^2 - ||x-y||^2)
			\end{equation}
			it's left to prove that $<x, y>$ is an inner product.\\
			Firstly, $<x, y+z>$ can be expanded according to the definition as follows
			\begin{equation}
				\begin{aligned}
					<x, y+z> & = \frac{1}{4}(||x+y+z||^2 - ||x-(y+z)||^2)\\
					         & = \frac{1}{4}(||y+z+x||^2 - ||y+z-x||^2)
				\end{aligned}
			\end{equation}
			Since
			\begin{equation}
				||y+z+x||^2 + ||y-z+x||^2 = 2(||y+x||^2 + ||z||^2)
			\end{equation}
			Then
			\begin{equation}
				||y+z+x||^2 = 2(||y+x||^2 + ||z||^2) - ||y-z+x||^2
			\end{equation}
			Swap $y$ and $z$ yields
			\begin{equation}
				||x+y+z||^2 = 2(||z+x||^2 + ||y||^2) - ||z-y+x||^2
			\end{equation}
			Adding the two equations above and divided by 2 yields
			\begin{equation}
				||y+z+x||^2 = ||y||^2 + ||z||^2 + ||x+y||^2 + ||x+z||^2 -\frac{1}{2}(||y-z+x||^2 + ||z-y+x||^2)
			\end{equation}
			Replace $x$ with $-x$
			\begin{equation}
				||y+z-x||^2 = ||y||^2 + ||z||^2 + ||x-y||^2 + ||x-z||^2 -\frac{1}{2}(||y-z-x||^2 + ||z-y-x||^2)
			\end{equation}
			Hence
			\begin{equation}
				||y+z+x||^2 + ||y+z-x||^2 = (||x+y||^2 - ||x-y||^2) + (||x+z||^2 - ||x-z||^2)
			\end{equation}
			As
			\begin{equation}
				<x, y> = \frac{1}{4} (||x+y||^2 - ||x-y||^2)
			\end{equation}
			and
			\begin{equation}
				<x, z> = \frac{1}{4} (||x+z||^2 - ||x-z||^2)
			\end{equation}
			It's therefore can be concluded that
			\begin{equation}
				<x, y+z> = <x, y> + <x, z>
			\end{equation}
			Secondly, 
		
	\end{enumerate}

\section{Exercise2.2}
	Denote $c_0 = \{(a_n)_{n\in\mathbb{N}}: \lim\limits_{n \rightarrow \infty} a_n = 0\}$ and $l^1 = \{(b_n)_{n\in\mathbb{N}}: \sum_{n=0}^{\infty} |b_n| \le \infty\}$
	\begin{enumerate}
		\item
		YES
		\begin{proof}
			For any $(a_n) \in c_0$ and $\epsilon > 0$, $\exists N_1$ s.t. when $n > N_1$, $|a_n| \le \frac{\epsilon}{2}$.\\
			Also, a sequence $(b_n) \in l^1$ can be found s.t. $|b_n| \le \frac{\epsilon}{2}$ when $n > N_2$, and $b_n = a_n$ for $n \leq N_3 \triangleq \max(N_1, N_2)$. This does not violate requirement for $(b_n)$ as the sum of $|b_n|$ here is finite.\\
			Thus, $||(a_n) - (b_n)||_\infty = \sup_{n \in \mathbb{N}} |a_n - b_n| \leq |a_n| + |b_n| \leq \frac{\epsilon}{2} + \frac{\epsilon}{2} = \epsilon$ when $n > N_3$. And it implies that $l^1$ is dense in $c_0$ under the $||.||_\infty$ norm.
		\end{proof}
		\item 
		NO
		\begin{proof}
			For example, pick a sequence $(a_n) \in c_0$ s.t. $a_n = \frac{1}{n+1}$, for any sequence $(b_n) \in l^1$. The norm $||(a_n) - (b_n)|| = \sum_{n=0}^{\infty} |a_n - b_n| \geq \sum_{n=0}^{\infty} (|a_n| - |b_n|) = (\sum_{n=0}^{\infty}|a_n|) - (\sum_{n=0}^{\infty}|b_n|)$ diverges as $\sum_{n=0}^{\infty}|a_n|$ goes infinite and $\sum_{n=0}^{\infty}|b_n|$ is finite.
		\end{proof}
	
	\end{enumerate}

\end{document}