\documentclass[paper=a4, fontsize=11pt]{scrartcl} % A4 paper and 11pt font size

\usepackage[T1]{fontenc} % Use 8-bit encoding that has 256 glyphs
\usepackage{fourier} % Use the Adobe Utopia font for the document - comment this line to return to the LaTeX default
\usepackage[english]{babel} % English language/hyphenation
\usepackage{amsmath,amsfonts,amsthm,amssymb} % Math packages

\usepackage{algorithm, algorithmic}
\renewcommand{\algorithmicrequire}{\textbf{Input:}} %Use Input in the format of Algorithm  
\renewcommand{\algorithmicensure}{\textbf{Output:}} %UseOutput in the format of Algorithm  

\usepackage{graphicx}

\usepackage{listings}
\lstset{language=Matlab}

\usepackage{lipsum} % Used for inserting dummy 'Lorem ipsum' text into the template

\usepackage{sectsty} % Allows customizing section commands
\allsectionsfont{\centering \normalfont\scshape} % Make all sections centered, the default font and small caps

\usepackage{fancyhdr} % Custom headers and footers
\pagestyle{fancyplain} % Makes all pages in the document conform to the custom headers and footers
\fancyhead{} % No page header - if you want one, create it in the same way as the footers below
\fancyfoot[L]{} % Empty left footer
\fancyfoot[C]{} % Empty center footer
\fancyfoot[R]{\thepage} % Page numbering for right footer
\renewcommand{\headrulewidth}{0pt} % Remove header underlines
\renewcommand{\footrulewidth}{0pt} % Remove footer underlines
\setlength{\headheight}{13.6pt} % Customize the height of the header

\numberwithin{equation}{section} % Number equations within sections (i.e. 1.1, 1.2, 2.1, 2.2 instead of 1, 2, 3, 4)
\numberwithin{figure}{section} % Number figures within sections (i.e. 1.1, 1.2, 2.1, 2.2 instead of 1, 2, 3, 4)
\numberwithin{table}{section} % Number tables within sections (i.e. 1.1, 1.2, 2.1, 2.2 instead of 1, 2, 3, 4)

\setlength\parindent{0pt} % Removes all indentation from paragraphs - comment this line for an assignment with lots of text

%----------------------------------------------------------------------------------------
%	TITLE SECTION
%----------------------------------------------------------------------------------------

\newcommand{\horrule}[1]{\rule{\linewidth}{#1}} % Create horizontal rule command with 1 argument of height

\title{	
\normalfont \normalsize 
\textsc{Shanghai Jiao Tong University, UM-SJTU JOINT INSTITUTE} \\ [25pt] % Your university, school and/or department name(s)
\horrule{0.5pt} \\[0.4cm] % Thin top horizontal rule
\huge Methods of Applied Mathematics I\\ HW7 \\ % The assignment title
\horrule{2pt} \\[0.5cm] % Thick bottom horizontal rule
}

\author{Yu Cang \quad 018370210001\\ Zhiming Cui \quad 017370910006} % Your name

\date{\normalsize \today} % Today's date or a custom date

\begin{document}

\maketitle % Print the title

\section{Exercise7.1}
	\begin{enumerate}
		\item 
			\begin{proof}
				Since $|\lambda|=1$, then
				\begin{equation}
					||e_\lambda^{(N)}||_2 = \frac{1}{\sqrt{N+1}} \sqrt{\sum_{i=0}^{N} \lambda^{2i}} =  \frac{1}{\sqrt{N+1}} \sqrt{N+1} = 1
				\end{equation}
			\end{proof}
		\item 
			
		\item 
			Not figured out...
		\item
			\begin{equation}
				Ru = \lambda u
			\end{equation} 
			Denote $u = (a_0, a_1, ..., a_n, ...)$, then
			\begin{equation}
				\begin{aligned}
					0 &= \lambda a_0\\
					a_0 &= \lambda a_1\\
					...\\
				\end{aligned}
			\end{equation}
	\end{enumerate}


\section{Exercise7.2}
	\begin{enumerate}
		\item 
			\begin{proof}
				Take a subset of the domain of $L^{-1}$, which is denoted as $M$ and is defined as
				\begin{equation}
					M = \Big\{u \in L^2\big([0, 1]\big) \Big| \exists \xi >0 \  \forall x \in [0, \xi] \ u(x) = 0\Big\}
				\end{equation}
				Given any $u(x) \in L^2$, then
				\begin{equation}
					\int_{0}^{1} u^2(x) dx < \infty
				\end{equation}
				Hence $u(x)$ is bounded over $[0, 1]$. \\
				Let 
				\begin{equation}
					T \triangleq \sup \limits_{x\in [0, 1]} |u(x)|
				\end{equation}
				then, $\forall \epsilon > 0$, $\exists v \in M $ s.t.
				\begin{equation}
					v(x) = \Bigg\{\begin{aligned}
					0, &\quad x \in [0, \delta]\\
					u(x), &\quad x \in (\delta, 1]
					\end{aligned}
				\end{equation}
				where $\delta = \frac{\epsilon}{T^2}$. Therefore
				\begin{equation}
					d(u, v) \triangleq ||u-v||_2 = \int_{0}^{1} [u(x) - v(x)]^2 dx = \int_{0}^{\delta} u^2(x) dx \leq T^2 \delta = \epsilon
				\end{equation}
				which indicates that $M$ is dense in $L^2$, so do the domain of $L^{-1}$.
			\end{proof}
		\item 
			\begin{proof}
				\begin{equation}
					||L|| = \sup_{u\in L^2}\frac{||Lu||_2}{||u||_2} = \sup_{u\in L^2}\frac{||xu(x)||_2}{||u(x)||_2} = \sup_{u\in L^2}\frac{|x| ||u||_2}{||u||_2} = \sup_{x\in [0, 1]} |x| = 1
				\end{equation}
				and
				\begin{equation}
					||L^{-1}|| = \sup_{u\in L^2}\frac{||L^{-1}u||_2}{||u||_2} = \sup_{x\in [0, 1]} \frac{1}{|x|} = \infty
				\end{equation}
				hence $L^{-1}$ is unbounded.
			\end{proof}
		\item 
			The state of $L$ is $(I, 1_n)$.\\
			The state of $L^{-1}$ is $(I, 2_c)$.
		\item 
			Yes
		\item 
			Since
			\begin{equation}
				[(L - \lambda I)u] (x) = (Lu)(x) - \lambda u(x) = (x-\lambda) u(x)
			\end{equation}
			The inverse of $(L - \lambda I)$ always exists, therefore $\sigma(L) = \emptyset$.
	\end{enumerate}


\section{Exercise7.3}
	\begin{enumerate}
		\item 
			\begin{proof}
				It's clear that $L^{-1}$ is the differentiate operator, and $L^{-1}$ is unbounded. So $L$ has unbounded inverse.\\
				Since the domain of $L$ is composed of square-integrable functions over $[0,1]$, say
				\begin{equation}
					\int_{0}^{1} f^2(x) dx < \infty
				\end{equation}
				An element within the range of $L$ is
				\begin{equation}
					g(x) = \int_{0}^{x} f(t) dt
				\end{equation}
				Then, if $f(x)$ is a polynomial in $L^2$ , it must be bounded over $[0, 1]$ as it is continous. Denote the supreme of $f(x)$ as $M$, then $g(x) \leq Mx$. Hence $g(x)$ is square-integrable over $[0,1]$, say
				\begin{equation}
					\int_{0}^{1} g^2(x) dx \leq M^2 \int_{0}^{1} x^2 dx < \infty
				\end{equation}
				Clearly, the domain of $L$ doesn't contain all the polynomials and therefore the range of $L$ is open and incomplete. The boundary of the range of $L$ are the limits of squences like $f_n(x) = nx$ when $n\rightarrow \infty$.\\
				Hence, the state of $L$ is $(III, 1_n)$.
			\end{proof}
		\item 
			\begin{equation}
				L^* = L
			\end{equation}
		
	\end{enumerate}

\section{Exercise7.4}
	\begin{proof}
		For $p = 1$
		\begin{equation}
			\begin{aligned}
				RHS & \triangleq ||(a_n)||_1 \cdot ||(b_n)||_1 = \sum_{i=0}^{\infty} |a_i| \cdot \sum_{j=0}^{\infty} |b_j| = \sum_{n=0}^{\infty} \sum_{i+j=n} |a_i||b_j|\\
				 & \geq \sum_{n=0}^{\infty} |\sum_{i+j=n} a_i b_j| = \sum_{n=0}^{\infty} |c_n| = ||(c_n)||_1 \triangleq LHS
			\end{aligned}
		\end{equation} 
		For $p>1$, take $q > 0$ s.t. $\frac{1}{p} + \frac{1}{q} = 1$\\
		Then, using the Holder's inequality
		\begin{equation}
			\begin{aligned}
				LHS & \triangleq ||(c_n)||_p = \Big(\sum_{n=0}^{\infty} |c_n|^p\Big)^{\frac{1}{p}} = \Big(\sum_{n=0}^{\infty} \Big|\sum_{i+j=n}a_i b_j\Big|^p\Big)^{\frac{1}{p}}\\
				& = \Big(\sum_{n=0}^{\infty} \Big|\sum_{i+j=n}a_i b_j^\frac{1}{p} b_j^{\frac{1}{q}}\Big|^p\Big)^{\frac{1}{p}} \leq  \Big(\sum_{n=0}^{\infty} \Big|\sum_{i+j=n}\Big(|a_i| |b_j|^\frac{1}{p}\Big) |b_j|^{\frac{1}{q}}\Big|^p\Big)^{\frac{1}{p}}\\
				& \leq \Big(\sum_{n=0}^{\infty} \Big[\Big(\sum_{i+j=n}|a_i|^p |b_j|\Big)^\frac{1}{p} \Big(\sum_{j=0}^{n}|b_j|\Big)^\frac{1}{q}\Big]^p\Big)^{\frac{1}{p}} \\ 
				& = \Bigg[\sum_{n=0}^{\infty} \Big(\sum_{i+j=n}|a_i|^p |b_j|\Big) \Big(\sum_{j=0}^{n}|b_j|\Big)^\frac{p}{q}\Bigg]^\frac{1}{p} \\
				& \leq \Big(\sum_{i=0}^{\infty}|a_i|^p\Big)^\frac{1}{p}\cdot\Big(\sum_{j=0}^{\infty}|b_j|\Big)^\frac{1}{p}\cdot\Big(\sum_{j=0}^{\infty}|b_j|\Big)^\frac{1}{q}\\
				& = ||(a_n)||_p \cdot ||(b_n)||_1 \triangleq RHS
			\end{aligned}
		\end{equation}
	\end{proof}

\end{document}