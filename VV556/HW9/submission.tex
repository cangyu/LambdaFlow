\documentclass[paper=a4, fontsize=11pt]{scrartcl} % A4 paper and 11pt font size

\usepackage[T1]{fontenc} % Use 8-bit encoding that has 256 glyphs
\usepackage{fourier} % Use the Adobe Utopia font for the document - comment this line to return to the LaTeX default
\usepackage[english]{babel} % English language/hyphenation
\usepackage{amsmath,amsfonts,amsthm,amssymb} % Math packages

\usepackage{algorithm, algorithmic}
\renewcommand{\algorithmicrequire}{\textbf{Input:}} %Use Input in the format of Algorithm  
\renewcommand{\algorithmicensure}{\textbf{Output:}} %UseOutput in the format of Algorithm  

\usepackage{graphicx}

\usepackage{listings}
\lstset{language=Matlab}

\usepackage{lipsum} % Used for inserting dummy 'Lorem ipsum' text into the template

\usepackage{sectsty} % Allows customizing section commands
\allsectionsfont{\centering \normalfont\scshape} % Make all sections centered, the default font and small caps

\usepackage{fancyhdr} % Custom headers and footers
\pagestyle{fancyplain} % Makes all pages in the document conform to the custom headers and footers
\fancyhead{} % No page header - if you want one, create it in the same way as the footers below
\fancyfoot[L]{} % Empty left footer
\fancyfoot[C]{} % Empty center footer
\fancyfoot[R]{\thepage} % Page numbering for right footer
\renewcommand{\headrulewidth}{0pt} % Remove header underlines
\renewcommand{\footrulewidth}{0pt} % Remove footer underlines
\setlength{\headheight}{13.6pt} % Customize the height of the header

\numberwithin{equation}{section} % Number equations within sections (i.e. 1.1, 1.2, 2.1, 2.2 instead of 1, 2, 3, 4)
\numberwithin{figure}{section} % Number figures within sections (i.e. 1.1, 1.2, 2.1, 2.2 instead of 1, 2, 3, 4)
\numberwithin{table}{section} % Number tables within sections (i.e. 1.1, 1.2, 2.1, 2.2 instead of 1, 2, 3, 4)

\setlength\parindent{0pt} % Removes all indentation from paragraphs - comment this line for an assignment with lots of text

%----------------------------------------------------------------------------------------
%	TITLE SECTION
%----------------------------------------------------------------------------------------

\newcommand{\horrule}[1]{\rule{\linewidth}{#1}} % Create horizontal rule command with 1 argument of height

\title{	
\normalfont \normalsize 
\textsc{Shanghai Jiao Tong University, UM-SJTU JOINT INSTITUTE} \\ [25pt] % Your university, school and/or department name(s)
\horrule{0.5pt} \\[0.4cm] % Thin top horizontal rule
\huge Methods of Applied Mathematics I\\ HW9 \\ % The assignment title
\horrule{2pt} \\[0.5cm] % Thick bottom horizontal rule
}

\author{Yu Cang \quad 018370210001\\ Zhiming Cui \quad 017370910006} % Your name

\date{\normalsize \today} % Today's date or a custom date

\begin{document}

\maketitle % Print the title

\section{Exercise9.1}
	\begin{enumerate}
		\item
			\begin{proof}
				Since
				\begin{equation}
					\begin{aligned}
						(LKu)(x) & = -\frac{d^2}{d x^2} \int_{0}^{1} g(x, \xi) u(\xi) d\xi\\
								 & = -\frac{d^2}{d x^2} \Bigg(\int_{0}^{x} g(x, \xi) u(\xi) d\xi + \int_{x}^{1} g(x, \xi) u(\xi) d\xi \Bigg)\\
								 & = -\frac{d^2}{d x^2} \Bigg(\int_{0}^{x} \xi(1-x) u(\xi) d\xi + \int_{x}^{1} x(1-\xi) u(\xi) d\xi \Bigg)\\
								 & = -\frac{d^2}{d x^2} \Bigg((1-x)\int_{0}^{x} \xi u(\xi) d\xi + x\int_{x}^{1} (1-\xi) u(\xi) d\xi \Bigg)\\
								 & = -\frac{d}{d x} \Bigg(-\int_{0}^{x} \xi u(\xi) d\xi + \int_{x}^{1} (1-\xi) u(\xi) d\xi \Bigg)\\
								 & = xu(x) + (1-x) u(x) = u(x)
 					\end{aligned}
				\end{equation}
				Thus, $LK=I$, which means $L = K^{-1}$.
			\end{proof}
		\item 
			\begin{proof}
				($\Rightarrow$) Suppose $\frac{1}{\lambda}$ is an eigenvalue for $K$, then
				\begin{equation}
					ku = \frac{1}{\lambda}u
				\end{equation} 
				Thus
				\begin{equation}
					u = \lambda K u
				\end{equation}
				Take the operator $L$ on it
				\begin{equation}
					Lu = \lambda LKu = \lambda Iu = \lambda u
				\end{equation}
				which indicates that $\lambda$ is an eigenvalue of $L$.\\
				($\Leftarrow$) Suppose $\lambda \neq 0$ is an eigenvalue of $L$, then
				\begin{equation}
					Lu = \lambda u
				\end{equation}
				Thus
				\begin{equation}
					u = \frac{1}{\lambda} Lu
				\end{equation}
				As $K = L^{-1}$ is known, take the operator $K$ onto it yields
				\begin{equation}
					Ku = \frac{1}{\lambda}KLu = \frac{1}{\lambda} Iu = \frac{1}{\lambda} u
				\end{equation}
				which indicates that $\frac{1}{\lambda}$ is an eigenvalue of $K$.\\
				It's clear that for the given $\lambda$, $K$ and $L$ share the same eigenfunction $u(x)$.
			\end{proof}
		\item 
			\begin{proof}
				It's obvious that $L$ is bounded over $M$.\\
				Since
				\begin{equation}
					\begin{aligned}
						<u, Lv> &= -\int_{0}^{1} u(x)v''(x) dx = -\Bigg(u(x)v'(x)\Big|_0^1 - \int_{0}^{1}u'(x)v'(x)dx\Bigg)\\
								&= \int_{0}^{1}u'(x)v'(x)dx = \int_{0}^{1}u'(x)dv(x) = u'(x)v(x)\Big|_0^1 - \int_{0}^{1} v(x)u''(x)dx\\
								& = -\int_{0}^{1} u''(x)v(x)dx = <Lu, v>
					\end{aligned}
				\end{equation}
				which indicates that $L$ is self-adjoint. Thus the eigenvalues of $L$ are real numbers.\\
		 		The eigenfunction $\psi_n(x)$ of $L$ together with $\lambda_n$ satisfy 
		 		\begin{equation}
		 			\psi_n''(x) + \lambda_n \psi_n(x) = 0
		 		\end{equation}
		 		and
		 		\begin{equation}
		 			\psi_n(0) = \psi_n(1) = 0
		 		\end{equation}
		 		Solution of this ODE depends on the sign of $\lambda_n$.\\
		 		\begin{enumerate}
		 			\item 
		 				If $\lambda_n > 0$, then
		 				\begin{equation}
		 					\psi_n(x) = C_1 sin(\sqrt{\lambda_n}x) + C_2 cos(\sqrt{\lambda_n}x)
		 				\end{equation}
		 				apply the boundary condition yields
		 				\begin{equation}
		 					C_2 = 0
		 				\end{equation}
		 				and
		 				\begin{equation}
		 					C_1 sin(\sqrt{\lambda_n}) = 0
		 				\end{equation}
		 				Thus
		 				\begin{equation}
		 					\lambda_n = (n\pi)^2
		 				\end{equation}
		 				and
		 				\begin{equation}
		 					\psi_n(x) = C_2 sin(n\pi x) \quad (C_2 \neq 0)
		 				\end{equation}
		 			\item 
		 				If $\lambda_n = 0$, then
		 				\begin{equation}
		 					\psi_n(x) = C_0 + C_1 x
		 				\end{equation}
		 				Apply the B.C. yields $C_0 = C_1 = 0$, which indicates 0 is not an eigenvalue for $L$.
		 			\item 
		 				If $\lambda_n < 0$, then
		 				\begin{equation}
		 					\psi_n(x) = C_1 e^{\sqrt{-\lambda_n}x} + C_2 e^{-\sqrt{-\lambda_n}x}
		 				\end{equation}
		 				Apply the B.C. yields $C_0 = C_1 = 0$, which indicates  $\lambda_n < 0$ are not eigenvalues for $L$.
		 		\end{enumerate}
			\end{proof}
		\item 
			It's obvious that, from previous proof, the eigenvalues of $K$ are
			\begin{equation}
				\sigma_{point}(K) = \{\frac{1}{(n\pi)^2}, n=1,2,3, ....\}
			\end{equation} 
			As $K$ is compact and self-adjoint, it's obvious bounded and
			\begin{equation}
				\sigma_{compression}(K) = \sigma_{point}(K)
			\end{equation} 
			If $0$ is an eigenvalue of $K$, then
			\begin{equation}
				Ku = 0
			\end{equation}
			thus
			\begin{equation}
				LKu = 0 \quad \Rightarrow \quad u = 0
			\end{equation}
			which indicates that 0 is not an eigenvalue for $K$.\\
			Hence, 
			\begin{equation}
				\sigma_{continous}(K) = {0}
			\end{equation}
			as $K^{-1} = L$ is unbounded.
		\item 
			It's obvious that, from previous proof, the eigenfunctions of $K$ are
			\begin{equation}
				u_n(x) = sin(n\pi x) \quad n = 1, 2, ...
			\end{equation}
			which form the Fouier basis.
	\end{enumerate}

\end{document}