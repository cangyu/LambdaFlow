\documentclass[paper=a4, fontsize=11pt]{scrartcl} % A4 paper and 11pt font size

\usepackage[T1]{fontenc} % Use 8-bit encoding that has 256 glyphs
\usepackage{fourier} % Use the Adobe Utopia font for the document - comment this line to return to the LaTeX default
\usepackage[english]{babel} % English language/hyphenation
\usepackage{amsmath,amsfonts,amsthm,amssymb} % Math packages

\usepackage{algorithm, algorithmic}
\renewcommand{\algorithmicrequire}{\textbf{Input:}} %Use Input in the format of Algorithm  
\renewcommand{\algorithmicensure}{\textbf{Output:}} %UseOutput in the format of Algorithm  

\usepackage{graphicx}

\usepackage{listings}
\lstset{language=Matlab}

\usepackage{lipsum} % Used for inserting dummy 'Lorem ipsum' text into the template

\usepackage{sectsty} % Allows customizing section commands
\allsectionsfont{\centering \normalfont\scshape} % Make all sections centered, the default font and small caps

\usepackage{fancyhdr} % Custom headers and footers
\pagestyle{fancyplain} % Makes all pages in the document conform to the custom headers and footers
\fancyhead{} % No page header - if you want one, create it in the same way as the footers below
\fancyfoot[L]{} % Empty left footer
\fancyfoot[C]{} % Empty center footer
\fancyfoot[R]{\thepage} % Page numbering for right footer
\renewcommand{\headrulewidth}{0pt} % Remove header underlines
\renewcommand{\footrulewidth}{0pt} % Remove footer underlines
\setlength{\headheight}{13.6pt} % Customize the height of the header

\numberwithin{equation}{section} % Number equations within sections (i.e. 1.1, 1.2, 2.1, 2.2 instead of 1, 2, 3, 4)
\numberwithin{figure}{section} % Number figures within sections (i.e. 1.1, 1.2, 2.1, 2.2 instead of 1, 2, 3, 4)
\numberwithin{table}{section} % Number tables within sections (i.e. 1.1, 1.2, 2.1, 2.2 instead of 1, 2, 3, 4)

\setlength\parindent{0pt} % Removes all indentation from paragraphs - comment this line for an assignment with lots of text

%----------------------------------------------------------------------------------------
%	TITLE SECTION
%----------------------------------------------------------------------------------------

\newcommand{\horrule}[1]{\rule{\linewidth}{#1}} % Create horizontal rule command with 1 argument of height

\title{	
\normalfont \normalsize 
\textsc{Shanghai Jiao Tong University, UM-SJTU JOINT INSTITUTE} \\ [25pt] % Your university, school and/or department name(s)
\horrule{0.5pt} \\[0.4cm] % Thin top horizontal rule
\huge Methods of Applied Mathematics I\\ HW5 \\ % The assignment title
\horrule{2pt} \\[0.5cm] % Thick bottom horizontal rule
}

\author{Yu Cang \\ 018370210001} % Your name

\date{\normalsize \today} % Today's date or a custom date

\begin{document}

\maketitle % Print the title

\section{Exercise5.1}
	\begin{proof}
		Suppose $\dim(U) = N$ and $\dim(V)=M$. Since all the norm are equivalent in finite-dimension spaces, the linear operator $L$ can be represented in matrix form as $\{l_{ji}\}$. Then
		\begin{equation}
			\begin{aligned}
				||Lu|| & = \sqrt{\sum_{j=1}^{M}\Big|\sum_{i=1}^{N}l_{ji} u_i\Big|^2}\\
					   & \leq \sqrt{\sum_{j=1}^{M} \Big[\sum_{i=1}^{N} l_{ji}^2 \sum_{i=1}^{N} u_{i}^2\Big]} \quad \text{(Cauchy-Schwartz)}\\
					   & = \sqrt{\sum_{j=1}^{M} ||u||^2\sum_{i=1}^{N} l_{ji}^2}
					    = ||u||\sqrt{\sum_{j=1}^{M}\sum_{i=1}^{N} l_{ji}^2}
			\end{aligned}
		\end{equation}
		Hence, $L$ is bounded.
	\end{proof}

\section{Exercise5.2}
	\begin{proof}
		Denote $w = \alpha u + \beta v$, then
		\begin{equation}
			(Tw)(x) = xw(x) = x(\alpha u(x) + \beta v(x)) = \alpha xu(x) + \beta xv(x) = \alpha (Tu)(x) + \beta (Tv)(x)
		\end{equation}
		Hence, $T$ is linear.\\ Further
		\begin{equation}
			||T|| = \sup_{u\in\mathbb{C}[0, 1]}\frac{||Tu||}{||u||} = \sup_{u\in\mathbb{C}[0, 1]} \frac{\sup_{x\in[0, 1]}|xu(x)|}{\sup_{x\in[0, 1]}|u(x)|} = \sup_{u\in\mathbb{C}[0, 1]} \frac{\sup_{x\in[0, 1]}|u(x)|}{\sup_{x\in[0, 1]}|u(x)|} = 1
		\end{equation}
	\end{proof}

\section{Exercise5.3}
	\begin{enumerate}
		\item 
			\begin{proof}
				Firstly, it's obvious that $||u|| \geq 0$. Meanwhile, $||u|| \geq \sup_{x\in[a, b]} |u(x)| > 0$ when $u(x) \neq 0$. Hence $||u|| = 0$ iff. $u(x) = 0$.\\
				Secondly, the linearity is valid as
					\begin{equation}
						||\alpha \cdot u|| = \sup_{x\in[a, b]} |\alpha u(x)| + \sup_{x\in[a, b]} |\alpha u'(x)| = |\alpha| (\sup_{x\in[a, b]} |u(x)| + \sup_{x\in[a, b]} |u'(x)|) = |\alpha| ||u||
					\end{equation}
				Finally, the triangle inequality is justified as
					\begin{equation}
						\begin{aligned}
							||u+v|| & = \sup_{x\in[a, b]}|u(x)+v(x)| + \sup_{x\in[a, b]}|u'(x) + v'(x)|\\
							& \leq \sup_{x\in[a, b]} |u(x)| + \sup_{x\in[a, b]} |u'(x)| + \sup_{x\in[a, b]} |v(x)| + \sup_{x\in[a, b]} |v'(x)|\\
							& = ||u|| + ||v||
						\end{aligned}
					\end{equation}
				Thus, $||u||$ defines a norm.
			\end{proof}
		\item 
			 For example, let $u(x) = x^n$, then
			\begin{equation}
				||T|| = \sup_{u\in\mathbb{C^1}[0, 1]} \frac{||Tu||}{||u||} = \sup_{u\in\mathbb{C^1}[0, 1]} \frac{\sup_{x\in[0, 1]}|u'(x)|}{\sup_{x\in[0, 1]}|u(x)|} = n
			\end{equation}
			which indicates that $T$ is not bounded, hence $T$ is not continuous.
	\end{enumerate}

\end{document}