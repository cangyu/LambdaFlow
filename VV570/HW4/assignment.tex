%----------------------------------------------------------------------------------------
%	PACKAGES AND OTHER DOCUMENT CONFIGURATIONS
%----------------------------------------------------------------------------------------

\documentclass[paper=a4, fontsize=11pt]{scrartcl} % A4 paper and 11pt font size

\usepackage[T1]{fontenc} % Use 8-bit encoding that has 256 glyphs
\usepackage{fourier} % Use the Adobe Utopia font for the document - comment this line to return to the LaTeX default
\usepackage[english]{babel} % English language/hyphenation
\usepackage{amsmath,amsfonts,amsthm,amssymb} % Math packages
\usepackage{mathrsfs}

\usepackage{algorithm, algorithmic}
\renewcommand{\algorithmicrequire}{\textbf{Input:}} %Use Input in the format of Algorithm  
\renewcommand{\algorithmicensure}{\textbf{Output:}} %UseOutput in the format of Algorithm  

\usepackage{listings}
\lstset{language=Matlab}

\usepackage{lipsum} % Used for inserting dummy 'Lorem ipsum' text into the template

\usepackage{sectsty} % Allows customizing section commands
\allsectionsfont{\centering \normalfont\scshape} % Make all sections centered, the default font and small caps

\usepackage{fancyhdr} % Custom headers and footers
\pagestyle{fancyplain} % Makes all pages in the document conform to the custom headers and footers
\fancyhead{} % No page header - if you want one, create it in the same way as the footers below
\fancyfoot[L]{} % Empty left footer
\fancyfoot[C]{} % Empty center footer
\fancyfoot[R]{\thepage} % Page numbering for right footer
\renewcommand{\headrulewidth}{0pt} % Remove header underlines
\renewcommand{\footrulewidth}{0pt} % Remove footer underlines
\setlength{\headheight}{13.6pt} % Customize the height of the header

\numberwithin{equation}{section} % Number equations within sections (i.e. 1.1, 1.2, 2.1, 2.2 instead of 1, 2, 3, 4)
\numberwithin{figure}{section} % Number figures within sections (i.e. 1.1, 1.2, 2.1, 2.2 instead of 1, 2, 3, 4)
\numberwithin{table}{section} % Number tables within sections (i.e. 1.1, 1.2, 2.1, 2.2 instead of 1, 2, 3, 4)

\setlength\parindent{0pt} % Removes all indentation from paragraphs - comment this line for an assignment with lots of text

%----------------------------------------------------------------------------------------
%	TITLE SECTION
%----------------------------------------------------------------------------------------

\newcommand{\horrule}[1]{\rule{\linewidth}{#1}} % Create horizontal rule command with 1 argument of height

\title{	
\normalfont \normalsize 
\textsc{Shanghai Jiao Tong University, UM-SJTU JOINT INSTITUTE} \\ [25pt] % Your university, school and/or department name(s)
\horrule{0.5pt} \\[0.4cm] % Thin top horizontal rule
\huge Introduction to Numerical Analysis \\ HW4 \\ % The assignment title
\horrule{2pt} \\[0.5cm] % Thick bottom horizontal rule
}

\author{Yu Cang \\ 018370210001} % Your name

\date{\normalsize \today} % Today's date or a custom date

\begin{document}

\maketitle % Print the title

%----------------------------------------------------------------------------------------
%	PROBLEM 1
%----------------------------------------------------------------------------------------
\section{Legendre Polynomials}
	\begin{enumerate}
		\item
			\begin{proof}
				Let
				\begin{equation}
					\varphi(x) = (x^2-1)^n
				\end{equation}
				then
				\begin{equation}
					Q_n(x) = \frac{1}{2^n n!}\varphi^{(n)}(x)
				\end{equation}
				and
				\begin{equation}
					\varphi^{(k)}(1) = \varphi^{(k)}(-1) = 0 \ \  (\text{$k=0,1, ... , n-1$})
				\end{equation}
				Suppose $h(x) \in C^n(-1,1)$, then performing integration by parts
				\begin{equation}
					\begin{aligned}
						\int_{-1}^{1} P_n(x)h(x)dx 
						& = \frac{1}{2^n n!}\int_{-1}^{1}\varphi^{(n)}(x)h(x)dx \\
						& = -\frac{1}{2^n n!} \int_{-1}^{1}\varphi^{(n-1)}(x)h'(x)dx \\
						& = ...\\
						& = \frac{(-1)^n}{2^n n!}\int_{-1}^{1}\varphi(x)h^{(n)}(x)dx
					\end{aligned}
				\end{equation}
				Thus, the proof can be discussed on 2 cases
				
				\begin{enumerate}
					\item 
						If the order of $g(x)$ is less than $n$, then 
						\begin{equation}
							g^{(n)}(x) = 0
						\end{equation}
						Thus
						\begin{equation}
							\int_{-1}^{1}Q_n(x)Q_m(x)dx = 0 \ \ (\text{$n\neq m$})
						\end{equation}
											
					\item
						If $g(x) = Q_n(x)$, then the $n-th$ derivative of $g(x)$ is
						\begin{equation}
							g^{(n)}(x) = Q^{(n)}(x) = \frac{(2n)!}{2^n n!}
						\end{equation} 
						Thus
						\begin{equation}
							\begin{aligned}
								\int_{-1}^{1}Q_n(x)Q_m(x)dx	
								& = \int_{-1}^{1}Q_n^2(x)dx \\
								& = \frac{(-1)^n (2n)!}{2^{2n}(n!)^2}\int_{-1}^{1}(x^2-1)^n dx\\
								& = \frac{(2n)!}{2^{2n}(n!)^2}\int_{-1}^{1}(1-x^2)^n dx\\
								& = \frac{(2n)!}{2^{2n}(n!)^2} \int_{0}^{\pi/2} cos^{2n+1}t dt\\
								& = \frac{(2n)!}{2^{2n}(n!)^2} \frac{2 \times 4 \times ... \times (2n)}{1\times3\times ... \times (2n+1)}\\
								& = \frac{2}{2n+1} \ \ (\text{$n = m$})
							\end{aligned}
						\end{equation}
				\end{enumerate}
				
				Thus, $(Q_n)_{n\in\mathbb{N}}$ are a sequence of orthogonal polynomials.
			\end{proof}
		
		\item
			\begin{proof}
				Denote
				\begin{equation}
				\varphi(x) = (x^2-1)^n
				\end{equation}
				then
				\begin{equation}
				Q_n(x) = \frac{1}{2^n n!}\varphi^{(n)}(x)
				\end{equation}
				As the power of each item in $\varphi(x)$ is even when $\varphi(x)$ is extended, thus $\varphi^{(n)}(x)$ is even function if the order of derivative is even, and $\varphi^{(n)}(x)$ is odd function if the order of derivative is odd. \\
				Therefore $Q_n(x)$ is even function if $n$ is even, and $Q_n(x)$ is odd function if $n$ is odd. So, it can be summarized as $Q_n(-x) = (-1)^n Q_n(x)$. 
			\end{proof}
		
		\item
			
		\item
	
	\end{enumerate}

%----------------------------------------------------------------------------------------
%	PROBLEM 2
%----------------------------------------------------------------------------------------
\section{Interpolation}
	$f(2)$ can be determined using the Lagrange interpolation scheme.
	As the lagrange interpolation polynomial can be written as below, and $n=8$ in this case.
	\begin{equation}
		f(x) = \sum_{i=1}^{n}f(x_i) l_i(x)\label{lag_interp}
	\end{equation} 
	
	$l_i(x)$ are the base functions that can be written as below.
	\begin{equation}
		l_i(x) = \frac{(x-x_1)(x-x_2)...(x-x_{i-1})(x-x_{i+1})...(x-x_{n-1})(x-x_n)}{(x_i-x_1)(x_i-x_2)...(x_i-x_{i-1})(x_i-x_{i+1})...(x_i-x_{n-1})(x_i-x_n)}
	\end{equation} 
	
	$l_i(2)$ are calculated accordingly as below.
	
	\begin{center}
		\begin{tabular}{cccc}
			$l_1(2) = -0.0006$ & $l_2(2) = 0.1224$ & $l_3(2) = -0.5600$ & $l_4(2) = 1.0606$ \\
			$l_5(2) = 0.4167$ & $l_6(2) = -0.0400$ & $l_7(2) = 0.0012$ & $l_8(2) = -0.0003$\\
		\end{tabular}
	\end{center}


	Thus, $f(2)$ is calculated according to (\ref{lag_interp}) as 11.0.
	
%----------------------------------------------------------------------------------------
%	PROBLEM 3
%----------------------------------------------------------------------------------------
\section{Newton's form of interpolation polynomial}
	\begin{enumerate}
		\item 

		\item
		
		\item 
		
		\item 
		
		\item 
		
		\item 
		
		\item 
		
	\end{enumerate}

\end{document}