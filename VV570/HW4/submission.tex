%----------------------------------------------------------------------------------------
%	PACKAGES AND OTHER DOCUMENT CONFIGURATIONS
%----------------------------------------------------------------------------------------

\documentclass[paper=a4, fontsize=11pt]{scrartcl} % A4 paper and 11pt font size

\usepackage[T1]{fontenc} % Use 8-bit encoding that has 256 glyphs
\usepackage{fourier} % Use the Adobe Utopia font for the document - comment this line to return to the LaTeX default
\usepackage[english]{babel} % English language/hyphenation
\usepackage{amsmath,amsfonts,amsthm,amssymb} % Math packages
\usepackage{mathrsfs}

\usepackage{algorithm, algorithmic}
\renewcommand{\algorithmicrequire}{\textbf{Input:}} %Use Input in the format of Algorithm  
\renewcommand{\algorithmicensure}{\textbf{Output:}} %UseOutput in the format of Algorithm  

\usepackage{listings}
\lstset{language=Matlab}

\usepackage{lipsum} % Used for inserting dummy 'Lorem ipsum' text into the template

\usepackage{sectsty} % Allows customizing section commands
\allsectionsfont{\centering \normalfont\scshape} % Make all sections centered, the default font and small caps

\usepackage{fancyhdr} % Custom headers and footers
\pagestyle{fancyplain} % Makes all pages in the document conform to the custom headers and footers
\fancyhead{} % No page header - if you want one, create it in the same way as the footers below
\fancyfoot[L]{} % Empty left footer
\fancyfoot[C]{} % Empty center footer
\fancyfoot[R]{\thepage} % Page numbering for right footer
\renewcommand{\headrulewidth}{0pt} % Remove header underlines
\renewcommand{\footrulewidth}{0pt} % Remove footer underlines
\setlength{\headheight}{13.6pt} % Customize the height of the header

\numberwithin{equation}{section} % Number equations within sections (i.e. 1.1, 1.2, 2.1, 2.2 instead of 1, 2, 3, 4)
\numberwithin{figure}{section} % Number figures within sections (i.e. 1.1, 1.2, 2.1, 2.2 instead of 1, 2, 3, 4)
\numberwithin{table}{section} % Number tables within sections (i.e. 1.1, 1.2, 2.1, 2.2 instead of 1, 2, 3, 4)

\setlength\parindent{0pt} % Removes all indentation from paragraphs - comment this line for an assignment with lots of text

%----------------------------------------------------------------------------------------
%	TITLE SECTION
%----------------------------------------------------------------------------------------

\newcommand{\horrule}[1]{\rule{\linewidth}{#1}} % Create horizontal rule command with 1 argument of height

\title{	
\normalfont \normalsize 
\textsc{Shanghai Jiao Tong University, UM-SJTU JOINT INSTITUTE} \\ [25pt] % Your university, school and/or department name(s)
\horrule{0.5pt} \\[0.4cm] % Thin top horizontal rule
\huge Introduction to Numerical Analysis \\ HW4 \\ % The assignment title
\horrule{2pt} \\[0.5cm] % Thick bottom horizontal rule
}

\author{Yu Cang \\ 018370210001} % Your name

\date{\normalsize \today} % Today's date or a custom date

\begin{document}

\maketitle % Print the title

%----------------------------------------------------------------------------------------
%	PROBLEM 1
%----------------------------------------------------------------------------------------
\section{Legendre Polynomials}
	\begin{enumerate}
		\item
			\begin{proof}
				Let
				\begin{equation}
					\varphi(x) = (x^2-1)^n
				\end{equation}
				then
				\begin{equation}
					Q_n(x) = \frac{1}{2^n n!}\varphi^{(n)}(x)
				\end{equation}
				and
				\begin{equation}
					\varphi^{(k)}(1) = \varphi^{(k)}(-1) = 0 \ \  (\text{$k=0,1, ... , n-1$})
				\end{equation}
				Suppose $h(x) \in C^n(-1,1)$, then performing integration by parts
				\begin{equation}
					\begin{aligned}
						\int_{-1}^{1} P_n(x)h(x)dx 
						& = \frac{1}{2^n n!}\int_{-1}^{1}\varphi^{(n)}(x)h(x)dx \\
						& = -\frac{1}{2^n n!} \int_{-1}^{1}\varphi^{(n-1)}(x)h'(x)dx \\
						& = ...\\
						& = \frac{(-1)^n}{2^n n!}\int_{-1}^{1}\varphi(x)h^{(n)}(x)dx
					\end{aligned}
				\end{equation}
				Thus, the proof can be discussed on 2 cases
				
				\begin{enumerate}
					\item 
						If the order of $h(x)$ is less than $n$, then 
						\begin{equation}
							h^{(n)}(x) = 0
						\end{equation}
						Thus
						\begin{equation}
							\int_{-1}^{1}Q_n(x)Q_m(x)dx = 0 \ \ (\text{$n\neq m$})
						\end{equation}
											
					\item
						If $h(x) = Q_n(x)$, then the $n-th$ derivative of $g(x)$ is
						\begin{equation}
							h^{(n)}(x) = Q^{(n)}(x) = \frac{(2n)!}{2^n n!}
						\end{equation} 
						Thus
						\begin{equation}
							\begin{aligned}
								\int_{-1}^{1}Q_n(x)Q_m(x)dx	
								& = \int_{-1}^{1}Q_n^2(x)dx  \ \ (\text{$n = m$}) \\
								& = \frac{(-1)^n (2n)!}{2^{2n}(n!)^2}\int_{-1}^{1}(x^2-1)^n dx\\
								& = \frac{(2n)!}{2^{2n}(n!)^2}\int_{-1}^{1}(1-x^2)^n dx\\
								& = \frac{(2n)!}{2^{2n}(n!)^2} \int_{0}^{\pi/2} cos^{2n+1}t dt\\
								& = \frac{(2n)!}{2^{2n}(n!)^2} \frac{2 \times 4 \times ... \times (2n)}{1\times3\times ... \times (2n+1)}\\
								& = \frac{2}{2n+1}
							\end{aligned}
						\end{equation}
				\end{enumerate}
				
				Thus, $(Q_n)_{n\in\mathbb{N}}$ are a sequence of orthogonal polynomials.
			\end{proof}
		
		\item
			\begin{proof}
				Denote
				\begin{equation}
				\varphi(x) = (x^2-1)^n
				\end{equation}
				then
				\begin{equation}
				Q_n(x) = \frac{1}{2^n n!}\varphi^{(n)}(x)
				\end{equation}
				As the power of each item in $\varphi(x)$ is even when $\varphi(x)$ is extended, thus $\varphi^{(n)}(x)$ is even function if the order of derivative is even, and $\varphi^{(n)}(x)$ is odd function if the order of derivative is odd. \\
				Therefore $Q_n(x)$ is even function if $n$ is even, and $Q_n(x)$ is odd function if $n$ is odd. So, it can be summarized as $Q_n(-x) = (-1)^n Q_n(x)$. 
			\end{proof}
		
		\item
			\begin{proof}
				As $xQ_n(x)$ can be written as
				\begin{equation}
					xQ_n(x) = \sum_{i=0}^{n+1} a_i Q_i(x)
				\end{equation}
				According to the orthogonality of Legendre polynomials, $a_i = 0$ for $i = 0, 1, ... , n-2, n$, as
				\begin{equation}
					0 = \int_{-1}^{1}xQ_n(x) Q_i(x) dx = a_i \int_{-1}^{1}Q_i^2(x)dx \ \ (i = 0, 1, ... , n-2, n)
				\end{equation}
				Thus,  $xQ_n(x)$ can be written as
				\begin{equation}
					xQ_n(x) = a_{n-1} Q_{n-1}(x) + a_{n+1} Q_{n+1}(x)
				\end{equation}
				Since the highest order item in $xQ_n(x)$ is $x^{n+1}$ and its coefficient is $\frac{(2n)!}{2^n (n!)^2}$, thus
				\begin{equation}
					\frac{(2n)!}{2^n (n!)^2} = a_{n+1} \frac{(2n+2)!}{2^{n+1} (n+1)!^2}
				\end{equation}
				which implies that
				\begin{equation}
					a_{n+1} = \frac{n+1}{2n+1} 
				\end{equation}
				Denote
				\begin{equation}
					I_n = \int_{-1}^{1} xQ_{n-1}(x)Q_{n}(x)dx
 				\end{equation}
 				Then
 				\begin{equation}
 					I_n = a_{n-1}\int_{-1}^{1} Q_{n-1}^2(x)dx = a_{n-1} \frac{2}{2n-1}
 				\end{equation}
 				As
 				\begin{equation}
 					I_{n+1} = a_{n+1}\int_{-1}^{1}Q_{n+1}^2(x)dx = a_{n+1} \frac{2}{2n+3} = \frac{2(n+1)}{(2n+1)(2n+3)}
 				\end{equation}
 				Thus
 				\begin{equation}
 					I_n = \frac{2n}{(2n-1)(2n+1)}
 				\end{equation}
 				which implies that
 				\begin{equation}
 					a_{n-1} = \frac{n}{2n+1}
 				\end{equation}
 				Hence
 				\begin{equation}
 					(2n+1)xQ_n(x) = nQ_{n-1}(x) + (n+1)Q_{n+1}(x)
 				\end{equation}
			\end{proof}
			
		\item
			\begin{proof}
				As 
				\begin{equation}
					\begin{aligned}
						Q_n(x) & = \frac{1}{2^n n!}[(x+1)^n(x-1)^n]^{(n)} \\
							   & = \frac{2^n}{n!}[(\frac{x+1}{2})^n(\frac{x-1}{2})^n]^{(n)}
					\end{aligned}
				\end{equation}
				Denote
				\begin{equation}
					f(x) = (\frac{x+1}{2})^n
				\end{equation}
				and
				\begin{equation}
					g(x) = (\frac{x-1}{2})^n
				\end{equation}
				Then
				\begin{equation}
					f^{(k)}(x) = \frac{n!}{2^k (n-k)!} (\frac{x+1}{2})^{n-k}
				\end{equation}
				and
				\begin{equation}
					g^{(k)}(x) = \frac{n!}{2^k (n-k)!} (\frac{x-1}{2})^{n-k}
				\end{equation}
				Since
				\begin{equation}
					(fg)^{(n)} = \sum_{k=0}^{n} C_n^k f^{(k)}g^{(n-k)}
				\end{equation}
				Thus
				\begin{equation}
					\begin{aligned}
						Q_n(x) & = \frac{2^n}{n!} \sum_{k=0}^{n} C_n^k f^{(k)}(x) g^{(n-k)}(x)\\
							   & = \frac{2^n}{n!} \sum_{k=0}^{n} C_n^k \frac{n!}{2^k (n-k)!} (\frac{x+1}{2})^{n-k} \frac{n!}{2^{n-k} k!} (\frac{x-1}{2})^k \\
							   & = \sum_{k=0}^{n} C_n^k \frac{n!}{(n-k)!k!} (\frac{x+1}{2})^{n-k} (\frac{x-1}{2})^k\\
							   & = \sum_{k=0}^{n} (C_n^k)^2 (\frac{x+1}{2})^{n-k} (\frac{x-1}{2})^k\\
							   &= \sum_{k=0}^{n} (-1)^k \binom{n}{k}^2 (\frac{1+x}{2})^{n-k} (\frac{1-x}{2})^k
					\end{aligned}
				\end{equation}
			\end{proof}
	
	\end{enumerate}

%----------------------------------------------------------------------------------------
%	PROBLEM 2
%----------------------------------------------------------------------------------------
\section{Interpolation}
	$f(2)$ can be determined using the Lagrange interpolation scheme.
	As the lagrange interpolation polynomial can be written as below, and $n=8$ in this case.
	\begin{equation}
		f(x) = \sum_{i=1}^{n}f(x_i) l_i(x)\label{lag_interp}
	\end{equation} 
	
	$l_i(x)$ are the base functions that can be written as below.
	\begin{equation}
		l_i(x) = \frac{(x-x_1)(x-x_2)...(x-x_{i-1})(x-x_{i+1})...(x-x_{n-1})(x-x_n)}{(x_i-x_1)(x_i-x_2)...(x_i-x_{i-1})(x_i-x_{i+1})...(x_i-x_{n-1})(x_i-x_n)}
	\end{equation} 
	
	$l_i(2)$ are calculated accordingly as below.
	
	\begin{center}
		\begin{tabular}{cccc}
			$l_1(2) = -0.0006$ & $l_2(2) = 0.1224$ & $l_3(2) = -0.5600$ & $l_4(2) = 1.0606$ \\
			$l_5(2) = 0.4167$ & $l_6(2) = -0.0400$ & $l_7(2) = 0.0012$ & $l_8(2) = -0.0003$\\
		\end{tabular}
	\end{center}


	Thus, $f(2)$ is calculated according to (\ref{lag_interp}) as 11.0.
	
%----------------------------------------------------------------------------------------
%	PROBLEM 3
%----------------------------------------------------------------------------------------
\section{Newton's form of interpolation polynomial}
	\begin{enumerate}
		\item 
			\begin{enumerate}
				\item
					\begin{proof}
						Denote $P^1(x) = a_0 + a_1 x$, where $a_0$ and $a_1$ are coefficients to be determined.Then
						\begin{equation}
							\left\{
							\begin{aligned}
								f(x_0) & = a_0 + a_1 x_0 \\
								f(x_1) & = a_0 + a_1 x_1
							\end{aligned}
							\right.
						\end{equation}
						$a_0$ and $a_1$ are solved as below
						\begin{equation}
							\left\{
							\begin{aligned}
								a_0 & = \frac{x_1 f(x_0) - x_0 f(x_1)}{x_1 - x_0}\\
								a_1 & = \frac{f(x_1)-f(x_0)}{x_1 - x_0}
							\end{aligned}
							\right.
						\end{equation}
						Thus
						\begin{equation}
							\begin{aligned}
								P^1(x) & = \frac{x_1 f(x_0) - x_0 f(x_1)}{x_1 - x_0} + \frac{f(x_1) - f(x_0)}{x_1 - x_0} x \\
									   & = \frac{(x_1 - x_0)f(x_0) - x_0 (f(x_1) - f(x_0))}{x_1 - x_0} + \frac{f(x_1) - f(x_0)}{x_1 - x_0}x \\
									   & = f(x_0) + \frac{f(x_1) - f(x_0)}{x_1 - x_0} (x-x_0)\\
									   & = P^0(x) + \frac{f(x_1) - f(x_0)}{x_1 - x_0} (x-x_0)
							\end{aligned}
						\end{equation}
					\end{proof}
					 
				\item 
					If $P^2(x) = P^1(x) + R(x)$, then $R(x_0) = 0$ and $R(x_1) = 0$, thus $R(x) = c(x-x_1)(x-x_0)$, where $c$ is the coefficient to be determined. 
					Since
					\begin{equation}
						R(x_2) = P^2(x_2) - P^1(x_2) 
					\end{equation}
					Thus
					\begin{equation}
						c(x_2 - x_1)(x_2 - x_0) = f(x_2) - [f(x_0) + \frac{f(x_1) - f(x_0)}{x_1 - x_0}(x_2 - x_0)]
					\end{equation}
					$c$ is solved as 
					\begin{equation}
						c = \frac{1}{x_2 - x_0} [\frac{f(x_2) - f(x_1)}{x_2 -x_1} - \frac{f(x_1) - f(x_0)}{x_1 - x_0}]
					\end{equation}
					Hence
					\begin{equation}
						R(x) = \frac{1}{x_2 - x_0} [\frac{f(x_2) - f(x_1)}{x_2 -x_1} - \frac{f(x_1) - f(x_0)}{x_1 - x_0}] (x-x_1)(x-x_0)
					\end{equation}
				
				\item
					\begin{proof}
						As have been shown above, the statement is valid for $j = 1, 2$. Suppose it's still valid for $j = n$, then
						\begin{equation}
							P^n(x) = P^{n-1}(x) + a_n \prod_{k=0}^{n-1}(x - x_k)
						\end{equation}
						where $a_n$ depends only on $a_0, a_1, ... , a_n$.\\
						Let $P^{n+1}(x) = P^n(x) + R(x)$, then $R(x_k) = 0$ for $k = 0, 1, 2, ... , n$, thus
						\begin{equation}
							R(x) = a_{n+1} \prod_{k=0}^{n}(x-x_k)
						\end{equation}
						Then $a_{n+1}$ is solved as
						\begin{equation}
							a_{n+1} = \frac{f(x_{n+1}) - P^n(x_{n+1})}{\prod_{k=0}^{n}(x-x_k)}
						\end{equation}
						As $a_n$ only depends on $a_0, a_1, ... , a_n$, $P^n(x_{n+1})$ only depends on $x_0, x_1, ... , x_n, x_{n+1}$, thus $a_{n+1}$ only depends on $a_0, a_1, ... , a_n, a_{n+1}$.
					\end{proof}
				
			\end{enumerate}
		
		\item
			\begin{proof}
				As
				\begin{equation}
					P^j(x) = P^{j-1}(x) + a_j \prod_{k=0}^{j-1}(x - x_k)
				\end{equation}
				Thus, by recursion, $P^j(x)$ can be written as
				\begin{equation}
					P^n(x) = P^0(x) + \sum_{j=1}^{n} a_j \prod_{k=0}^{j-1}(x-x_k)
				\end{equation}
			\end{proof}
		
		\item 
			\begin{proof}
				(haven't figured out yet...)
			\end{proof}
		
		\item 
			Algorithm that calculate the approximation of $f(x)$ is given as below
			
			\begin{algorithm}
			\caption{Calculation of the approximated value of $f(x)$}
			\begin{algorithmic}[1]
				\REQUIRE $x$, $n$, $x[i]$, $f[i]$ ($i = 0, 1, ... , n$)
				\ENSURE $P^n(x) \approx f(x)$
				\STATE $ret \gets 0$
				\STATE $prod \gets 1$
				
				\FOR{$i=0 \to n$}
					\STATE $coef[i] \gets f[i]$
				\ENDFOR

				\FOR{$i=0 \to n$}
					\STATE $ret \gets coef[i] * prod$
					\STATE $prod \gets prod * (x-x_i)$

					\FOR{$j= n \to i+1$}
						\STATE $coef[j] \gets \frac{coef[j] - coef[j-1]}{x[j] - x[j-(i+1)]}$
					\ENDFOR

				\ENDFOR

				\RETURN ret
			\end{algorithmic}  
			\end{algorithm}
		\item 
			\begin{proof}
				As
				\begin{equation}
					f[x_k, x_{k+1}] = \frac{f_{k+1}-f_k}{x_{k+1} - x_k} = \frac{\triangledown f_k}{h}
				\end{equation}
				The statement is valid for $m = 1$.\\
				Assume it is still valid for $m = n$, then
				\begin{equation}
					f[x_i, x_{i+1}, ... , x_{i+n}] = \frac{\triangledown^n f_i}{n! h^n}
				\end{equation}
				Consider $m = n+1$, by the definition of $f[x_i, x_{i+1}, ... , x_{i+n+1}]$
				\begin{equation}
					f[x_i, x_{i+1}, ... , x_{i+n+1}] = \frac{f[x_{i+1}, ... , x_{i+n+1}]-f[x_i, x_{i+1}, ... , x_n]}{x_{i+n+1} - x_i}
				\end{equation}
				By the induction hypothesis, it can be simplified as
				\begin{equation}
					f[x_i, x_{i+1}, ... , x_{i+n+1}] = \frac{1}{(n+1)h} \frac{1}{n! h^n} [\triangledown^n f_{i+1} - \triangledown^n f_i]
				\end{equation}
				By the definition of operator $\triangledown$, it can be further simplified as
				\begin{equation}
					f[x_i, x_{i+1}, ... , x_{i+n+1}] = \frac{\triangledown^{n+1} f_i}{(n+1)! h^{n+1}}
				\end{equation}
				Thus, the statement is valid for all $m \geq 1$.
				
			\end{proof}
		\item 
			(haven't figured out yet...)
		
		\item 
		Algorithm that calculate the approximation of $f(x)$ with equidistant nodes is given as below
		
		\begin{algorithm}
			\caption{Calculation of the approximated value of $f(x)$ with equidistant nodes}
			\begin{algorithmic}[1]
				\REQUIRE $h$, $n$, $x$, $f[i]$ ($i = 0, 1, ... , n$)
				\ENSURE $P^n(x) \approx f(x)$
				\STATE $ret \gets 0$
				\STATE $prod \gets 1$
				\STATE $t \gets \frac{x-x_0}{h}$
				
				\FOR{$i=0 \to n$}
					\STATE $coef[i] \gets f[i]$
				\ENDFOR
				
				\FOR{$i=0 \to n$}
					\STATE $ret \gets coef[i] \times prod$
					\STATE $prod \gets prod \times \frac{t- i}{i+1}$
					
					\FOR{$j= n \to i+1$}
						\STATE $coef[j] \gets coef[j] - coef[j-1]$
					\ENDFOR
				\ENDFOR

				\RETURN ret
			\end{algorithmic}  
		\end{algorithm}
		
	\end{enumerate}

\end{document}