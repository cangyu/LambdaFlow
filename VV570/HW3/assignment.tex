%----------------------------------------------------------------------------------------
%	PACKAGES AND OTHER DOCUMENT CONFIGURATIONS
%----------------------------------------------------------------------------------------

\documentclass[paper=a4, fontsize=11pt]{scrartcl} % A4 paper and 11pt font size

\usepackage[T1]{fontenc} % Use 8-bit encoding that has 256 glyphs
\usepackage{fourier} % Use the Adobe Utopia font for the document - comment this line to return to the LaTeX default
\usepackage[english]{babel} % English language/hyphenation
\usepackage{amsmath,amsfonts,amsthm,amssymb} % Math packages
\usepackage{mathrsfs}

\usepackage{algorithm, algorithmic}
\renewcommand{\algorithmicrequire}{\textbf{Input:}} %Use Input in the format of Algorithm  
\renewcommand{\algorithmicensure}{\textbf{Output:}} %UseOutput in the format of Algorithm  

\usepackage{listings}
\lstset{language=Matlab}

\usepackage{lipsum} % Used for inserting dummy 'Lorem ipsum' text into the template

\usepackage{sectsty} % Allows customizing section commands
\allsectionsfont{\centering \normalfont\scshape} % Make all sections centered, the default font and small caps

\usepackage{fancyhdr} % Custom headers and footers
\pagestyle{fancyplain} % Makes all pages in the document conform to the custom headers and footers
\fancyhead{} % No page header - if you want one, create it in the same way as the footers below
\fancyfoot[L]{} % Empty left footer
\fancyfoot[C]{} % Empty center footer
\fancyfoot[R]{\thepage} % Page numbering for right footer
\renewcommand{\headrulewidth}{0pt} % Remove header underlines
\renewcommand{\footrulewidth}{0pt} % Remove footer underlines
\setlength{\headheight}{13.6pt} % Customize the height of the header

\numberwithin{equation}{section} % Number equations within sections (i.e. 1.1, 1.2, 2.1, 2.2 instead of 1, 2, 3, 4)
\numberwithin{figure}{section} % Number figures within sections (i.e. 1.1, 1.2, 2.1, 2.2 instead of 1, 2, 3, 4)
\numberwithin{table}{section} % Number tables within sections (i.e. 1.1, 1.2, 2.1, 2.2 instead of 1, 2, 3, 4)

\setlength\parindent{0pt} % Removes all indentation from paragraphs - comment this line for an assignment with lots of text

%----------------------------------------------------------------------------------------
%	TITLE SECTION
%----------------------------------------------------------------------------------------

\newcommand{\horrule}[1]{\rule{\linewidth}{#1}} % Create horizontal rule command with 1 argument of height

\title{	
\normalfont \normalsize 
\textsc{Shanghai Jiao Tong University, UM-SJTU JOINT INSTITUTE} \\ [25pt] % Your university, school and/or department name(s)
\horrule{0.5pt} \\[0.4cm] % Thin top horizontal rule
\huge Introduction to Numerical Analysis \\ HW3 \\ % The assignment title
\horrule{2pt} \\[0.5cm] % Thick bottom horizontal rule
}

\author{Yu Cang \\ 018370210001} % Your name

\date{\normalsize \today} % Today's date or a custom date

\begin{document}

\maketitle % Print the title

%----------------------------------------------------------------------------------------
%	PROBLEM 1
%----------------------------------------------------------------------------------------
\section{Cantor's Set}
	\begin{enumerate}
		\item
			\begin{proof}
				As $C_i$ is closed and $C$ is the intersection of these closed sets, $C$ is closed.\\
				It's clear that $C$ is bounded. Thus, by Heine-Borel theorem, $C$ is compact.
			\end{proof}
		\item
			\begin{proof}
				Suppose $x \in C_m$, $y \in C_n$ and $x < y$. There are $2^m$ subsets in $C_m$ and the length of each subset is $\frac{1}{3^m}$. Also, there are $2^n$ subsets in $C_n$ and the length of each subset is $\frac{1}{3^n}$. As $C \subset C_m \cap C_n$, suppose $ m \leq n$, there must exist a subset in $C_n$ s.t. $x \in C_n$. \\
				If $x$ and $y$ lie in the same subset of $C_n$, with further division of the subset, there will be a gap between $x$ and $y$, thus, there exists an element $z$ lies in the gap and satisfies $x < z < y$.\\
				If $x$ and $y$ lie in different subsets of $C_n$, denoted as $C_{n, i}$ and $C_{n, j}$,it's obvious that such a $z$ exists in the gap between $C_{n, i}$ and $C_{n, j}$ and satisfies $x < z < y$.
			\end{proof}
		\item 
			\begin{enumerate}
				\item
					0
				\item
					\begin{proof}
						As there are $2^n$ subsets in $C_n$ and the length of each subset is $\frac{1}{3^n}$, thus the Lebesgue measure of $C_n$ is the sum of the Lebesgue measure of each closed subset. Thus $\lambda(C_n) = (\frac{2}{3})^n$.\\
						As $C = \bigcap\limits_{n=1}^{\infty} C_n$, then $\lambda(C) \leq \lambda(C_n) = (\frac{2}{3})^n$, thus $\lambda(C) = 0$.
					\end{proof}
			\end{enumerate}
		\item 
			\begin{enumerate}
				\item 
					\begin{proof}
						As the end points of each subset in $C_n$ is not removed in any subdivision, thus $C$ is not empty.
					\end{proof}
				\item 
					\begin{equation}
						x = \sum_{i=1}^{\infty} \frac{a_i}{3^i}, \ \ \text{with $a_i \in \{0, 2\}$}
					\end{equation}
				\item 
					Consider the $i-th$ digit in elemnet $s_i$, it will be possible to construct such an element that the $i-th$ digit is different from that in $s_i$, thus $s$ is not included in the original list.
				\item 
					\begin{proof}
						Suppose $C$ is countable, express each $x \in C$ in ternary form, then each digit in $x$ is either 0 or 2. Thus, the choice of each digit appears to be binary. Consider the $i-th$ digit in elemnet $x_i$, it will be possible to construct such an element $t$, whose $i-th$ digit is different from that in $x_i$(complementary). Thus $t$ is not included in the original list, which implies the assumption fails.
					\end{proof}
			\end{enumerate}
		
		\item 
			Althouth $C$ is uncountable, but the measure of its complement is 1, which is illustrated below, thus the measure of $C$ is 0.		
			\begin{equation}
				\begin{aligned}
					\lambda(C^c) & = \frac{1}{3} + 2 * \frac{1}{9} + ... \\
					             & = \frac{1}{3}\sum_{i=0}^{\infty} (\frac{2}{3})^n\\
					             & = \frac{1}{3} \lim\limits_{n\rightarrow\infty}\frac{1-(2/3)^n}{1-(2/3)}\\
					             & = 1
          		\end{aligned}
			\end{equation} 
		
	\end{enumerate}


%----------------------------------------------------------------------------------------
%	PROBLEM 2
%----------------------------------------------------------------------------------------
\section{Cantor's Function}
	\begin{enumerate}
		\item
	
		\item
	
		\item 
		
		\item 
		
	\end{enumerate}


%----------------------------------------------------------------------------------------
%	PROBLEM 3
%----------------------------------------------------------------------------------------
\section{Taylor's theorem}
	\begin{enumerate}
		\item 

		\item 
		
	\end{enumerate}


%----------------------------------------------------------------------------------------
%	PROBLEM 4
%----------------------------------------------------------------------------------------
\section{Convergence of rationals to irrationals}
	\begin{enumerate}
		\item
	
		\item
		
		\item
		
	\end{enumerate}

\end{document}