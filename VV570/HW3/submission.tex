%----------------------------------------------------------------------------------------
%	PACKAGES AND OTHER DOCUMENT CONFIGURATIONS
%----------------------------------------------------------------------------------------

\documentclass[paper=a4, fontsize=11pt]{scrartcl} % A4 paper and 11pt font size

\usepackage[T1]{fontenc} % Use 8-bit encoding that has 256 glyphs
\usepackage{fourier} % Use the Adobe Utopia font for the document - comment this line to return to the LaTeX default
\usepackage[english]{babel} % English language/hyphenation
\usepackage{amsmath,amsfonts,amsthm,amssymb} % Math packages
\usepackage{mathrsfs}

\usepackage{algorithm, algorithmic}
\renewcommand{\algorithmicrequire}{\textbf{Input:}} %Use Input in the format of Algorithm  
\renewcommand{\algorithmicensure}{\textbf{Output:}} %UseOutput in the format of Algorithm  

\usepackage{listings}
\lstset{language=Matlab}

\usepackage{lipsum} % Used for inserting dummy 'Lorem ipsum' text into the template

\usepackage{sectsty} % Allows customizing section commands
\allsectionsfont{\centering \normalfont\scshape} % Make all sections centered, the default font and small caps

\usepackage{fancyhdr} % Custom headers and footers
\pagestyle{fancyplain} % Makes all pages in the document conform to the custom headers and footers
\fancyhead{} % No page header - if you want one, create it in the same way as the footers below
\fancyfoot[L]{} % Empty left footer
\fancyfoot[C]{} % Empty center footer
\fancyfoot[R]{\thepage} % Page numbering for right footer
\renewcommand{\headrulewidth}{0pt} % Remove header underlines
\renewcommand{\footrulewidth}{0pt} % Remove footer underlines
\setlength{\headheight}{13.6pt} % Customize the height of the header

\numberwithin{equation}{section} % Number equations within sections (i.e. 1.1, 1.2, 2.1, 2.2 instead of 1, 2, 3, 4)
\numberwithin{figure}{section} % Number figures within sections (i.e. 1.1, 1.2, 2.1, 2.2 instead of 1, 2, 3, 4)
\numberwithin{table}{section} % Number tables within sections (i.e. 1.1, 1.2, 2.1, 2.2 instead of 1, 2, 3, 4)

\setlength\parindent{0pt} % Removes all indentation from paragraphs - comment this line for an assignment with lots of text

%----------------------------------------------------------------------------------------
%	TITLE SECTION
%----------------------------------------------------------------------------------------

\newcommand{\horrule}[1]{\rule{\linewidth}{#1}} % Create horizontal rule command with 1 argument of height

\title{	
\normalfont \normalsize 
\textsc{Shanghai Jiao Tong University, UM-SJTU JOINT INSTITUTE} \\ [25pt] % Your university, school and/or department name(s)
\horrule{0.5pt} \\[0.4cm] % Thin top horizontal rule
\huge Introduction to Numerical Analysis \\ HW3 \\ % The assignment title
\horrule{2pt} \\[0.5cm] % Thick bottom horizontal rule
}

\author{Yu Cang \\ 018370210001} % Your name

\date{\normalsize \today} % Today's date or a custom date

\begin{document}

\maketitle % Print the title

%----------------------------------------------------------------------------------------
%	PROBLEM 1
%----------------------------------------------------------------------------------------
\section{Cantor's Set}
	\begin{enumerate}
		\item
			\begin{proof}
				As $C_i$ is closed and $C$ is the intersection of these closed sets, $C$ is closed.\\
				It's clear that $C$ is bounded. Thus, by Heine-Borel theorem, $C$ is compact.
			\end{proof}
		\item
			\begin{proof}
				Suppose $x \in C_m$, $y \in C_n$ and $x < y$. There are $2^m$ subsets in $C_m$ and the length of each subset is $\frac{1}{3^m}$. Also, there are $2^n$ subsets in $C_n$ and the length of each subset is $\frac{1}{3^n}$. As $C \subset C_m \cap C_n$, suppose $ m \leq n$, there must exist a subset in $C_n$ s.t. $x \in C_n$. \\
				If $x$ and $y$ lie in the same subset of $C_n$, with further division of the subset, there will be a gap between $x$ and $y$, thus, there exists an element $z$ lies in the gap and satisfies $x < z < y$.\\
				If $x$ and $y$ lie in different subsets of $C_n$, denoted as $C_{n, i}$ and $C_{n, j}$,it's obvious that such a $z$ exists in the gap between $C_{n, i}$ and $C_{n, j}$ and satisfies $x < z < y$.
			\end{proof}
		\item 
			\begin{enumerate}
				\item
					0
				\item
					\begin{proof}
						As there are $2^n$ subsets in $C_n$ and the length of each subset is $\frac{1}{3^n}$, thus the Lebesgue measure of $C_n$ is the sum of the Lebesgue measure of each closed subset. Thus $\lambda(C_n) = (\frac{2}{3})^n$.\\
						As $C = \bigcap\limits_{n=1}^{\infty} C_n$, then $\lambda(C) \leq \lambda(C_n) = (\frac{2}{3})^n$, thus $\lambda(C) = 0$.
					\end{proof}
			\end{enumerate}
		\item 
			\begin{enumerate}
				\item 
					\begin{proof}
						As the end points of each subset in $C_n$ is not removed in any subdivision, thus $C$ is not empty.
					\end{proof}
				\item 
					\begin{equation}
						x = \sum_{i=1}^{\infty} \frac{a_i}{3^i}, \ \ \text{with $a_i \in \{0, 2\}$}
					\end{equation}
				\item 
					Consider the $i-th$ digit in elemnet $s_i$, it will be possible to construct such an element that the $i-th$ digit is different from that in $s_i$, thus $s$ is not included in the original list.
				\item 
					\begin{proof}
						Suppose $C$ is countable, express each $x \in C$ in ternary form, then each digit in $x$ is either 0 or 2. Thus, the choice of each digit appears to be binary. Consider the $i-th$ digit in elemnet $x_i$, it will be possible to construct such an element $t$, whose $i-th$ digit is different from that in $x_i$(complementary). Thus $t$ is not included in the original list, which implies the assumption fails.
					\end{proof}
			\end{enumerate}
		
		\item 
			Althouth $C$ is uncountable, but the measure of its complement is 1, which is illustrated below, thus the measure of $C$ is 0.		
			\begin{equation}
				\begin{aligned}
					\lambda(C^c) & = \frac{1}{3} + 2 * \frac{1}{9} + ... \\
					             & = \frac{1}{3}\sum_{i=0}^{\infty} (\frac{2}{3})^n\\
					             & = \frac{1}{3} \lim\limits_{n\rightarrow\infty}\frac{1-(2/3)^n}{1-(2/3)} = 1
          		\end{aligned}
			\end{equation} 
		
	\end{enumerate}


%----------------------------------------------------------------------------------------
%	PROBLEM 2
%----------------------------------------------------------------------------------------
\section{Cantor's Function}
	\begin{enumerate}
		\item
			\begin{proof}
				It's easy to verify that $f_0, f_1$ are monotonically increasing continuous functions.\\
				Suppose $f_n$ are still monotonically increasing continuous functions for $ n = k$.\\
				For $n = k+1$, as $f_{k+1}$ is only different from $f_k$ on each closed subset of $C_k$, whose length is $\frac{1}{3^n}$ and is denoted as $I_{k, p} (1\leq p \leq 2^k$) here, it is left to prove that $f_{k+1}$ remains  monotonically increasing continuous on each $I_{k, p}$ after the construction process.\\
				Since $f_k$ is linear on each $I_{k, p}$, and the recursive construction process doesn't change the values on each ending points, on which the values of $f_k$ is denoted as $a, b$ recursively, the subset of $I_{k, p}$ is valid and the value of $f_{k+1}$ on the central part, whose length is $\frac{1}{3^{k+1}}$, is $\frac{a+b}{2}$. As the measure of the remaining parts of $I_{k, p}$ is not 0, the linear function connecting each ending points is obviously valid. Thus, the new function $f_{k+1}$ is still  monotonically increasing continuous.
			\end{proof}
		\item
			\begin{proof}
				Let $g_n(x) = \vert f_n(x) - f(x) \vert$, then $g_n(x)$ is positive on $C_n$ and 0 on other places. Further, $g_n(x)$ holds the same value on each closed subset of $C_n$.\\
				Given any $\xi > 0$, it is left to find $N$ s.t. $g_n(x) < \xi$ when $n > N$. \\
				As $g_n(x)$ reaches its extream $\frac{1}{6} \frac{1}{3^n}$ at the $\frac{1}{3}$ and $\frac{2}{3}$ of each compact subset of $C_n$, thus, given any $\xi$
				
				\begin{equation}
					\frac{1}{6} \frac{1}{3^n} < \xi
				\end{equation}
				
				which is equivalent to 
				
				\begin{equation}
					n > \frac{ln(\frac{1}{6\xi})}{ln(3)} \triangleq N
				\end{equation}
				
				and the target $N$ is given as above.
			\end{proof}
		\item 
			\begin{enumerate}
				\item 
					\begin{proof}
						As $f_c(x)$ is continous over $[0, 1]$, it is left to prove that a continous function on a compact is uniformly continous, which can be proved by contradiction.\\
						(the proof haven't figured out yet...)
					\end{proof}
				
				\item 
					\begin{proof}
						For any $x \in (0, 1] \cap C$, $f_c(x)$ can be expressed as below
						
						\begin{equation}
							f_c(x) = sup\{ f_c(y) | y < x, y \in [0, 1] \backslash C  \}
						\end{equation}
						
						Thus $f_c(x)$ is monotonically increasing.
					\end{proof}
				
				\item
					\begin{proof}
						As $f_c(x)$ is constant on each segment dropped by $C$, thus $f_c'(x)$ is almost 0 everywhere.
					\end{proof} 
				
			\end{enumerate}
			
		\item 
			\begin{proof}
				Assume $f_c(x)$  is absolutely continous, then its Riemann integral exists, which can be expressed as 
				\begin{equation}
					\int_{0}^{1} f_c'(x) dx = f_c(1) - f_c(0) \label{cantor_rie}
				\end{equation} 
				
				However, the $LHS$ of (\ref{cantor_rie}) is 0 as $f_c'(x)$ is 0 almost everywhere, while the $RHS$ of (\ref{cantor_rie}) is 1. Thus its riemann integral doesn't exist, and therefore the assumption fails.
				
			\end{proof}
		
			
			
			
		
	\end{enumerate}


%----------------------------------------------------------------------------------------
%	PROBLEM 3
%----------------------------------------------------------------------------------------
\section{Taylor's theorem}
	\begin{enumerate}
		\item 
			\begin{proof}
				For $n=0$, apply the fundamental theorem of calculus to $f(x)$ and its derivatives, $f(x)$ can be written as below. It's clear that the theorem is valid in such case.
				\begin{equation}
					f(x) = f(a) + \int_{a}^{x} f'(t) dt
				\end{equation}
				
				Suppose the theorem is still valid for $n = k$, then $f(x)$ can be written as 
				\begin{equation}
					f(x) = \sum_{i=0}^{k}\frac{f^{i}(a)}{i!}(x-a)^i + \frac{1}{k!}\int_{a}^{x} (x-t)^k f^{(k+1)}(t) dt \label{rhs}
				\end{equation}
				
				For $n = k+1$, as $f^{(k+1)}$ is absolutely continous, apply the fundamental theorem of calculus to $f^{(k+1)}$, the second part of RHS of $f(x)$ can be written as below
				\begin{equation}
					\begin{aligned}
						RHS_2 & = \frac{1}{k!}\int_{a}^{x} (x-t)^k [f^{(k+1)}(a) + \int_{a}^{t}f^{(k+2)}(u) du] dt\\
						           & = \frac{f^{(k+1)}(a)}{k!}\int_{a}^{x}(x-t)^k dt + \frac{1}{k!}\int_{a}^{x}(x-t)^k \int_{a}^{t} f^{(k+2)}(u) du dt\\
						           & =  \frac{f^{k+1}(a)}{(k+1)!}(x-a)^{k+1} + \frac{1}{k!}\int_{a}^{x}(x-t)^k \int_{a}^{t} f^{(k+2)}(u) du dt
					\end{aligned}
				\end{equation}
				Let
				\begin{equation}
					g(t) = \int_{a}^{t} f^{(k+2)}(u) du
				\end{equation}
				$RHS_2$ can be further expressed as
				\begin{equation}
					\begin{aligned}
						RHS_2 & =  \frac{f^{k+1}(a)}{(k+1)!}(x-a)^{k+1} + \frac{1}{k!}\int_{a}^{x}(x-t)^k g(t) dt\\
								   & =  \frac{f^{k+1}(a)}{(k+1)!}(x-a)^{k+1} - \frac{1}{(k+1)!}\int_{a}^{x} g(t) d(x-t)^{k+1}\\
								   & =  \frac{f^{k+1}(a)}{(k+1)!}(x-a)^{k+1} - \frac{1}{(k+1)!}[\left. (x-t)^{k+1} g(t) \right|_{a}^{x} - \int_{a}^{x}(x-t)^{k+1} g'(t) dt]
					\end{aligned}
				\end{equation}
				As $g(a) = 0$ and $g'(t) = f^{(k+2)}(t)$, $RHS_2$ can be simplified as 
				\begin{equation}
					RHS_2 = \frac{f^{k+1}(a)}{(k+1)!}(x-a)^{k+1} + \frac{1}{(k+1)!}\int_{a}^{x}(x-t)^{k+1}  f^{(k+2)}(t)dt \label{rhs2}
				\end{equation}
				
				Substitude (\ref{rhs2}) into (\ref{rhs}), $f(x)$ can be expressed as below when $n = k+1$
				\begin{equation}
					\begin{aligned}
						f(x) & = \sum_{i=0}^{k+1}\frac{f^{i}(a)}{i!}(x-a)^i + \frac{1}{(k+1)!}\int_{a}^{x} (x-t)^{k+1} f^{(k+2)}(t) dt\\
						      & = \sum_{i=0}^{n}\frac{f^{i}(a)}{i!}(x-a)^i + \frac{1}{n!}\int_{a}^{x} (x-t)^{n} f^{(n+1)}(t) dt\\
					\end{aligned}
				\end{equation}
				Thus the theorem is still valid when $n$ is extended to $k+1$.
			\end{proof}

		\item 
			\begin{proof}
				It is left to prove that there exists $c\in [a, x]$ s.t. 
				\begin{equation}
					\frac{1}{n!}\int_{a}^{x} (x-t)^{n} f^{(n+1)}(t) dt = \frac{f^{(n+1)}(c)}{(n+1)!}(x-a)^{n+1}
				\end{equation}
				which is equal to 
				\begin{equation}
					(n+1)\int_{a}^{x} (x-t)^{n} f^{(n+1)}(t) dt = f^{(n+1)}(c)(x-a)^{n+1} \label{target}
				\end{equation}
				
				Denote $m = min(f^{(n+1)}(x))$ and $ M = max(f^{(n+1)}(x))$, apply the first medium value theorem, there exists $c \in (m, M)$ s.t. 
				\begin{equation}
					\int_{a}^{x} (x-t)^{n} f^{(n+1)}(t) dt = f^{(n+1)}(c) \int_{a}^{x} (x-t)^{n} dt
				\end{equation}
				Thus, (\ref{target}) is verified.
			\end{proof}
		
	\end{enumerate}


%----------------------------------------------------------------------------------------
%	PROBLEM 4
%----------------------------------------------------------------------------------------
\section{Convergence of rationals to irrationals}
	\begin{enumerate}
		\item
			\begin{proof}
				As $e$ can be written as 
				\begin{equation}
					e = \sum_{i=0}^{\infty} \frac{1}{i!}
				\end{equation}
				
				Suppose it is rational, then it can be written as below, where $p, q$ are prime to each other.
				
				\begin{equation}
					e = \frac{p}{q} 
				\end{equation}
				Thus
				\begin{equation}
					p(q-1)! = q! \sum_{i=0}^{q} \frac{1}{i!} +  q! \sum_{i=q+1}^{\infty} \frac{1}{i!} 
				\end{equation}
				Since both LHS  and  the first part of RHS are integers, the second part of RHS(denoted as $RHS_2$) should be an integer as well.//
				$RHS_2$ can be further expanded as below
				\begin{equation}
					\begin{aligned}
						RHS_2 & = \frac{1}{q+1} + \frac{1}{(q+1)(q+2)} + \frac{1}{(q+1)(q+2)(q+3)} + ...\\
						           & <  \frac{1}{q+1} + \frac{1}{(q+1)^2} + \frac{1}{(q+1)^3} + ...\\
						           & = \frac{1}{q} < 1 \\
					\end{aligned}
				\end{equation}
				Therefore $RHS_2$ is not an integer as $RHS_2 > 0$. Thus the assumption fails, and $e$  is irrational.
			\end{proof}
	
		\item
			\begin{proof}
				It's clear that $u_n$ is increasing, and the limit is $e$, so the maximum distance between 2 element can be denoted as 
				\begin{equation}
					d_{max} < e - (1+\frac{1}{n})^n
				\end{equation} 
				
				For any $\xi > 0$, let $d_{max} < \xi$, it is satisfied when $n > N $, where $N$ is the index of the first element s.t. $u_N > e-\xi$.\\
				Thus $u_n$ is a cauchy sequence converging to $e$.
			\end{proof}
		
		\item
			No, as the limit of a rational sequence may not converging to a rational.
	\end{enumerate}

\end{document}