%----------------------------------------------------------------------------------------
%	PACKAGES AND OTHER DOCUMENT CONFIGURATIONS
%----------------------------------------------------------------------------------------
\documentclass[paper=a4, fontsize=11pt]{scrartcl} % A4 paper and 11pt font size

\usepackage[T1]{fontenc} % Use 8-bit encoding that has 256 glyphs
\usepackage{fourier} % Use the Adobe Utopia font for the document - comment this line to return to the LaTeX default
\usepackage[english]{babel} % English language/hyphenation
\usepackage{amsmath,amsfonts,amsthm,amssymb} % Math packages
\usepackage{mathrsfs}
\usepackage{algorithm, algorithmic}
\renewcommand{\algorithmicrequire}{\textbf{Input:}} %Use Input in the format of Algorithm  
\renewcommand{\algorithmicensure}{\textbf{Output:}} %UseOutput in the format of Algorithm  
\usepackage{listings}
\lstset{language=Matlab}
\usepackage{enumerate}
\usepackage{graphicx}

\usepackage{lipsum} % Used for inserting dummy 'Lorem ipsum' text into the template
\usepackage{sectsty} % Allows customizing section commands
\allsectionsfont{\centering \normalfont\scshape} % Make all sections centered, the default font and small caps
\usepackage{fancyhdr} % Custom headers and footers
\pagestyle{fancyplain} % Makes all pages in the document conform to the custom headers and footers
\fancyhead{} % No page header - if you want one, create it in the same way as the footers below
\fancyfoot[L]{} % Empty left footer
\fancyfoot[C]{} % Empty center footer
\fancyfoot[R]{\thepage} % Page numbering for right footer
\renewcommand{\headrulewidth}{0pt} % Remove header underlines
\renewcommand{\footrulewidth}{0pt} % Remove footer underlines
\setlength{\headheight}{13.6pt} % Customize the height of the header

\numberwithin{equation}{section} % Number equations within sections (i.e. 1.1, 1.2, 2.1, 2.2 instead of 1, 2, 3, 4)
\numberwithin{figure}{section} % Number figures within sections (i.e. 1.1, 1.2, 2.1, 2.2 instead of 1, 2, 3, 4)
\numberwithin{table}{section} % Number tables within sections (i.e. 1.1, 1.2, 2.1, 2.2 instead of 1, 2, 3, 4)

\setlength\parindent{0pt} % Removes all indentation from paragraphs - comment this line for an assignment with lots of text

%----------------------------------------------------------------------------------------
%	TITLE SECTION
%----------------------------------------------------------------------------------------
\newcommand{\horrule}[1]{\rule{\linewidth}{#1}} % Create horizontal rule command with 1 argument of height

\title{	
\normalfont \normalsize 
\textsc{Shanghai Jiao Tong University, UM-SJTU JOINT INSTITUTE} \\ [25pt] % Your university, school and/or department name(s)
\horrule{0.5pt} \\[0.4cm] % Thin top horizontal rule
\huge Introduction to Numerical Analysis \\ Project2 \\ % The assignment title
\horrule{2pt} \\[0.5cm] % Thick bottom horizontal rule
}

\author{Yu Cang \\ 018370210001} % Your name

\date{\normalsize \today} % Today's date or a custom date

\begin{document}

\maketitle % Print the title

\section{Task 1}
	Note the symmetry of $x_1$, for any solution $(a, b)$, $(-a, b)$ is also valid.\\
	If no special claim, the norm function for a vector is taken as the 2-norm.
	\begin{enumerate}[(a)]
		\item
			For fixed-point iteration, the iteration function $G$ is taken as
			\begin{equation}
				G \triangleq
				\begin{bmatrix}
					\sqrt{x_2}\\
					\sqrt{1-x_1^2}
				\end{bmatrix}
				=
				\begin{bmatrix}
					x_1\\
					x_2\\
				\end{bmatrix}
				=x
			\end{equation}
			The iteration converges to $(0.7862, 0.6180)$ with intial guess $x_{init} = [\frac{1}{\sqrt{2}}, \frac{1}{\sqrt{2}}]^T$ and tolerance set to $10^{-10}$. 
		\item 
			The newton iteration is implemented as mentioned in the project sheet, each time the linear system
			\begin{equation}
				J_F(x_k) w_k = y_k
			\end{equation}
			is solved with the built-in function 'linsolve' inside matlab.\\
			The iteration converges to $(0.7862, 0.6180)$ with intial guess $x_{init} = [\frac{1}{\sqrt{2}}, \frac{1}{\sqrt{2}}]^T$ and tolerance set to $10^{-10}$. 
		\item
			The input of the Broyden's method is a little bit different from that in newton or fixed-point. The initial Jacobi matrix should be provided, denoted as $A_0$. In my implementation, it is given as
			\begin{equation}
				A_0 = J_F(x_0)
			\end{equation}
			An extra calculation on $x_1$ should be done before the iteration loop as the iteration process requires both $x_0$ and $x_1$. It is approximated through newton's method in my implementation as
			\begin{equation}
				x_1 \approxeq x_0 - A_0^{-1}F(x_0)
			\end{equation}
			After all the preparation work, the iteration loop can be carried out as mentioned in the project sheet, and the key part is the application of Sherman-Morrison's formula to calculated the inverse of $A_k$ iteratively as
			\begin{equation}
				A_k^{-1} = A_{k-1}^{-1} + \frac{(s_k^T A_{k-1}^{-1} y_k)s_k^T A_{k-1}^{-1}}{s_k^T A_{k-1}^{-1} y_k}
			\end{equation}
			where $s_k = x_k - x_{k-1}$ and $y_k = F(x_k) - F(x_{k-1})$. And the iteration of $x_k$ is given as
			\begin{equation}
				x_{k+1} = x_k - A_k^{-1}F(x_k)
			\end{equation}
			The iteration converges to $(0.7862, 0.6180)$ with intial guess $x_{init} = [\frac{1}{\sqrt{2}}, \frac{1}{\sqrt{2}}]^T$ and tolerance set to $10^{-10}$. 
		\item 
			With numerical tests, newton's method out-performs the other two in terms of time cost.
		
	\end{enumerate}

\section{Task 2}
	With trival inequality analysis, the exact solution is given as $x=[0.5, 0, -\frac{\pi}{6}]$, which can be used for varification.
	\subsection{Solve directly}
		Applying the fixed-point iteration, newton's method and the broyden's method to this non-linear system respectively, ...(TODO)
	\subsection{Downhill simplex algorithm}
	In general, the initial value for such a non-linear system is hard to know, even the range for each component is hard to estimate. In this case, the downhill simplex method can be applied to produce a fine inital guess for further calculation like newton or quasi-newton.
	\subsection{Solve with initialization}
	
	Numerical experiments were done to identify the difference in terms of time cost. 
	
	\begin{center}
		\begin{tabular}{ccc}
			\hline
			\quad       & Without initialization & With initialization\\
			\hline
			fixed-point & $4.033 ms$             & $3.970 ms$\\
			neweon      & $4.661 ms$             & $0.679 ms$\\
			broyden     & $4.447 ms$             & $0.799 ms$\\
			\hline
		\end{tabular}
	\end{center}
	
	It can be observed from the table that the downhill simplex initialization process does help a lot to reduce time cost for further newton or broyden calculation as it provide a better initial guess. \\
	However, the time cost is almost equivalent for the fixed-point iteration, which indicates that it does not benefit from the downhill simplex initialization. 

\section{Task 3}
	\begin{enumerate}[(a)]
		\item 
			
		\item 

		\item 

		\item 

		\item 
			
	\end{enumerate}

\end{document}