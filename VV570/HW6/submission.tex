%----------------------------------------------------------------------------------------
%	PACKAGES AND OTHER DOCUMENT CONFIGURATIONS
%----------------------------------------------------------------------------------------

\documentclass[paper=a4, fontsize=11pt]{scrartcl} % A4 paper and 11pt font size

\usepackage[T1]{fontenc} % Use 8-bit encoding that has 256 glyphs
\usepackage{fourier} % Use the Adobe Utopia font for the document - comment this line to return to the LaTeX default
\usepackage[english]{babel} % English language/hyphenation
\usepackage{amsmath,amsfonts,amsthm,amssymb} % Math packages
\usepackage{mathrsfs}

\usepackage{algorithm, algorithmic}
\renewcommand{\algorithmicrequire}{\textbf{Input:}} %Use Input in the format of Algorithm  
\renewcommand{\algorithmicensure}{\textbf{Output:}} %UseOutput in the format of Algorithm  

\usepackage{listings}
\lstset{language=Matlab}

\usepackage{lipsum} % Used for inserting dummy 'Lorem ipsum' text into the template

\usepackage{sectsty} % Allows customizing section commands
\allsectionsfont{\centering \normalfont\scshape} % Make all sections centered, the default font and small caps

\usepackage{fancyhdr} % Custom headers and footers
\pagestyle{fancyplain} % Makes all pages in the document conform to the custom headers and footers
\fancyhead{} % No page header - if you want one, create it in the same way as the footers below
\fancyfoot[L]{} % Empty left footer
\fancyfoot[C]{} % Empty center footer
\fancyfoot[R]{\thepage} % Page numbering for right footer
\renewcommand{\headrulewidth}{0pt} % Remove header underlines
\renewcommand{\footrulewidth}{0pt} % Remove footer underlines
\setlength{\headheight}{13.6pt} % Customize the height of the header

\numberwithin{equation}{section} % Number equations within sections (i.e. 1.1, 1.2, 2.1, 2.2 instead of 1, 2, 3, 4)
\numberwithin{figure}{section} % Number figures within sections (i.e. 1.1, 1.2, 2.1, 2.2 instead of 1, 2, 3, 4)
\numberwithin{table}{section} % Number tables within sections (i.e. 1.1, 1.2, 2.1, 2.2 instead of 1, 2, 3, 4)

\setlength\parindent{0pt} % Removes all indentation from paragraphs - comment this line for an assignment with lots of text

%----------------------------------------------------------------------------------------
%	TITLE SECTION
%----------------------------------------------------------------------------------------
\newcommand{\horrule}[1]{\rule{\linewidth}{#1}} % Create horizontal rule command with 1 argument of height

\title{	
\normalfont \normalsize 
\textsc{Shanghai Jiao Tong University, UM-SJTU JOINT INSTITUTE} \\ [25pt] % Your university, school and/or department name(s)
\horrule{0.5pt} \\[0.4cm] % Thin top horizontal rule
\huge Introduction to Numerical Analysis \\ HW6 \\ % The assignment title
\horrule{2pt} \\[0.5cm] % Thick bottom horizontal rule
}

\author{Yu Cang \\ 018370210001} % Your name

\date{\normalsize \today} % Today's date or a custom date

\begin{document}

\maketitle % Print the title

%----------------------------------------------------------------------------------------
%	PROBLEM 1
%----------------------------------------------------------------------------------------
\section{Richardson Extrapolation}
\begin{enumerate}
	\item 
		\begin{proof}
			When $k = 0$, it's clear that $A_0(t) = a_0 + O(t)$.\\
			Assume that the statement is still valid when $k=n-1$, which means $A_{n-1} = a_0 + O(t^n)$\\
			Thus, when $k = n$, apply the induction hypothesis
			\begin{equation}
				\begin{aligned}
					A_n(t) & = \frac{r^n A_{n-1}(t) - A_{n-1}(rt)}{r^n-1}\\
					          & = \frac{r^n (a_0 + O(t^n)) - (a_0 + r^n O(t^n))}{r^n-1}\\
					          & = a_0 + \frac{r^n}{r^n-1} (O(t^n) - O(t^n))\\
					          & = a_0 + O(t^{n+1})
				\end{aligned}
			\end{equation}
			Hence, the statement is valid.
		\end{proof}
	\item
		\begin{enumerate}
			\item 
				\begin{proof}
					When $t_m = \frac{t_0}{r_0^m}$, and $r_0 > 1$
					\begin{equation}
						\lim\limits_{m\rightarrow\infty} A_n(t_m) = a_0 + O((\frac{t_0}{r_0^m})^{n+1}) = a_0 + \frac{O(t_0 ^ {n+1})}{r_0^{m(n+1)}} = 0
					\end{equation}
					The equation above is valid as $O(t_0 ^{n+1})$ is finite.
			\end{proof}
			
			\item 
				\begin{proof}
					As $O(t_0 ^{n+1})$ is finite, it's clear that
					\begin{equation}
						 A_n(t_m) = a_0 + O((\frac{t_0}{r_0^m})^{n+1}) = a_0 + O(r_0^{-m(n+1)})
					\end{equation}
				\end{proof}
			
		\end{enumerate}
	\item
		Algorithm used to do extrapolation is given as below.
		
		\begin{algorithm}
		\caption{Calculation of the extrapolation matrix and the improved quadrature}
			\begin{algorithmic}[1]
				\REQUIRE $A(t)$, $n$, $m$, $a$, $b$, $r_0$
				\ENSURE $M$, $a_0$
				\STATE $h \gets b-a$
				\STATE $i \gets 0$
				\WHILE {$i \neq m$}
					\STATE $M[i][0] \gets A(\frac{h}{2^i})$
					\STATE $i \gets i + 1$
				\ENDWHILE
				\STATE $j \gets 1$
				\WHILE {$j \neq n$}
					\STATE $i \gets j$
					\WHILE {$i \neq m$}
						\STATE $M[i][j] \gets \frac{r_0^i}{r_0^i - 1} M[i][j-1] - \frac{1}{r_0^i -1} M[i-1][j-1]$
						\STATE $i \gets i+1$
					\ENDWHILE
					\STATE $j \gets j+1$
				\ENDWHILE
				\STATE $a_0 \gets M[m-1][n-1]$
				\RETURN $M$, $a_0$
			\end{algorithmic}
		\end{algorithm}
	
	\item
		\begin{enumerate}
			\item 
				With proper combination, lower-order terms inside original quadratures can be eliminated when generating the new quadrature.
			\item 
				Consider the Runge's Function
				\begin{equation}
					f(x) = \frac{1}{1+25x^2}
				\end{equation}
				(to be added...)
		\end{enumerate}
	
\end{enumerate}


%----------------------------------------------------------------------------------------
%	PROBLEM 2
%----------------------------------------------------------------------------------------
\section{Integration}
\begin{enumerate}
	\item 
		The number of nodes is too small s.t. the order of error is too large.
	\item 
		Denote
		\begin{equation}
			E(f) = \int_{a}^{b} f(x)dx - (b-a)f(\frac{a+b}{2})
		\end{equation}
		Thus
		\begin{equation}
			E(1) = (b-a) - (b-a) = 0
		\end{equation}
		\begin{equation}
			E(x) = \frac{b^2-a^2}{2} - (b-a)\frac{b+a}{2} = 0
		\end{equation}
		and
		\begin{equation}
			E(x^2) = \frac{b^3-a^3}{3} - (b-a)(\frac{a+b}{2})^2 \neq 0
		\end{equation}
		Hence, the degree of this quadrature formula is $1$.\\
		The kernel is determined as
		\begin{equation}
			\begin{aligned}
				K_N(t) & = E(x\in [a, b] \rightarrow [(x-t)_+])\\
					   & = \int_{a}^{b} f(x)dx - (b-a)f(\frac{a+b}{2})\\
					   & = \int_{a}^{b} (x-t)_+dx - (b-a)(\frac{a+b}{2}-t)_+\\
					   & = \frac{(b-t)_+^2 + (a-t)_+^2}{2} - (b-a)(\frac{a+b}{2}-t)_+
			\end{aligned}
		\end{equation} 
		Divide the whole range into 4 sub-gap as $(-\infty, a]$,$(a, \frac{a+b}{2}]$,$(\frac{a+b}{2}, b]$,$(b, +\infty)$.
		The non-negative property is seen when examine each gap.
	\item 
		\begin{proof}
			As has been proved above, $k_N(t)$ is non-negative over $[a, b]$, so the first mean value theorem can be applied
			\begin{equation}
				\begin{aligned}
					E(f) & = \frac{1}{N!} \int_{a}^{b}k_N(t)f''(t)dt\\
					     & = f''(\xi) \int_{a}^{b}k_N(t)dt\\
					     & = \frac{(b-a)^3}{24} f''(\xi)
				\end{aligned}
			\end{equation}
		\end{proof}
\end{enumerate}

	
%----------------------------------------------------------------------------------------
%	PROBLEM 3
%----------------------------------------------------------------------------------------
\section{Gauss's method}
\begin{enumerate}
	\item 
		\begin{enumerate}
			\item 
				\begin{proof}
					It's clear that $w(x)$ is positive.\\
					And
					\begin{equation}
						\int_{-1}^{1} w(x) dx = \int_{-1}^{1} \sqrt{1-x^2}dx = \int_{-\pi/2}^{\pi/2} cos^2\theta d\theta = \frac{\pi}{2}
					\end{equation}
					Thus, $w(x)$ is a weight function.
				\end{proof}
			\item 
				\begin{proof}
					Since
					\begin{equation}
						\begin{aligned}
							\int_{-1}^{1}w(x)q_m(x)q_n(x)dx 
							& = \int_{\pi}^{0} \sqrt{1-cos^2\theta} \frac{sin(m+1)\theta}{sin\theta} \frac{sin(n+1)\theta}{sin\theta}(-sin\theta)d\theta\\
							& = \int_{0}^{\pi} sin(m+1)\theta sin(n+1)\theta d\theta
						\end{aligned}
					\end{equation}
					It's clear that when $n=m$, the value of the equation above is $\frac{\pi}{2}$, and is $0$ when $n \neq m$.
					Thus, $q_k\  (k\in \mathbb{N})$ defines a sequence of orthogonal polynomials for weight function $w(x)$.
				\end{proof}
			\item
				\begin{equation}
					p_k(x) = \sqrt{\frac{2}{\pi}} q_k(x)
				\end{equation} 
		\end{enumerate}
	\item 
		\begin{enumerate}
			\item 
				$x_k$ are roots of orthogonal polynomial over $[a, b]$, whose order is $n+1$ and weight is $w(x)$.
			\item 
				With the Lagrange interpolation function $f(x) = \sum_{i=0}^{n} l_i(x)f(x_i)$, $A_k$ can be determined as
				\begin{equation}
					A_k = \int_{-1}^{1} l_k(x)w(x)dx
				\end{equation}
				As $x_k = cos(\frac{k\pi}{n+2})$, it can be further resolved that $A_k = \frac{\pi}{n+2}sin^2\frac{(k+1)\pi}{n+2}$.
			\item 
				\begin{proof}
					Apply Hermitan interpolation polynomials $H_{2n+1}(x)$ onto $f(x_k)$ s.t. for $k = 0, 1, ... , n$
					\begin{equation}
						H_{2n+1}(x_k) = f(x_k) 
					\end{equation}
					and
					\begin{equation}
						H'_{2n+1}(x_k) = f'(x_k)
					\end{equation}
					Thus
					\begin{equation}
						f(x) = H_{2n+1}(x) + \frac{f^{(2n+2)}(\xi)}{(2n+2)!}\omega^2(x)
					\end{equation}
					Then integrate over $[-1, 1]$ with weight $w(x)$
					\begin{equation}
						I = \int_{-1}^{1} f(x) w(x) dx = \int_{-1}^{1} H_{2n+1}(x) w(x)dx + R_n[f]
					\end{equation}
					Thus
					\begin{equation}
						R_n[f] = I- \sum_{k=0}^{n}A_k f(x_k) = \int_{-1}^{1}\frac{f^{(2n+2)}(\xi)}{(2n+2)!}\omega^2(x)w(x)dx
					\end{equation}
					Since $\omega^2(x)w(x) \geq 0$, it can be concluded from the first intermediate value theorem that
					\begin{equation}
						R_n[f] = \frac{f^{(2n+2)}(\eta)}{(2n+2)!}\int_{-1}^{1}\omega^2(x)w(x)dx \triangleq c  \frac{f^{(2n+2)}(\eta)}{(2n+2)!}
					\end{equation}
					
				\end{proof}
			
		\end{enumerate}
\end{enumerate}

\end{document}