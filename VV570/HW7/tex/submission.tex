%----------------------------------------------------------------------------------------
%	PACKAGES AND OTHER DOCUMENT CONFIGURATIONS
%----------------------------------------------------------------------------------------
\documentclass[paper=a4, fontsize=11pt]{scrartcl} % A4 paper and 11pt font size

\usepackage[T1]{fontenc} % Use 8-bit encoding that has 256 glyphs
\usepackage{fourier} % Use the Adobe Utopia font for the document - comment this line to return to the LaTeX default
\usepackage[english]{babel} % English language/hyphenation
\usepackage{amsmath,amsfonts,amsthm,amssymb} % Math packages
\usepackage{mathrsfs}
\usepackage{algorithm, algorithmic}
\renewcommand{\algorithmicrequire}{\textbf{Input:}} %Use Input in the format of Algorithm  
\renewcommand{\algorithmicensure}{\textbf{Output:}} %UseOutput in the format of Algorithm  
\usepackage{listings}
\lstset{language=Matlab}
\usepackage{enumerate}

\usepackage{lipsum} % Used for inserting dummy 'Lorem ipsum' text into the template
\usepackage{sectsty} % Allows customizing section commands
\allsectionsfont{\centering \normalfont\scshape} % Make all sections centered, the default font and small caps
\usepackage{fancyhdr} % Custom headers and footers
\pagestyle{fancyplain} % Makes all pages in the document conform to the custom headers and footers
\fancyhead{} % No page header - if you want one, create it in the same way as the footers below
\fancyfoot[L]{} % Empty left footer
\fancyfoot[C]{} % Empty center footer
\fancyfoot[R]{\thepage} % Page numbering for right footer
\renewcommand{\headrulewidth}{0pt} % Remove header underlines
\renewcommand{\footrulewidth}{0pt} % Remove footer underlines
\setlength{\headheight}{13.6pt} % Customize the height of the header

\numberwithin{equation}{section} % Number equations within sections (i.e. 1.1, 1.2, 2.1, 2.2 instead of 1, 2, 3, 4)
\numberwithin{figure}{section} % Number figures within sections (i.e. 1.1, 1.2, 2.1, 2.2 instead of 1, 2, 3, 4)
\numberwithin{table}{section} % Number tables within sections (i.e. 1.1, 1.2, 2.1, 2.2 instead of 1, 2, 3, 4)

\setlength\parindent{0pt} % Removes all indentation from paragraphs - comment this line for an assignment with lots of text

%----------------------------------------------------------------------------------------
%	TITLE SECTION
%----------------------------------------------------------------------------------------
\newcommand{\horrule}[1]{\rule{\linewidth}{#1}} % Create horizontal rule command with 1 argument of height

\title{	
\normalfont \normalsize 
\textsc{Shanghai Jiao Tong University, UM-SJTU JOINT INSTITUTE} \\ [25pt] % Your university, school and/or department name(s)
\horrule{0.5pt} \\[0.4cm] % Thin top horizontal rule
\huge Introduction to Numerical Analysis \\ HW7 \\ % The assignment title
\horrule{2pt} \\[0.5cm] % Thick bottom horizontal rule
}

\author{Yu Cang \\ 018370210001} % Your name

\date{\normalsize \today} % Today's date or a custom date

\begin{document}

\maketitle % Print the title

%----------------------------------------------------------------------------------------
%	PROBLEM 1
%----------------------------------------------------------------------------------------
\section{Question 1}
	\begin{enumerate}[(a)]
		\item 
			For example
			\begin{equation}
				y = tan(x), \ \ x\in (-\frac{\pi}{2}, \frac{\pi}{2})
			\end{equation}
			It's differentiable over $(-\frac{\pi}{2}, \frac{\pi}{2})$, and the derivative is 
			\begin{equation}
				y' = \frac{1}{cos^2(x)}
			\end{equation}
			It's obvious that $y' \rightarrow \infty$ when $x \rightarrow \frac{\pi}{2}$.
		\item 
			\begin{proof}
				Denote $g(x)$ over $[x_1, x_2]$ as
				\begin{equation}
					g(x) = f(x) - f(x_1) - \frac{f(x_2)-f(x_1)}{x_2-x_1}(x-x_1) 
				\end{equation}
				where $a< x_1 < x_2 < b$. Then, $g(x_1) = 0$ and $g(x_2)=0$.
				
				Thus, from Rolle's theorem, there exists $\xi \in (x_1, x_2)$ s.t. 
				\begin{equation}
					g'(\xi)= 0
				\end{equation}
				namely
				\begin{equation}
					\frac{f(x_2) - f(x_1)}{x_2-x_1} = f'(\xi)
				\end{equation}
				Since $f'$ is bounded, then $|f'(\xi)| \leq M$. Thus
				\begin{equation}
					|f(x_2) - f(x_1)| \leq M |x_2 - x_1|
				\end{equation}
				which means that $f(x)$ is Lipschitz continous.
			\end{proof}
		\item 
			For example
			\begin{equation}
				y = \frac{1}{x}, \ \ x \in (0, 1)
			\end{equation}
			$y$ is obviously differentiable and its derivative is
			\begin{equation}
				y' = -\frac{1}{x^2}, \ \ x \in (0, 1)
			\end{equation}
			Suppose there exist a constant $c>0$ s.t. 
			\begin{equation}
				|y_2 - y_1| \leq c|x_2 - x_1|
			\end{equation}
			is valid for all $0 < x_1 < x_2 < 1$.
			Let $y(b) - y(a) = c(b-a)$, where $a < b$then
			\begin{equation}
				c = \frac{1}{ab}
			\end{equation}
			Take the mid-point of $a, b$, then
			\begin{equation}
				\frac{y(a)-y(\frac{b+a}{2})}{\frac{b+a}{2}-a} = \frac{1}{a(\frac{b+a}{2})} > \frac{1}{ab} = c
			\end{equation}
			Thus, the assumption fails, which means that $y$ is not Lipschitz continous.
		\item 
			For example
			\begin{equation}
				y = |x|, \ \ x \in (-1, 1)
			\end{equation}
			It's Lipschitz continous as for any $-1 < x_1 < x_2 < 1$
			\begin{equation}
				\frac{|y(x_2)-y(x_1)|}{|x_2 - x_1|} \leq 1
			\end{equation}
			but it is not differentiable at $x=0$.
		
	\end{enumerate}

%----------------------------------------------------------------------------------------
%	PROBLEM 2
%----------------------------------------------------------------------------------------
\section{Question 2}
	\begin{enumerate}
		\item 
			\begin{proof}
				Since $|g'(x)|<1$, there exists $0<L<1$ s.t. $|g'(x)| \leq L$.
				Then, according to the Lagrange intermediate value theorem
				\begin{equation}
					\begin{aligned}
						|g(x_j) - g(x_i)| &= |g'(\xi)||x_j - x_i|\\
										  &\leq L|x_j - x_i|
					\end{aligned}
				\end{equation}
				As $g(x^*) = x^*$, applying the fixed point iteration, then
				\begin{equation}
					\begin{aligned}
						|x_{k+1} - x^*| & = |g(x_k) - g(x^*)|\\
										& \leq L|x_k - x^*|\\
										& = L |(x_{k+1}-x^*) - (x_{k+1} - x_k)|\\
										& \leq L|x_{k+1}-x^*| + L|x_{k+1} - x_k|
					\end{aligned}
				\end{equation}
				Thus
				\begin{equation}
					\begin{aligned}
						|x_{k+1} - x^*| & \leq \frac{L}{1-L} |x_{k+1} - x_k|\\
						                & \leq \frac{L^2}{1-L}|x_k - x_{k-1}|\\
						                & \leq ...\\
						                & \leq \frac{L^{k+1}}{1-L} |x_1 - x_0|
					\end{aligned}
				\end{equation}
				Since $0<L<1$, and $|x_1 - x_0|$ is finite, it indicates that
				\begin{equation}
					\lim\limits_{k\rightarrow\infty}|x_{k+1} - x^*| = 0
				\end{equation}
				Hence, the fixed-point iteration will converge to the unique fixed point $x^*$.
			\end{proof}

		\item 
			\begin{proof}
				Since $|g'(x)|>1$, there exists $L>1$ s.t. $|g'(x)| \geq L$.
				Then, according to the Lagrange intermediate value theorem
				\begin{equation}
					\begin{aligned}
						|g(x_j) - g(x_i)| &= |g'(\xi)||x_j - x_i|\\
										  &\geq L|x_j - x_i|
					\end{aligned}
				\end{equation}
				As $g(x^*) = x^*$, applying the fixed point iteration, then
				\begin{equation}
					\begin{aligned}
					|x_{k+1} - x^*| & = |g(x_k) - g(x^*)|\\
									& \geq L|x_k - x^*|\\
									& = L |(x_{k+1} - x_k)-(x_{k+1}-x^*)|\\
									& \geq L(|x_{k+1} - x_k|-|x_{k+1}-x^*|)\\
					\end{aligned}
				\end{equation}
				Thus
				\begin{equation}
					\begin{aligned}
						|x_{k+1} - x^*| & \geq \frac{L}{1+L} |x_{k+1} - x_k|\\
										& \geq \frac{L^2}{1+L}|x_k - x_{k-1}|\\
										& \geq ...\\
										& \geq \frac{L^{k+1}}{1+L} |x_1 - x_0|
					\end{aligned}
				\end{equation}
						Since $L>1$, and $|x_1 - x_0|$ is finite, it indicates that
				\begin{equation}
				\lim\limits_{k\rightarrow\infty}|x_{k+1} - x^*| = \infty
				\end{equation}
				Hence, the fixed-point iteration will never converge to the unique fixed point $x^*$.
			\end{proof}

		
	\end{enumerate}
%----------------------------------------------------------------------------------------
%	PROBLEM 3
%----------------------------------------------------------------------------------------
\section{Question 3}
\begin{enumerate}
	\item 
		For resolving the smallest positive root, the equation can be reformed as
		\begin{equation}
			x = tg^{-1}(4x) \triangleq g(x)
		\end{equation}
		It can be easily observed that the root lies in $(1, \frac{\pi}{2})$. 
		
		Applying the fundamental inequality, $g'(x)$ can be further determined as
		\begin{equation}
			0 < g'(x) = \frac{4}{1+(4x)^2} = \frac{1}{\frac{1}{4}+ 4x^2} \leq \frac{1}{x} < 1
		\end{equation}
		Thus, the fixed-point iteration can be employed to resolve the result. The starting value is given as
		\begin{equation}
			x_0 = \frac{1+\frac{\pi}{2}}{2}
		\end{equation}
		A python script is written to perform the numerical iteration, and the final result is
		\begin{equation}
			x^* \approx 1.393
		\end{equation}
	\item 
		For resolving the second smallest positive root, the original coordinate axes $Oxy$ can be transfromed to $O'xy$ to simplify to equation, where $(\pi, 0)$ in origial coordinate is selected as the transformed origin. Thus, it's equivalent to solve
		\begin{equation}
			tg(x) = 4(x+\pi)
		\end{equation}
		For similar consideration, the equation is reformed as
		\begin{equation}
			x = tg^{-1}(4(x+\pi)) \triangleq g(x)
		\end{equation}
		Obviously, the root is located in $(1, \frac{\pi}{2})$ in the transformed coordinate.Similarly, the range of $g'(x)$ is examined as
		\begin{equation}
			0<g'(x)=\frac{4}{1+16(x+\pi)^2} = \frac{1}{\frac{1}{4} + 4(x+\pi)^2} <\frac{1}{\frac{1}{4}+4x^2} <\frac{1}{x} < 1
		\end{equation}
		Thus, the fixed-point iteration can be employed to resolve the result. The starting value is given as
		\begin{equation}
			x'_0 = \frac{1+\frac{\pi}{2}}{2}
		\end{equation}
		A python script is written to perform the numerical iteration, and the final result is
		\begin{equation}
			x^* = x'^* + \pi \approx 1.517 + \pi \approx 4.659 
		\end{equation}
\end{enumerate}
%----------------------------------------------------------------------------------------
%	PROBLEM 6
%----------------------------------------------------------------------------------------
\section{Question 6}
	\begin{enumerate}
		\item 
			For example, $e^x = x+1$ over $[-1, 1]$ can be solved by bisection, but $g'(x) \triangleq (e^x-1)' = e^x > 1$ when $x>0$. Thus the fixed-point iteration will diverge if the initial value is given as $x_0 > 0$.
		\item 
			For example, $cos(x) = 0$ over $[0, \frac{\pi}{2}]$. Let $g'(x)\triangleq (cos(x)+x)' = 1-sin(x)$, it's clear that $0 \leq g'(x) \leq 1$, thus, it can be resolved by the fixed-point iteration given an starting value $x_0 \in (0, \frac{\pi}{2})$ . However, the Newton method would fail if $x_0$ is very close to $0$. This is caused as $cos'(x) = sin(x) \rightarrow 0$, which lead to the further displacement of $x_1$ as $\frac{f(x_0)}{f'(x_0)}\rightarrow\infty$.
		
	\end{enumerate}

\end{document}