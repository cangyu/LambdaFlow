%----------------------------------------------------------------------------------------
%	PACKAGES AND OTHER DOCUMENT CONFIGURATIONS
%----------------------------------------------------------------------------------------

\documentclass[paper=a4, fontsize=11pt]{scrartcl} % A4 paper and 11pt font size

\usepackage[T1]{fontenc} % Use 8-bit encoding that has 256 glyphs
\usepackage{fourier} % Use the Adobe Utopia font for the document - comment this line to return to the LaTeX default
\usepackage[english]{babel} % English language/hyphenation
\usepackage{amsmath,amsfonts,amsthm,amssymb} % Math packages
\usepackage{mathrsfs}

\usepackage{algorithm, algorithmic}
\renewcommand{\algorithmicrequire}{\textbf{Input:}} %Use Input in the format of Algorithm  
\renewcommand{\algorithmicensure}{\textbf{Output:}} %UseOutput in the format of Algorithm  

\usepackage{listings}
\lstset{language=Matlab}

\usepackage{lipsum} % Used for inserting dummy 'Lorem ipsum' text into the template

\usepackage{sectsty} % Allows customizing section commands
\allsectionsfont{\centering \normalfont\scshape} % Make all sections centered, the default font and small caps

\usepackage{fancyhdr} % Custom headers and footers
\pagestyle{fancyplain} % Makes all pages in the document conform to the custom headers and footers
\fancyhead{} % No page header - if you want one, create it in the same way as the footers below
\fancyfoot[L]{} % Empty left footer
\fancyfoot[C]{} % Empty center footer
\fancyfoot[R]{\thepage} % Page numbering for right footer
\renewcommand{\headrulewidth}{0pt} % Remove header underlines
\renewcommand{\footrulewidth}{0pt} % Remove footer underlines
\setlength{\headheight}{13.6pt} % Customize the height of the header

\numberwithin{equation}{section} % Number equations within sections (i.e. 1.1, 1.2, 2.1, 2.2 instead of 1, 2, 3, 4)
\numberwithin{figure}{section} % Number figures within sections (i.e. 1.1, 1.2, 2.1, 2.2 instead of 1, 2, 3, 4)
\numberwithin{table}{section} % Number tables within sections (i.e. 1.1, 1.2, 2.1, 2.2 instead of 1, 2, 3, 4)

\setlength\parindent{0pt} % Removes all indentation from paragraphs - comment this line for an assignment with lots of text

%----------------------------------------------------------------------------------------
%	TITLE SECTION
%----------------------------------------------------------------------------------------

\newcommand{\horrule}[1]{\rule{\linewidth}{#1}} % Create horizontal rule command with 1 argument of height

\title{	
\normalfont \normalsize 
\textsc{Shanghai Jiao Tong University, UM-SJTU JOINT INSTITUTE} \\ [25pt] % Your university, school and/or department name(s)
\horrule{0.5pt} \\[0.4cm] % Thin top horizontal rule
\huge Introduction to Numerical Analysis \\ HW2 \\ % The assignment title
\horrule{2pt} \\[0.5cm] % Thick bottom horizontal rule
}

\author{Yu Cang \\ 018370210001} % Your name

\date{\normalsize \today} % Today's date or a custom date

\begin{document}

\maketitle % Print the title

%----------------------------------------------------------------------------------------
%	PROBLEM 1
%----------------------------------------------------------------------------------------

\section{Connected Space}
\begin{enumerate}
	\item
		\begin{proof}
			\begin{enumerate}
				\item (i) $\Rightarrow$ (ii) \\
					Suppose (ii) is not true, which means $X = U_1 \cup U_2$, $U_1 \cap U_2 = \emptyset$, $U_1 \neq \emptyset$, $U_2 \neq \emptyset$, and both $U_1$ and $U_2$ are open.\\
					Thus, $U_1$ and $U_2$ are closed as $U_1 = U_2^\complement$ and $U_2 = U_1^\complement$.\\
					So, $U_1$ and $U_2$ are both open and closed in $X$, which is contradictory to (i).\\
					Thus the assumption fails and (ii) is true when (i) is true.
				\item (ii) $\Rightarrow$ (i)\\
					Suppose (i) is not true, which means there exists $U$ s.t. $U \subset X$, $U \neq \emptyset$ and $U$ is both open and closed in $X$.\\
					Thus, $U^\complement$ is open as $U$ is closed.\\
					As $X = U \cup U^\complement$, then $X$ can be written as the union of two disjoint, non-empty open subsets, which is contradictory to (ii).\\
					Thus the assumption fails and (i) holds true when (ii) is true. 
				\item (i)$\Rightarrow$(iii)\\
					Suppose (iii) is not true, which means $X = U_1 \cup U_2$, $U_1 \cap U_2 = \emptyset$, $U_1 \neq \emptyset$, $U_2 \neq \emptyset$ and both $U_1$ and $U_2$ are closed.\\
					Thus, $U_1$ and $U_2$ are open as $U_1 = U_2^\complement$ and $U_2 = U_1^\complement$.\\
					So, $U_1$ and $U_2$ are both open and closed in $X$, which is contradictory to (i).\\
					Thus the assumption fails and (iii) is true when (i) is true.
				\item (iii) $\Rightarrow$ (i)\\
					Suppose (i) is not true, which means there exists $U$ s.t. $U \subset X$, $U \neq \emptyset$ and $U$ is both open and closed in $X$.\\
					Thus, $U^\complement$ is closed as $U$ is open.\\
					As $X = U \cup U^\complement$, then $X$ can be written as the union of two disjoint, non-empty closed subsets, which is contradictory to (iii).\\
					Thus the assumption fails and (i) holds true when (iii) is true. 
			\end{enumerate}
		\end{proof}
	\item
		\begin{proof}
			If (iv) is false, then there exists a continuous, surjective application from $X$ into $[0, 1]\subset U$, which can be denoted as $f$.\\
			$[0, 1]$ can be written as $[0, a) \cup [a, 1] \triangleq V_1 \cup V_2$, where $0 < a < 1$, $V_1$ and $V_2$ are closed. Denote $U_1 = f^{-1}(V_1)$ and  $U_2 = f^{-1}(V_2)$. \\
			As $f$ is surjective, it follows that $U_1 \neq \emptyset$, $U_2 \neq \emptyset$ and $U_1 \cap U_2 = \emptyset$. \\
			As $f$ is continous, it follows that $U_1$ and $U_2$ are also closed, $U_1 \cap U_2 = X$.\\
			Thus, it is contradictory to (iii) as $X$ can be written as the union of two disjoint, non-empty closed subsets.\\
			So, if (iv) is not true then (iii) is also false.
		\end{proof}
	\item 
		\begin{proof}
			If (iii) is false, then $X = U_1 \cup U_2$, where $U_1$ and $U_2$ are two disjoint, non-empty closed subsets.\\
			(haven't figured out yet...)
		\end{proof}
\end{enumerate}


%----------------------------------------------------------------------------------------
%	PROBLEM 2
%----------------------------------------------------------------------------------------

\section{Intermediate value theorem}

\begin{enumerate}
	\item
      \begin{proof}
      	Suppose $f(A) = V_1 \cup V_2$, where $V_1$ and $V_2$ are two disjoint, non-empty open subsets.
      	Denote $U_1 = f^{-1}(V_1)$, $U_2 = f^{-1}(V_2)$. $A = U_1 \cup U_2$ as each element in $A$ is mapped to either $V_1$ or $V_2$.
      	Further, $U_1$ and $U_2$ are open as $f$ is a continous map. 
      	Thus $A$ can be written as the union of two disjoint, non-empty open subsets, which is contradictory to the fact that $A$ is a connected space. Therefore, $f(A)$ is connected.
      \end{proof}
  
	\item
	  \begin{proof}
		\begin{enumerate}
			\item
				It's clear that $\emptyset$ is connected as $X$ is itself.\\
				For $A$ containing only 1 element, it is connected as it can no be written as the union of two disjoint non-empty closed subsets.
			\item
				If $A$ is not an interval and the corner cases in a) are excluded, then it can be written as union of non-empty, disjoint closed subsets. Thus $A$ is not connected.
			\item
				\begin{enumerate}
					\item 
						The continuous bijection mapping $f$ can be given as 
						\begin{equation}
							f(x) = \frac{x-I_{min}}{I_{max} - I_{min}} (J_{max} - J{min})
						\end{equation} 
						where $I_{min}, I_{max}, J_{max}, J{min}$ are the limits of corresponding interval.\\
						The inverse continuous bijection can be constructed in the same way, which can be given as
						\begin{equation}
							f^{-1}(y) = \frac{y-J_{min}}{J_{max} - J_{min}} (I_{max} - I_{min})
						\end{equation}
					\item 
						Consider open interval $X = (0, 1)$, it is clear that $X$ is connected. As there exists a continous bijection mapping which maps $X$ to $\mathbb{R}$, thus $f(A) = \mathbb{R}$ is connected as well.
					\item 
						If $U$ is both open and closed, then $U^\complement$ is also both open and closed. As there must exist a minimum for a closed and non-empty set, $R = U \cup U^\complement$ is then bounded, which is false. Thus, the only subsets that are both open and closed in $\mathbb{R}$ are $\mathbb{R}$ and $\emptyset$, which is equivalent to say $\mathbb{R}$ is connected. 
				\end{enumerate}
		\end{enumerate}
	  \end{proof}
	\item 
		\begin{proof}
			For any connected set $A$, as is indicated above, $f(A)$ is also connected, where $f$ is a continuous function.\\
			And the connected subsets of $\mathbb{R}$ are all intervals.\\
			Then $f(X)$ is an interval of $\mathbb{R}$, which contains both $f(a)$ and $f(b)$.\\
			Thus, $f(X)$ contains both $f(a)$ and $f(b)$.
		\end{proof}
\end{enumerate}


%----------------------------------------------------------------------------------------
%	PROBLEM 3
%----------------------------------------------------------------------------------------

\section{Rolle's theorem}
	\begin{proof}
		\begin{enumerate}
			\item 
				For $n=1$, if $f(x)$ has 2 distinct roots in $[a, b]$, then there exists the maximum $M$ and minimum $m$ between $[a, b]$ according to the extream value theorem.\\
				If $M = m$, then $f(x)$ is constant, and it's obvious that for any $c \in [a, b]$, $f\prime(c) = 0$; \\
				If $M \neq m$, then $\exists \xi \in (a, b)$, s.t.  $f(\xi)$ reaches its extream, and equals to 0.
			\item
				As induction hypothesis, assume the statement is true for $n = k$.
			\item 
				For $n = k+1$, where $f(x)$ has $k+2$ distinct roots denoted as $c_0 < c_1 < ... <c_k < c_{k+1}$, applying the results for $n = 1$ on each gap $[c_i, c_{i+1}]\ (i = 0, 1, ... , k)$, then $g(x) \triangleq f'(x)$ has $k+1$ roots in $[c_0, c_{k+1}]$. By induction hypothesis, there exists $c \in [c_0, c_{k+1}]$ s.t. $g^{(k)}(c) = f^{(k+1)}(c) = f^{(n)}(c)= 0$. Thus the statement holds true for $n = k+1$.
		\end{enumerate}
	\end{proof}


%----------------------------------------------------------------------------------------
%	PROBLEM 4
%----------------------------------------------------------------------------------------

\section{Extreme value theorem}

\begin{enumerate}
	\item
		\begin{proof}
			
		\end{proof}

	\item
		\begin{proof}
			\begin{enumerate}
				\item
					Given an open covering $\mathscr{U}$ of $A$, an open covering of $X$ by throwing in the open subset $U_0 = X /\ A$. Since $X$ is compact, there exists finitely many sets $U_1, U_2, U_3, ... , U_n \in \mathscr{U}$ \ s.t. $X = U_0 \cup U_1 \cup ... \cup U_n$. Then $A \subseteq  U_1 \cup ... \cup U_n$, proving that $A$ is compact.
				\item
					(haven't figured out yet...)
			\end{enumerate}
		\end{proof}
	
	\item
		\begin{proof}
			
		\end{proof}
\end{enumerate}

%----------------------------------------------------------------------------------------
%	PROBLEM 5
%----------------------------------------------------------------------------------------

\section{Continuity}
	\begin{enumerate}
		\item
			\begin{proof}
				(i)$\Rightarrow$(ii): For each $ y \in B(f(a), \xi)$, there exists $U_x \subset X, U_x \neq \emptyset$ s.t. $y = f(U_x)$. Thus, $d(f(x), f(a)) < \xi$ is valid for any $x \in U \triangleq \bigcup\limits_{x \in X} U_x$. As indicated by (i), there exists $\eta$ s.t. $B(a, \eta) \subset U$. Thus, $\eta$ is valid, and $d(a, x)$ for $x \in B(a, \eta)$ is less than $\eta$.\\
				(ii)$\Rightarrow$(i): As $X$ and $Y$ are two metrix spaces, then the set containing all the elements in $d(x, a) < \eta$ is equivalent to the ball $B(a, \eta) \subset X$. It suffices to show that the $\eta$ in (i) exists.
			\end{proof}
		\item
			\begin{proof}
				Given $\xi$ where $B(f(a), \xi) \subset V$, then it is indicated by (i) that there exists $\eta$ where $f(B(a, \eta)) \in B(f(a), \xi)$. Denote $U = B(a, \eta)$, then $f(U) \subset B(f(a), \xi) \subset V$.
			\end{proof}
		\item
			\begin{proof}
				As indicated by (iii), $U$ is a neighborhood of $a$ and $f(U) \subset V$. Since $U \subset f^{-1}(V)$, thus, by observation, $f^{-1}(V)$ is a neighborhood of $a$. 
			\end{proof}
		\item
			\begin{proof}
				For any $\xi \in \mathbb{R}_+$, take the neighborhood $V$ of $f(a)$ \ s.t. $V \subset B(f(a), \xi)$. Then, by (iv), $f^{-1}(V)$ is a neighborhood of $a$. Thus, there exists $\eta \in \mathbb{R}_+$ s.t. $B(a, \eta) \subset f^{-1}(V)$, and it is obvious that $f(B(a, \eta)) \subset B(f(a), \xi)$.
			\end{proof}
	\end{enumerate}

\end{document}