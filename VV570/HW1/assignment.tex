%----------------------------------------------------------------------------------------
%	PACKAGES AND OTHER DOCUMENT CONFIGURATIONS
%----------------------------------------------------------------------------------------

\documentclass[paper=a4, fontsize=11pt]{scrartcl} % A4 paper and 11pt font size

\usepackage[T1]{fontenc} % Use 8-bit encoding that has 256 glyphs
\usepackage{fourier} % Use the Adobe Utopia font for the document - comment this line to return to the LaTeX default
\usepackage[english]{babel} % English language/hyphenation
\usepackage{amsmath,amsfonts,amsthm,amssymb} % Math packages

\usepackage{lipsum} % Used for inserting dummy 'Lorem ipsum' text into the template

\usepackage{sectsty} % Allows customizing section commands
\allsectionsfont{\centering \normalfont\scshape} % Make all sections centered, the default font and small caps

\usepackage{fancyhdr} % Custom headers and footers
\pagestyle{fancyplain} % Makes all pages in the document conform to the custom headers and footers
\fancyhead{} % No page header - if you want one, create it in the same way as the footers below
\fancyfoot[L]{} % Empty left footer
\fancyfoot[C]{} % Empty center footer
\fancyfoot[R]{\thepage} % Page numbering for right footer
\renewcommand{\headrulewidth}{0pt} % Remove header underlines
\renewcommand{\footrulewidth}{0pt} % Remove footer underlines
\setlength{\headheight}{13.6pt} % Customize the height of the header

\numberwithin{equation}{section} % Number equations within sections (i.e. 1.1, 1.2, 2.1, 2.2 instead of 1, 2, 3, 4)
\numberwithin{figure}{section} % Number figures within sections (i.e. 1.1, 1.2, 2.1, 2.2 instead of 1, 2, 3, 4)
\numberwithin{table}{section} % Number tables within sections (i.e. 1.1, 1.2, 2.1, 2.2 instead of 1, 2, 3, 4)

\setlength\parindent{0pt} % Removes all indentation from paragraphs - comment this line for an assignment with lots of text

%----------------------------------------------------------------------------------------
%	TITLE SECTION
%----------------------------------------------------------------------------------------

\newcommand{\horrule}[1]{\rule{\linewidth}{#1}} % Create horizontal rule command with 1 argument of height

\title{	
\normalfont \normalsize 
\textsc{Shanghai Jiao Tong University, UM-SJTU JOINT INSTITUTE} \\ [25pt] % Your university, school and/or department name(s)
\horrule{0.5pt} \\[0.4cm] % Thin top horizontal rule
\huge Introduction to Numerical Analysis \\ HW1 \\ % The assignment title
\horrule{2pt} \\[0.5cm] % Thick bottom horizontal rule
}

\author{Yu Cang \\ 018370210001} % Your name

\date{\normalsize\today} % Today's date or a custom date

\begin{document}

\maketitle % Print the title

%----------------------------------------------------------------------------------------
%	PROBLEM 1
%----------------------------------------------------------------------------------------

\section{Metric Space}


%----------------------------------------------------------------------------------------
%	PROBLEM 2
%----------------------------------------------------------------------------------------

\section{Continuity}

\begin{enumerate}
\item $y=\frac{1}{x}$ is continuous over $(0, +\infty)$, but not uniform continuous.
      \begin{proof}
      	Let $x_1 = \frac{1}{n+1}$,  $x_2 = \frac{1}{n}$. 
      	Then $\lim\limits_{n \rightarrow +\infty} x_2-x_1 = \frac{1}{n(n+1)} = 0 $.
      	But $\lim\limits_{n \rightarrow +\infty} y_2- y_1 = -1 \neq 0 $.
      	Thus it is not uniform continuous.
      \end{proof}
\item $y=\sqrt{x}$ is uniform continuous over $(0, +\infty)$ but not Lipschitz continuous.
	  \begin{proof}
	  	As $y\prime = \frac{1}{2\sqrt{x}}$, 
	  	the slope tends towards infty when $x$ approaches 0, 
	  	thus it is not Lipschitz continuous.  
	  \end{proof}
\end{enumerate}


%----------------------------------------------------------------------------------------
%	PROBLEM 3
%----------------------------------------------------------------------------------------

\section{Cardinalty}
\begin{enumerate}
	\item 
		\begin{proof}
			The following function is a one-to-one mapping from N to Z, thus N and Z have the same number of elements.
		
			\begin{equation}
			f(n) = 
				\begin{cases}
					-\frac{n}{2} & \text{n is even}\\
					\frac{n+1}{2} & \text{n is odd}
				\end{cases}
			\end{equation}
			
			Arange the elements in N and Q in the pascal triangle style as below, and it is clear enough to see the one-to-one mapping from N to Q with rationals are grouped according to the sum of dividend and divisor. Thus N and Q have the same number of elements.
			
			\begin{tabular}{c|c}
				\begin{tabular}{rccccccccc}
					&&&&&1\\\noalign{\smallskip\smallskip}
					&&&&2&&3\\\noalign{\smallskip\smallskip}
					&&&4&&&&5\\\noalign{\smallskip\smallskip}
					&&6&&7&&8&&9\\\noalign{\smallskip\smallskip}
					&10&&&&&&&&11\\\noalign{\smallskip\smallskip}
					&&&&&...\\\noalign{\smallskip\smallskip}
				\end{tabular} &
				\begin{tabular}{rccccccccc}
					&&&&&$\frac{1}{1}$ \\\noalign{\smallskip\smallskip}
					&&&&$\frac{1}{2}$&&$\frac{2}{1}$\\\noalign{\smallskip\smallskip}
					&&&$\frac{1}{3}$&&&&$\frac{3}{1}$\\\noalign{\smallskip\smallskip}
					&&$\frac{1}{4}$&&$\frac{2}{3}$&&$\frac{3}{2}$&&$\frac{4}{1}$\\\noalign{\smallskip\smallskip}
					&$\frac{1}{5}$&&&&&&&&$\frac{5}{1}$\\\noalign{\smallskip\smallskip}
					&&&&&...\\\noalign{\smallskip\smallskip}
				\end{tabular}
			\end{tabular}
			
		\end{proof}	
	\item 
		\begin{proof}
			The following function is a one-to-one mapping from [0, 1] to R, thus they have the same number of elements.	
			
			\begin{equation}
				f(x) = 
				\begin{cases}
				-\infty& \text{x=0}\\
				tan[\pi(x-0.5)] & \text{0 < x < 1}\\
				+\infty& \text{x=1}
				\end{cases}
			\end{equation}
			
		\end{proof}
	\item
		\begin{proof}
			Suppose all the real numbers can be listed, and each one is marked as $r_i$, where $i=1,2,3...$. Given a real number $r\prime$ such that its \textbf{i-th} digit is different from that in $r_i$, it's obvious that $r\prime$ is not included in the real numbers listed above, which is contradictory to the hypothesis. Thus real numbers can not be listed and it contains more elements than N.
		\end{proof}
\end{enumerate}

%----------------------------------------------------------------------------------------
%	PROBLEM 4
%----------------------------------------------------------------------------------------

\section{Slides}

\begin{enumerate}
	\item
		\begin{proof}
			Let $E$ be an inner product space over $\mathbb{C}$, and $u, v\in E$.
			Given $Y$ defined as below, where $\lambda \in \mathbb{R}$.
						
			\begin{equation}
				Y = {\vert u-\lambda v \vert}^2
			\end{equation}
			
			It's obvious that $Y\geq0$. Expand the squares according to the definition of inner products, the inequality can be written as below.
			
			\begin{equation}
				<v, v>\lambda^2 -(<u, v>+<v, u>)\lambda + <u, u> \ \ \geq \ 0
			\end{equation}
			
			The LHS of the inequality can be viewed as a quadratic function where $\lambda$ is the variable. Thus, the discriminant is semi-negative definite, which can be written as below.
			
			\begin{equation}
				\Delta = (<u, v>+<v, u>)^2 - 4<u,u><v,v> \ \ \leq \ 0
			\end{equation}
			
			As $<v, u> = \overline{<u, v>}$, the inequality above can be simplified as below.
			
			\begin{equation}
				<u, v>^2 + {\overline{<u, v>}}^2 + 2{\vert <u,v> \vert}^2 \ \ \leq \ 4 \lVert u \rVert^2 \lVert v \rVert^2 \label{cs}			
			\end{equation}
			
			For $x \in \mathbb{C}$, the following equality is justified.
			\begin{equation}
				x^2 + \bar{x}^2 = \frac{1}{2}[(x+\bar{x})^2+(x-\bar{x})^2] = \frac{1}{2}[(2Re(x))^2+(2Im(x))^2] = 2 {\vert x \vert}^2 \label{der}
			\end{equation}
			
			Since $<u, v>$ is an complex number, apply (\ref{der}) into (\ref{cs}) and eliminate the constant 4, the Cauchy-Schwarz inequality is obtained at last.
			
			\begin{equation}
				{\vert <u,v> \vert}^2 \leq \lVert u \rVert^2 \lVert v \rVert^2
			\end{equation}
  
		\end{proof}

	\item
		\begin{proof}
			$d(x, y)$ is non-negative follows from the definition of metric space.
			\begin{equation}
				\begin{aligned}
					d(x, y) & = \frac{1}{2}(d(x, y) + d(x, y)) \\
							& = \frac{1}{2}(d(x, y) + d(y, x))  &\text{By symmetry}\\
							& \geq  \frac{1}{2} d(x, x)  &\text{By triangle inequality} \\
							&  = 0  &\text{By identity of indiscernible}
				\end{aligned}
			\end{equation}
			

		\end{proof}
\end{enumerate}



%----------------------------------------------------------------------------------------

\end{document}