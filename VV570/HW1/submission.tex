%----------------------------------------------------------------------------------------
%	PACKAGES AND OTHER DOCUMENT CONFIGURATIONS
%----------------------------------------------------------------------------------------

\documentclass[paper=a4, fontsize=11pt]{scrartcl} % A4 paper and 11pt font size

\usepackage[T1]{fontenc} % Use 8-bit encoding that has 256 glyphs
\usepackage{fourier} % Use the Adobe Utopia font for the document - comment this line to return to the LaTeX default
\usepackage[english]{babel} % English language/hyphenation
\usepackage{amsmath,amsfonts,amsthm,amssymb} % Math packages

\usepackage{algorithm, algorithmic}
\renewcommand{\algorithmicrequire}{\textbf{Input:}} %Use Input in the format of Algorithm  
\renewcommand{\algorithmicensure}{\textbf{Output:}} %UseOutput in the format of Algorithm  

\usepackage{listings}
\lstset{language=Matlab}

\usepackage{lipsum} % Used for inserting dummy 'Lorem ipsum' text into the template

\usepackage{sectsty} % Allows customizing section commands
\allsectionsfont{\centering \normalfont\scshape} % Make all sections centered, the default font and small caps

\usepackage{fancyhdr} % Custom headers and footers
\pagestyle{fancyplain} % Makes all pages in the document conform to the custom headers and footers
\fancyhead{} % No page header - if you want one, create it in the same way as the footers below
\fancyfoot[L]{} % Empty left footer
\fancyfoot[C]{} % Empty center footer
\fancyfoot[R]{\thepage} % Page numbering for right footer
\renewcommand{\headrulewidth}{0pt} % Remove header underlines
\renewcommand{\footrulewidth}{0pt} % Remove footer underlines
\setlength{\headheight}{13.6pt} % Customize the height of the header

\numberwithin{equation}{section} % Number equations within sections (i.e. 1.1, 1.2, 2.1, 2.2 instead of 1, 2, 3, 4)
\numberwithin{figure}{section} % Number figures within sections (i.e. 1.1, 1.2, 2.1, 2.2 instead of 1, 2, 3, 4)
\numberwithin{table}{section} % Number tables within sections (i.e. 1.1, 1.2, 2.1, 2.2 instead of 1, 2, 3, 4)

\setlength\parindent{0pt} % Removes all indentation from paragraphs - comment this line for an assignment with lots of text

%----------------------------------------------------------------------------------------
%	TITLE SECTION
%----------------------------------------------------------------------------------------

\newcommand{\horrule}[1]{\rule{\linewidth}{#1}} % Create horizontal rule command with 1 argument of height

\title{	
\normalfont \normalsize 
\textsc{Shanghai Jiao Tong University, UM-SJTU JOINT INSTITUTE} \\ [25pt] % Your university, school and/or department name(s)
\horrule{0.5pt} \\[0.4cm] % Thin top horizontal rule
\huge Introduction to Numerical Analysis \\ HW1 \\ % The assignment title
\horrule{2pt} \\[0.5cm] % Thick bottom horizontal rule
}

\author{Yu Cang \\ 018370210001} % Your name

\date{\normalsize \today} % Today's date or a custom date

\begin{document}

\maketitle % Print the title

%----------------------------------------------------------------------------------------
%	PROBLEM 1
%----------------------------------------------------------------------------------------

\section{Metric Space}
\begin{enumerate}
	\item
		\begin{proof}
			Since there's no elements in $\emptyset$, so $\emptyset$ is open, as $X^\complement = \emptyset$, so $X$ is closed.\\
			$X$ is open as it contains all the elemnets, since $\emptyset^\complement = X$, thus $\emptyset$ is closed. 
		\end{proof}
	\item
		\begin{proof}
			Since
			\begin{equation}
				T \triangleq (U_1 \cap U_2 \cap ... \cap U_m)^\complement = U_1^\complement \cup U_2^\complement \cup ... \cup U_m^\complement
			\end{equation}
			And ${U_i}^\complement$ is open, as $U_i$ is closed.\\
			Thus, $T$ is open, and therefore the intersection is closed. 
		\end{proof}
	\item 
		\begin{proof}
			Since
			\begin{equation}
				T \triangleq (U_1 \cup U_2 \cup ... \cup U_m)^\complement = U_1^\complement \cap U_2^\complement \cap ... \cap U_m^\complement
			\end{equation}
			And ${U_i}^\complement$ is open, as $U_i$ is closed.\\
			Thus, $T$ is open, and therefore the intersection is closed. 
		\end{proof}

\end{enumerate}


%----------------------------------------------------------------------------------------
%	PROBLEM 2
%----------------------------------------------------------------------------------------

\section{Continuity}

\begin{enumerate}
\item $y=\frac{1}{x}$ is continuous over $(0, +\infty)$, but not uniform continuous.
      \begin{proof}
      	Let $x_1 = \frac{1}{n+1}$,  $x_2 = \frac{1}{n}$. 
      	Then $\lim\limits_{n \rightarrow +\infty} \vert x_2-x_1 \vert = \frac{1}{n(n+1)} = 0 $.
      	But $\lim\limits_{n \rightarrow +\infty} \vert y_2- y_1 \vert = 1 \neq 0 $.
      	Thus it is not uniform continuous.
      \end{proof}
\item $y=\sqrt{x}$ is uniform continuous over $(0, +\infty)$ but not Lipschitz continuous.
	  \begin{proof}
	  	As $y\prime = \frac{1}{2\sqrt{x}}$, 
	  	the slope tends towards infty when $x$ approaches 0, 
	  	thus it is not Lipschitz continuous.  
	  \end{proof}
\end{enumerate}


%----------------------------------------------------------------------------------------
%	PROBLEM 3
%----------------------------------------------------------------------------------------

\section{Cardinalty}
\begin{enumerate}
	\item 
		\begin{proof}
			The following function is a one-to-one mapping from N to Z, thus N and Z have the same number of elements.
		
			\begin{equation}
			f(n) = 
				\begin{cases}
					-\frac{n}{2} & \text{n is even}\\
					\frac{n+1}{2} & \text{n is odd}
				\end{cases}
			\end{equation}
			
			Arange the elements in N and Q in the pascal triangle style as below, and it is clear enough to see the one-to-one mapping from N to Q with rationals are grouped according to the sum of dividend and divisor. Thus N and Q have the same number of elements.
			
			\begin{tabular}{c|c}
				\begin{tabular}{rccccccccc}
					&&&&&1\\\noalign{\smallskip\smallskip}
					&&&&2&&3\\\noalign{\smallskip\smallskip}
					&&&4&&&&5\\\noalign{\smallskip\smallskip}
					&&6&&7&&8&&9\\\noalign{\smallskip\smallskip}
					&10&&&&&&&&11\\\noalign{\smallskip\smallskip}
					&&&&&...\\\noalign{\smallskip\smallskip}
				\end{tabular} &
				\begin{tabular}{rccccccccc}
					&&&&&$\frac{1}{1}$ \\\noalign{\smallskip\smallskip}
					&&&&$\frac{1}{2}$&&$\frac{2}{1}$\\\noalign{\smallskip\smallskip}
					&&&$\frac{1}{3}$&&&&$\frac{3}{1}$\\\noalign{\smallskip\smallskip}
					&&$\frac{1}{4}$&&$\frac{2}{3}$&&$\frac{3}{2}$&&$\frac{4}{1}$\\\noalign{\smallskip\smallskip}
					&$\frac{1}{5}$&&&&&&&&$\frac{5}{1}$\\\noalign{\smallskip\smallskip}
					&&&&&...\\\noalign{\smallskip\smallskip}
				\end{tabular}
			\end{tabular}
			
		\end{proof}	
	\item 
		\begin{proof}
			The following function is a one-to-one mapping from [0, 1] to R, thus they have the same number of elements.	
			
			\begin{equation}
				f(x) = 
				\begin{cases}
				-\infty& \text{x=0}\\
				tan[\pi(x-0.5)] & \text{0 < x < 1}\\
				+\infty& \text{x=1}
				\end{cases}
			\end{equation}
			
		\end{proof}
	\item
		\begin{proof}
			Suppose all the real numbers can be listed, and each one is marked as $r_i$, where $i=1,2,3...$. Given a real number $r\prime$ such that its \textbf{i-th} digit is different from that in $r_i$, it's obvious that $r\prime$ is not included in the real numbers listed above, which is contradictory to the hypothesis. Thus real numbers can not be listed and it contains more elements than N.
		\end{proof}
\end{enumerate}

%----------------------------------------------------------------------------------------
%	PROBLEM 4
%----------------------------------------------------------------------------------------

\section{Slides}

\begin{enumerate}
	\item
		Here I give 2 proofs, the first one comes directly from the class, the other from previous reading.(thanks to Ran Yi for pointing out that)
		
		\begin{proof}
			Let $E$ be an inner product space over $\mathbb{C}$, and $u, v\in E$.\\
			Given $Y$ defined as below, where $\lambda \in \mathbb{C}$.
						
			\begin{equation}
				Y = {\vert u-\lambda v \vert}^2
			\end{equation}
			
			It's obvious that $Y\geq0$. Expand the squares according to the definition of inner products, the inequality can be written as below.
			
			\begin{equation}
				<v, v>\lambda^2 -(<u, v>+<v, u>)\lambda + <u, u> \ \ \geq \ 0
			\end{equation}
			
			The LHS of the inequality can be viewed as a quadratic function where $\lambda$ is the variable. Thus, the discriminant is semi-negative definite, which can be written as below.
			
			\begin{equation}
				\Delta = (<u, v>+<v, u>)^2 - 4<u,u><v,v> \ \ \leq \ 0
			\end{equation}
			
			As $<v, u> = \overline{<u, v>}$, the inequality above can be simplified as below.
			
			\begin{equation}
				<u, v>^2 + {\overline{<u, v>}}^2 + 2{\vert <u,v> \vert}^2 \ \ \leq \ 4 \lVert u \rVert^2 \lVert v \rVert^2 \label{cs}			
			\end{equation}
			
			For $x \in \mathbb{C}$, the following equality is justified.
			\begin{equation}
				x^2 + \bar{x}^2 = \frac{1}{2}[(x+\bar{x})^2+(x-\bar{x})^2] = \frac{1}{2}[(2Re(x))^2+(2Im(x))^2] = 2 {\vert x \vert}^2 \label{der}
			\end{equation}
			
			Since $<u, v>$ is an complex number, apply (\ref{der}) into (\ref{cs}) and eliminate the constant 4, the Cauchy-Schwarz inequality is obtained at last.
			
			\begin{equation}
				{\vert <u,v> \vert}^2 \leq \lVert u \rVert^2 \lVert v \rVert^2
			\end{equation}
		\end{proof}
	
		\begin{proof}
			Let $E$ be an inner product space over $\mathbb{C}$, and $u, v\in E$.\\
			
			Given $Y$ defined as below, where $\lambda \in \mathbb{C}$.
			
			\begin{equation}
				\begin{aligned}
					Y & = {\vert u-\lambda v \vert}^2 \\
					  & = <v, v>\lambda^2 -(<u, v>+<v, u>)\lambda + <u, u> \label{lbd}
				\end{aligned}
			\end{equation}
			
			It's obvious that $Y\geq0$ for any $\lambda$. \\
			Given $\lambda$ as
			\begin{equation}
				\lambda = \frac{<u,v>}{<v, v>}
			\end{equation}
			Substitude it into (\ref{lbd}), the inequality reads as
			\begin{equation}
				<u, u><v,v> - <u, v>(<u, v> + <v, u>) + <u, v>^2 \ \ \geq 0
			\end{equation}
			It can be further simplified as
			\begin{equation}
				\begin{aligned}
					<u, u> \cdot <v, v> \ \  & = ||u||^2 \cdot ||v||^2 \\
								        \ \  & \geq \ \ <u, v> \cdot <v, u>\\
								        \ \  & = \ \  <u, v> \cdot \overline{<u, v>}\\
								        \ \  & = \ \ |<u, v>|^2
				\end{aligned}
			\end{equation}
			
			Thus, the Cauchy-Schwarz inequality got proved.
		\end{proof}
		
		

	\item
		\begin{proof}
			$d(x, y)$ is non-negative follows from the definition of metric space.
			\begin{equation}
				\begin{aligned}
					d(x, y) & = \frac{1}{2}(d(x, y) + d(x, y)) \\
							& = \frac{1}{2}(d(x, y) + d(y, x)) &\text{By symmetry}\\
							& \geq \frac{1}{2} d(x, x)         &\text{By triangle inequality} \\
							& = 0                              &\text{By identity of indiscernible}
				\end{aligned}
			\end{equation}
			

		\end{proof}
\end{enumerate}

%----------------------------------------------------------------------------------------
%	PROBLEM 5
%----------------------------------------------------------------------------------------

\section{Linear Algebra}

\begin{enumerate}
	\item
		\begin{proof}
			Suppose $\{u_1, u_2, ... , u_m \}$ forms the basis of Ker($f$), and it can be extended to form the basis of $V_1$, which is denoted as $\{u_1, u_2, ... , u_m, v_1, v_2, ... , v_n \}$. It's clear that the dimension of Ker($f$) is $m$ and the dimension of $V_1$ is $m+n$. So, it is left to prove that the dimension of $V_2$ is $n$.
			
			The dimension of $V_2$ is $n$ means $\{f(v_1), f(v_2), ... , f(v_n)\}$ forms a basis of $V_2$. To see that, let $w$ be an arbitary vector in $V_1$, thus, there exist unique scalars $a_i$, $b_j$ such that 
			\begin{equation}
				w = a_1 u_1 + a_2 u_2 + ... + a_m u_m + b_1 v_1 + b_2 v_2 + ... + b_n v_n
			\end{equation}
			
			The image of $w$ under mapping $f$ is given as below.
			\begin{equation}
				\begin{aligned}
					f(w) & = \sum_{i=1}^{m}a_i f(u_i) + \sum_{j=1}^{m}b_j f(v_j) \\
						 & = \sum_{j=1}^{m}b_j f(v_j)  \ \ \ \  \text{As $f(u_i) = 0$}
				\end{aligned}
			\end{equation}
			
			Thus $\{f(v_1), f(v_2), ... , f(v_n)\}$ spans $V_2$. It is left to show that they are linearly independent, which means there's no redundancy in the list.
			
			Given coefficents $c_i$ such that
			\begin{equation}
				c_1 f(v_1) + c_2 f(v_2) + ... + c_n f(v_n) = 0 \label{cdis}
			\end{equation}
			
			Since $f$ is a linear mapping, (\ref{cdis}) can be grouped as below.
			\begin{equation}
				f(c_1 v_1 + c_2 v_2 + ... + c_n v_n) = 0
			\end{equation}
			
			Thus $c_1 v_1 + c_2 v_2 + ... + c_n v_n \in Ker(f)$. As $\{u_1, u_2, ... , u_m \}$ forms the basis of Ker($f$), there exists coefficients $d_i$ such that
			\begin{equation}
				c_1 v_1 + c_2 v_2 + ... + c_n v_n = d_1 u_1 + d_2 u_2 + ... + d_m u_m \label{diff}
			\end{equation}
			
			Since $\{u_1, u_2, ... , u_m, v_1, v_2, ... , v_n \}$ forms the basis of $V_1$, all the coefficients in (\ref{diff}) should be 0. Thus, (\ref{cdis}) is valid if and only if all the coefficents $c_i$ is 0, which implies that $\{f(v_1), f(v_2), ... , f(v_n)\}$ are linearly independent.
			
			Thus, $\{f(v_1), f(v_2), ... , f(v_n)\}$ forms the basis of $V_2$, and the dimension of $V_2$ is therefore $n$. The rank-nullity theorem written as below is proved.
			\begin{equation}
				dim(V_1) = dim(ker(f)) + dim(V_2)
			\end{equation}
		\end{proof}
	
	\item
		\begin{proof}
			Let $U, V, W$ be vector spaces over the same field $K$, function $f$ be a linear map from $U$ to $V$ and function $g$ be a linear map from $V$ to $W$. It's left to show that the composition of $f$ and $g$, denoted as $h$, which maps a vector in $U$ to $W$, is still a linear map.
			
			Given $u_1, u_2 \in U$ and any scalar $c \in \mathbb{K}$, then
			\begin{equation}
				\begin{aligned}
					h(u_1 + u_2) & = g(f(u_1 + u_2)) \\
								 & = g(f(u_1) + f(u_2)) \\
								 & = g(f(u_1)) + g(f(u_2)) \\
								 & = h(u_1) + h(u_2)
				\end{aligned}
			\end{equation}
			
			\begin{equation}
				\begin{aligned}
					h(c u_1) & = g(f(c u_1)) \\
							 & = g(c f(u_1)) \\
						     & = c g(f(u_1)) \\
							 & = c h(u_1)
				\end{aligned}
			\end{equation}
			
			Thus, the composition of two linear maps is still a linear map.
		\end{proof}
	
	\item
		\begin{proof}
			Let $U, V$ be vector spaces over the same field $K$, function $f$ be linear maps from $U$ to $V$. It's left to show that the inverse of $f$, denoted as $f^{-1}$, which maps a vector in $V$ to $U$ is still a linear map.
			
			By inverse, it means that
			\begin{equation}
				f(f^{-1}(x)) = f^{-1}(f(x)) = x
			\end{equation}
			
			Given $u_1, u_2 \in U$ and any scalar $c \in \mathbb{K}$, denote the counterpart of $u_1, u_2$ in $V$ as $v_1, v_2$ under the linear mapping function $f$. Then
			
			\begin{equation}
				\begin{aligned}
					f^{-1}(v_1 + v_2) & = f^{-1}(f(u_1) + f(u_2)) \\
									  & = f^{-1}(f(u_1 + u_2)) \\
									  & = u_1 + u_2 \\
									  & = f^{-1}(v_1) + f^{-1}(v_2)
				\end{aligned}
			\end{equation}
			
			\begin{equation}
				\begin{aligned}
					f^{-1}(c v_1) & = f^{-1}(c f(u_1)) \\
							      & = f^{-1}(f(c u_1)) \\
								  & = c u_1 \\
								  & = c f^{-1}(v_1)
				\end{aligned}
			\end{equation}
			
			Thus, the inverse of a linear map is still a linear map.
		\end{proof}
\end{enumerate}

%----------------------------------------------------------------------------------------
%	PROBLEM 6
%----------------------------------------------------------------------------------------

\section{Discontinuous linear maps}
\begin{enumerate}
	\item 
		\begin{proof}
			The $k$ times derivative of $f_n(x)$ can be written as below.
			\begin{equation}
				f_n^{(k)}(x) = n^{k-1} sin(nx+ \frac{k\pi}{2}) \ \ \ \ k = 0, 1, 2, 3, ...
			\end{equation}
			It's clear that $f_n^{(k)}(x)$ is continous for any $k$, thus $f_n(x) \in C^\infty(\mathbb{R})$.
		\end{proof}
	\item 
		\begin{equation}
			d f_n(x) = cos(nx) dx
		\end{equation}
	\item
		\begin{proof}
			Let $x_0 = 0, x_1 = \frac{\pi}{2n}$, then
			\begin{equation}
				\lim\limits_{n \rightarrow +\infty} \vert x_1-x_0 \vert = 0
			\end{equation}
			\begin{equation}
				\lim\limits_{n \rightarrow +\infty} \vert df_n(x_1) - df_n(x_0) \vert = \vert 0-dx \vert = |dx| \neq 0
			\end{equation}
			Thus, the differential is not continous when n tends to infty.
		\end{proof}
	\item 
		Sorry, haven't figured out yet...
		
	
\end{enumerate}

%----------------------------------------------------------------------------------------
%	PROBLEM 7
%----------------------------------------------------------------------------------------

\section{pi}
\begin{enumerate}
	\item 
		The main program(see Algorithm\ref{machin}) adopts Machin's formula to calculate $\pi$.
		\begin{equation}
			\pi = 4[4atan(\frac{1}{5}) - atan(\frac{1}{239})]
		\end{equation}
		
		Subroutine(see Algorithm\ref{atan_inv}) calculating $atan(\frac{1}{x})$ is used by the main program. Since
		\begin{equation}
			atan(x) = \frac{1}{1}x - \frac{1}{3}x^3 + \frac{1}{5}x^5 - \frac{1}{7} x^7 + ...
		\end{equation}
		
		Therefore
		\begin{equation}
			atan(\frac{1}{x}) = \frac{1}{x} - \frac{1}{3x^3} + \frac{1}{5x^5} - \frac{1}{7x^7} + ...
		\end{equation}		
		
		\begin{algorithm}
			\caption{Calculation of $\pi$ using Machin's formula}  
			\label{machin}
			\begin{algorithmic}[1]
				\REQUIRE
				None.
				\ENSURE
				The value of $\pi$ with approximation.
				\RETURN $4[4 atan(1/5) - atan(1/239)]$
			\end{algorithmic}  
		\end{algorithm}
		
		\begin{algorithm}
			\caption{Calculation of $atan(\frac{1}{x})$}
			\label{atan_inv}
			\begin{algorithmic}[1]
				\REQUIRE x
				\ENSURE $atan(\frac{1}{x})$
				\STATE $ret \gets 0$
				\STATE $e \gets 1/x$
				\STATE $s \gets - x^2$
				\STATE $c \gets 1$
				\WHILE {$e \neq 0$}
				\STATE $ret \gets ret + e/c$
				\STATE $e \gets e/s$
				\STATE $c \gets c + 2$
				\ENDWHILE
				\RETURN ret
			\end{algorithmic}  
		\end{algorithm}
	
	\newpage
	
	\item
		The matlab code is given as below.
		\begin{verbatim}
			function [pi] = pi_machin()
			pi = 4*(4*atan_inv(5) - atan_inv(239));
			
			function [r] = atan_inv(x)
			r = 0;
			e = 1/x;
			s = - x*x;
			c = 1;
			while (e ~= 0)
			r = r + e/c;
			e = e/s;
			c = c+2;
			end
		\end{verbatim}	
\end{enumerate}


%----------------------------------------------------------------------------------------

\end{document}