%----------------------------------------------------------------------------------------
%	PACKAGES AND OTHER DOCUMENT CONFIGURATIONS
%----------------------------------------------------------------------------------------

\documentclass[paper=a4, fontsize=11pt]{scrartcl} % A4 paper and 11pt font size

\usepackage[T1]{fontenc} % Use 8-bit encoding that has 256 glyphs
\usepackage{fourier} % Use the Adobe Utopia font for the document - comment this line to return to the LaTeX default
\usepackage[english]{babel} % English language/hyphenation
\usepackage{amsmath,amsfonts,amsthm,amssymb} % Math packages
\usepackage{mathrsfs}

\usepackage{algorithm, algorithmic}
\renewcommand{\algorithmicrequire}{\textbf{Input:}} %Use Input in the format of Algorithm  
\renewcommand{\algorithmicensure}{\textbf{Output:}} %UseOutput in the format of Algorithm  

\usepackage{listings}
\lstset{language=Matlab}

\usepackage{lipsum} % Used for inserting dummy 'Lorem ipsum' text into the template

\usepackage{sectsty} % Allows customizing section commands
\allsectionsfont{\centering \normalfont\scshape} % Make all sections centered, the default font and small caps

\usepackage{fancyhdr} % Custom headers and footers
\pagestyle{fancyplain} % Makes all pages in the document conform to the custom headers and footers
\fancyhead{} % No page header - if you want one, create it in the same way as the footers below
\fancyfoot[L]{} % Empty left footer
\fancyfoot[C]{} % Empty center footer
\fancyfoot[R]{\thepage} % Page numbering for right footer
\renewcommand{\headrulewidth}{0pt} % Remove header underlines
\renewcommand{\footrulewidth}{0pt} % Remove footer underlines
\setlength{\headheight}{13.6pt} % Customize the height of the header

\numberwithin{equation}{section} % Number equations within sections (i.e. 1.1, 1.2, 2.1, 2.2 instead of 1, 2, 3, 4)
\numberwithin{figure}{section} % Number figures within sections (i.e. 1.1, 1.2, 2.1, 2.2 instead of 1, 2, 3, 4)
\numberwithin{table}{section} % Number tables within sections (i.e. 1.1, 1.2, 2.1, 2.2 instead of 1, 2, 3, 4)

\setlength\parindent{0pt} % Removes all indentation from paragraphs - comment this line for an assignment with lots of text

%----------------------------------------------------------------------------------------
%	TITLE SECTION
%----------------------------------------------------------------------------------------

\newcommand{\horrule}[1]{\rule{\linewidth}{#1}} % Create horizontal rule command with 1 argument of height

\title{	
\normalfont \normalsize 
\textsc{Shanghai Jiao Tong University, UM-SJTU JOINT INSTITUTE} \\ [25pt] % Your university, school and/or department name(s)
\horrule{0.5pt} \\[0.4cm] % Thin top horizontal rule
\huge Introduction to Numerical Analysis \\ HW5 \\ % The assignment title
\horrule{2pt} \\[0.5cm] % Thick bottom horizontal rule
}

\author{Yu Cang \\ 018370210001} % Your name

\date{\normalsize \today} % Today's date or a custom date

\begin{document}

\maketitle % Print the title

%----------------------------------------------------------------------------------------
%	PROBLEM 1
%----------------------------------------------------------------------------------------
\section{Lebesgue constant for Chebyshev nodes}
\begin{enumerate}
	\item 
		\begin{enumerate}
			\item 
				\begin{proof}
					Denote
					\begin{equation}
						\begin{aligned}
							LHS &\triangleq (x-x_i)l_i(x) \\
							RHS &\triangleq \frac{T_{n+1}(x)}{T_{n+1}'(x_i)}
						\end{aligned}
					\end{equation}
					It is left to prove $LHS = RHS$.\\
					The left part can be written as
					\begin{equation}
						LHS = c_l \omega(x)
					\end{equation}
					where 
					\begin{equation}
						\omega(x) = \prod_{i=0}^{n}(x-x_i)
					\end{equation}
					and
					\begin{equation}
						c_l = \frac{1}{\prod_{k=0, k\neq i}^{n}(x_i - x_k)} 
					\end{equation}

					Since both $LHS$ and $RHS$ are polynomials of order $n+1$, they are equivalent iff. they have same roots and leading coefficients.\\
					On the one hand,as for all $x_i$, where $i = 0, 1, ... , n$
					\begin{equation}
						T_{n+1}(x_i) = cos((n+1)\theta_i) = cos(\frac{2i+1}{2}\pi) = 0
					\end{equation}
					Thus, $LHS$ and $RHS$ have same roots. $RHS$ can therefore be written as
					\begin{equation}
						RHS(x) = c_r \omega(x)
					\end{equation}
					On the other hand, since
					\begin{equation}
						LHS'(x)|_{x=x_i} = (l_i(x) + (x-x_i)l_i'(x))|_{x=x_i} = 1
					\end{equation}
					and
					\begin{equation}
						RHS'(x)|_{x=x_i} = \frac{T'_{n+1}(x)}{T_{n+1}'(x_i)}\Big|_{x=x_i} = 1
					\end{equation}
					Thus, the leading coefficients of $LHS$ and $RHS$ are equal, namely $c_l = c_r$.\\
					Hence, $LHS=RHS$.
				\end{proof}
			\item 
				\begin{proof}
					\begin{equation}
						\begin{aligned}
							T'_{n+1}(x) & = (cos((n+1)arccos(x)))'\\
							            & = sin((n+1)arccos(x)) (n+1) \frac{1}{\sqrt{1-x^2}}\\
							            & = \frac{n+1}{\sqrt{1-cos^2(\theta)}} sin((n+1)\theta) 
						\end{aligned}
					\end{equation}
					As $\theta_k = \frac{2k+1}{2(n+1)}\pi$, thus, $sin(\theta_k) > 0$, and
					\begin{equation}
						T'_{n+1}(x_k) = \frac{n+1}{sin(\theta_k)} sin(\frac{2k+1}{2}\pi) = {(-1)}^k \frac{n+1}{sin(\theta_k)}
					\end{equation}
				\end{proof}
			\item
				\begin{proof}
					As
					\begin{equation}
						T_{n+1}(1) = cos((n+1)arccos(1)) = 1	
					\end{equation}
					Thus
					\begin{equation}
						\begin{aligned}
							\sum_{i=0}^{n}|l_i(1)| & = \sum_{i=0}^{n}\Bigg|\frac{T_{n+1}(1)}{(1-x_i)T'_{n+1}(x_i)}\Bigg| \\
												   & = \sum_{i=0}^{n} \frac{1}{\Big|(1-x_i)T'_{n+1}(x_i)\Big|}\\
												   & = \frac{1}{n+1} \sum_{i=0}^{n} \bigg|\frac{sin\theta_i}{(1-cos\theta_i)} \bigg|\\
												   & = \frac{1}{n+1} \sum_{i=0}^{n} \bigg|\frac{sin\theta_i}{2sin^2(\frac{\theta_i}{2})} \bigg|\\
												   & \geq  \frac{1}{n+1} \sum_{i=0}^{n} cot(\frac{\theta_i}{2})\\
						\end{aligned}
					\end{equation}
				\end{proof} 			
		\end{enumerate}
	\item 
		\begin{enumerate}
			\item 
				\begin{proof}
					According to the mean value theorem, there exists $\theta \in [\frac{\theta_k}{2}, \frac{\theta_{k+1}}{2}]$, s.t.
					\begin{equation}
						\int_{\frac{\theta_k}{2}}^{\frac{\theta_{k+1}}{2}} cot(t) dt = \frac{\theta_{k+1} - \theta_k}{2} cot(\theta)
					\end{equation}
					As $cot'(t) = -\frac{1}{sin^2(t)} < 0$, and $\theta_k \leq \theta \leq \theta_{k+1}$, thus
					\begin{equation}
						cot(\theta) \leq cot(\theta_k)
					\end{equation}
					Therefore
					\begin{equation}
						\int_{\frac{\theta_k}{2}}^{\frac{\theta_{k+1}}{2}} cot(t) dt \leq \frac{\theta_{k+1} - \theta_k}{2} cot(\theta_k)
					\end{equation}
				\end{proof}
			
			\item 
				\begin{proof}
					As $\theta_{k+1} - \theta_k = \frac{\pi}{n+1}$ and according to the result that have been proved above
					\begin{equation}
						\begin{aligned}
							\sum_{k=0}^{n} \int_{\frac{\theta_k}{2}}^{\frac{\theta_{k+1}}{2}} cot(t) dt 
							& \leq \sum_{k=0}^{n} \frac{\theta_{k+1} - \theta_k}{2}cot(\frac{\theta_k}{2})\\
							& = \frac{\pi}{2(n+1)} \sum_{k=0}^{n} cot(\frac{\theta_k}{2})
						\end{aligned}
					\end{equation}
				\end{proof}
			
			\item 
				\begin{proof}
					As $\theta_n = \frac{2n+1}{2n+2}\pi < \pi$, $\theta_{n+1} = \frac{2n+3}{2n+2} > \pi$, and $cot(x)$ is positive over $(0, \frac{\pi}{2})$, while negative otherwise.
					Thus
					\begin{equation}
							\int_{\frac{\theta_0}{2}}^{\frac{\pi}{2}} cot(t) dt 
							\leq \int_{\frac{\theta_0}{2}}^{\frac{\theta_n}{2}} cot(t) dt
							= \sum_{k=0}^{n-1} \int_{\frac{\theta_k}{2}}^{\frac{\theta_{k+1}}{2}} cot(t) dt
					\end{equation}
					Hence
					\begin{equation}
						\int_{\frac{\theta_0}{2}}^{\frac{\pi}{2}} cot(t) dt \leq \frac{\pi}{2(n+1)} \sum_{i=0}^{n}cot(\frac{\theta_i}{2})
					\end{equation}
					(... not fine)
				\end{proof}
			
		\end{enumerate}

	\item 
		\begin{proof}
			\begin{equation}
				\begin{aligned}
					\Lambda_n & = \max_{x \in [a, b]} \sum_{i=0}^{n} |l_i(x)|\\
							  & \geq \sum_{i=0}^{n} |l_i(1)|\\
							  & \geq \frac{1}{n+1} \sum_{i=0}^{n} cot(\frac{\theta_i}{2})\\
							  & \geq \frac{2}{\pi} \int_{\theta_0 / 2}^{\pi/2} cot(t)dt\\
							  & = \frac{2}{\pi} ln(|sin(t)|) \Bigg| _{\theta_0/2}^{\pi/2}\\
							  & = - \frac{2}{\pi} ln(sin(\frac{\theta_0}{2}))\\
							  & \geq \frac{2}{\pi} ln(\frac{2}{\theta_0}) = \frac{2}{\pi} ln(\frac{4(n+1)}{\pi})\\
							  & \ge \frac{2}{\pi} ln(n)
				\end{aligned}
			\end{equation}
		\end{proof}
	
\end{enumerate}


%----------------------------------------------------------------------------------------
%	PROBLEM 2
%----------------------------------------------------------------------------------------
\section{Interpolation}

	
%----------------------------------------------------------------------------------------
%	PROBLEM 3
%----------------------------------------------------------------------------------------
\section{Trigonometric polynomials}


\end{document}